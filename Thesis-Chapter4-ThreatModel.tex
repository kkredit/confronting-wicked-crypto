\chapter{Threat Model}
\label{chap-threatmodel}

In this chapter, develop a threat model using data flow diagrams. May need separate ones for different scenarios (data
at rest on encrypted devices, data in motion using E2E), but want at least one high level diagram encapsulating
everything.

\begin{figure}[h]
    \centering\CaptionFontSize
    \includegraphics[width=\linewidth]{dfds/build/DAR-level-1-crypto-iphone-LDAMSR.png}
    \caption[A DAR EA Proposal]{A DAR EA Proposal based on \cite{savage_lawful_2018}.}
    \label{fig-dfd-lawful}
\end{figure}

\begin{figure}[h]
    \centering\CaptionFontSize
    \includegraphics[width=\linewidth]{dfds/build/DIM-level-1-crypto-E2E-signal-X3DH.png}
    \caption[The Signal X3DH Protocol]{The Signal session creation protocol \cite{marlinspike_x3dh_2016}.}
    \label{fig-dfd-lawful}
\end{figure}

\begin{figure}[h]
    \centering\CaptionFontSize
    \includegraphics[width=\linewidth]{dfds/build/DIM-level-1-crypto-E2E-signal-Double-Ratchet.png}
    \caption[The Signal Double Ratchet Protocol]{The Signal \ac{E2EE} protocol \cite{marlinspike_2016}.}
    \label{fig-dfd-lawful}
\end{figure}

% For analysis of threat actors, see bindenagel_discussion_2017

% Generalize for DAR and DIM, then apply to specific examples. This proves that my analysis is helpful. Ex: apply to email: is a DIM _and_ DAR problem due to the server.

%%%%%%%%%%%%
% Chapter 4, threat models
% (how to be specific enough to be useful but generic enough to apply to many systems? Focus on the EA type (DAR v DIM),
%   then apply to specific systems)
% DAR, DIM threat models (make sure to cover abelson_2015)
% Basic proposals
%  --> mappings onto threat models
% At work: 3 example systems analyzed with this threat model
% 1. IM (E2E IM too?) -- in motion
% 2. Cloud storage -- at rest
% 3. Mobile phone -- at rest, but way different scenario than cloud
%  --> include Ozzie's CLEAR, since it was higher profile
%%%%%%%%%%%%