\chapter{Introduction}
\label{chap-introduction}

Exceptional access is a wicked problem. Characterized by a dynamic technological environment, disagreement about
underlying values, and resistance to a clear solution, the debate on exceptional access refuses to end. This paper will
not attempt to end the debate, but to inform it with additional structure.

\section{Motivation}

\begin{itemize}
    \item In the pursuit of data privacy, regulatory action is part of the threat model--and action is likely
    \item If the technical community acts, we can not only be prepared for regulation, but lead and shape it
\end{itemize}

\section{Premises}

\begin{itemize}
    \item The failure to convince legislative bodies and law enforcement agencies demands that the technical community continue the discussion
    \item The ability to access digital data is important and will be increasingly important for investigations
    \item Various forms of encryption are foundational all elements of online security
    \item Mass surveillance is a grave danger that representative governments are not immune from
    \item International and human rights considerations significantly constrain workable solutions
    \item Transparency to data access is critical for public trust
    % \item In pluralistic societies, government is obligated to respect and balance opposing viewpoints
\end{itemize}

\section{Contribution}

The paper serves to clarify the arguments composing debate and the nature of the technical problem. After reproducing a
brief history of encryption regulation, technology, and conflict, it introduces argument maps and threat models using
data flow diagrams. In the following chapters it analyzes exceptional access arguments in detail and creates a
multi-level threat model, broken down by the significantly different challenges posed by data at rest and data in
motion. The paper concludes with a set of policies and principles to apply to the problem in light of valid arguments
and threats.

% \begin{itemize}
%     \item Argument maps formalizing the debate
%     \item A broad threat model describing the environment
%     \item A structure for organizing future progress
% \end{itemize}
