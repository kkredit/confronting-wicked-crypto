\chapter{Introduction}
\label{chap-introduction}

Certain policy issues are \acp{wicked-prob}. Originally coming from the field of design theory, the term
``\ac{wicked-prob}'' uses ``wicked'' not in the moral sense, but in the malignant, vicious, and tricky sense
\cite{rittel_dilemmas_1973}. Unlike their ``tame'' counterparts which science, engineering, and traditional policymaking
are well equipped to answer, \acp{wicked-prob} lack clear formulations, causes, resolutions, or measurements. Each
attempted solution has permanent and often unintended consequences, and likely exists in a pattern of chronic policy
failure.

\Acl{EA} is a wicked problem \cite{rozenshtein_wicked_2018}. In encryption policy, \ac{EA} is alternative means of
decryption intended for law enforcement use. Characterized by a dynamic technological environment, disagreement about
underlying values, and resistance to a clear solution, the debate on \ac{EA} will not go away. This thesis does not
attempt to end the debate, but to structure and analyze it.


\section{Motivation}
\label{sec-motivation}

The conflict at the heart of the encryption and \ac{EA} debate is that the same cryptographic and design principles that
underlie nearly all digital security also enable an unprecedented degree of individual privacy. Encryption is a
foundational tool to the integrity and confidentiality of all connected computing systems. Its increasing ubiquity in
communications and storage provides clear benefits and should be welcomed by all. In a world where information security
is frightfully poor yet increasingly important, strong encryption's necessity can not be understated. However, the
privacy afforded by certain encryption technologies hampers law enforcement investigations and hides wrongdoing
\cite{cox_2020} \cite{keller_internet_2019}. In a society that cares about bringing wrongdoers to justice, this risk
should not be ignored.

The conversation over encryption is in a stalemate. Governments and law enforcement agencies frequently cite the need to
access encrypted data to perform their duties \cite{ministerial_2018} \cite{intl_2020} \cite{fbi_2020}. Human rights
groups and technical leaders counter that a weakened encryption environment would fatally compromise privacy and
security as we know it \cite{abelson_2015} \cite{eightythree_2017} \cite{ruiz_there_2018}. As the debate continues,
sides have become entrenched, and both sides have engaged in disingenuous attacks against their opponents.

In the long term, increased regulation of the tech industry in general is inevitable, and regulation of encryption is
possible. Despite strong encryption's benefits, regulatory interest in the subject has not subdued. The form that
regulation will take in the field of encryption depends on the good faith efforts made to equitably balance all the
benefits and risks involved in deploying a cryptographic system. If the technical community acts, it can lead and shape
legislation instead of being subjected to it.

Most importantly, in the pursuit of data privacy, regulatory action is part of the threat model. If the technical
community fails to act but regulators move forward, everyone may become subject to technically misguided laws that do
great harm. Bad \ac{EA} policy is just as much a threat as poor passwords. Due to this threat, as long as lawmakers
continue pursuing \ac{EA} regulation, it is the responsibility of the technical community to engage in discussion and
respond to the arguments being presented.

For these reasons, it is important that the technical community keeps moving the debate forward.


\section{Premises}
\label{sec-premises}

This paper accepts and builds on the following premises. These premises are not principles for potential \ac{EA}
designs, but the starting place for discussing the \ac{EA} debate.

\newcommand{\principlesstart}{\begin{enumerate}}
\newcommand{\principle}[2]{\item \textbf{#1.} #2}
\newcommand{\principlesend}{\end{enumerate}}

\principlesstart

\principle{Cybersecurity is critical}{With current levels of reliance on computer information systems, cybersecurity is
critical to the wellbeing of society. Two important elements of cybersecurity are cryptography and system architecture.
Policy which supports security does not undercut cryptographic integrity or require high-risk architectures.}

\principle{Absolute privacy is not an absolute right}{Certain rights supersede the legitimate claims of government, but
privacy in all contexts is not one of them. While individuals under a limited government are entitled to an expectation
of privacy, an absolute right to privacy does not apply in all circumstances. Investigators should have access to some
classes of data. In particular, access to certain classes of digital data is important today and will be increasingly
important in the future.}

\principle{\Ac{EA} is an inherently complex problem}{The factors factors that make \ac{EA} a \ac{wicked-prob} mean that
proposed solutions must be analyzed at many levels. Some these factors are \ac{EA}'s relation to mass surveillance,
potential for abuse, international consequences, economic impact, and need for transparency.}

\principle{Perfection is not the standard}{Wicked policy problems do not have perfect solutions, or even verifiably
optimal solutions, so we cannot use perfection as the standard. To quote the ``insecurity axioms,'' insecurity cannot be
destroyed, only moved around \cite{nrc_schneider_1999}. \Ac{EA}, like all security problems, is a matter of finding the
right balance of risk given the threats under consideration---including the threat of regulatory action.}

\principlesend


\section{Contribution}
\label{sec-contribution}

This thesis aims to clarify the arguments composing debate and the nature of the technical problem. After reproducing a
brief U.S.--focused history of encryption regulation, technology, and conflict, I introduce \acp{argmap} and threat
models using data flow diagrams. In the following chapters, I analyze strategies for tacking \acp{wicked-prob}, map the
\acl{EA} arguments in detail, and create a multi-level threat model, broken down by the significantly different
challenges posed by \acl{DAR} and \acl{DIM}. The thesis concludes with a set of policies and principles to apply to the
problem in light of valid arguments and threats.
