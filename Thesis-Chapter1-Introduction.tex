\chapter{Introduction}
\label{chap-introduction}

Difficult policy matters are \acp{wicked-prob}. Originally coming from the field of design theory, the term
\ac{wicked-prob} uses ``wicked'' not in the moral sense, but in the malignant, vicious, and tricky sense
\cite{rittel_dilemmas_1973}. Unlike their ``tame'' counterparts which science and engineering are well equipped to
answer, \acp{wicked-prob} lack clear formulations, causes, resolutions, or metrics. Each attempt at a solution matters,
has unintended consequences, and often exists in a pattern of chronic policy failure.

\Acl{EA} is a wicked problem. In encryption policy, \ac{EA} is alternative means of decryption reserved for law
enforcement use. Characterized by a dynamic technological environment, disagreement about underlying values, and
resistance to a clear solution, the debate on \acl{EA} refuses to end. This paper will not attempt to end the debate,
but to inform it with additional structure.


\section{Motivation}
\label{sec-motivation}

The conflict at the heart of the encryption and \acl{EA} is that the same cryptographic and design principles that
underlie nearly all digital security also enable an unprecedented degree of individual privacy. Encryption is a
foundational tool to the integrity and confidentiality of all connected computing systems. Its increasing ubiquity in
communications and storage provides clear benefits and should be welcomed by all. In a world where information security
is frightfully poor yet increasingly important, strong encryption's necessity can not be understated. However, the
privacy afforded by certain encryption technologies hampers law enforcement investigations and hides wrongdoing. In a
society that cares about bringing wrongdoers to justice, this risk should not be ignored.

The conversation over encryption is in a stalemate. Governments and law enforcement agencies frequently cite the need to
access encrypted data to perform their duties. Human rights groups and technical leaders counter that a weakened
encryption environment would fatally compromise privacy and security as we know it. As the debate continues, sides have
become entrenched, and both sides have engaged in disingenuous attacks against their perceived opponents.

In the long term, increased regulation of the tech industry in general is inevitable, and regulation of encryption is
possible. Despite strong encryption's benefits, regulatory interest in the subject has not subdued. The form that
regulation will take in the field of encryption depends on the good faith efforts made to equitably balance all the
benefits and risks involved in deploying a cryptographic system. If the technical community acts, it can not only be
prepared for regulation, but lead and shape it.

Most importantly, in the pursuit of data privacy, regulatory action is part of the threat model. If the technical
community fails to act but regulators move forward, everyone may become subject to technically misguided laws that do
great harm. Bad \ac{EA} policy is just as much a threat as poor passwords.

For all these reasons, it is time that the technical community moves the debate forward.


\section{Premises}
\label{sec-premises}

This paper accepts and builds on the following premises, some of which were introduced in \mysec{sec-motivation}. These
premises are not principles for potential \ac{EA} designs, but the starting place for discussing the \ac{EA} debate.

\newcommand{\principlesstart}{\begin{enumerate}}
\newcommand{\principle}[1]{\item \textbf{#1.}}
\newcommand{\principlesend}{\end{enumerate}}

\principlesstart

\principle{The discussion must go on} As long as regulators continue pursuing \ac{EA} regulation, it is the
responsibility of the technical community to engage and respect the valid arguments being presented.

\principle{Cryptography is foundational} Strong cryptography in its various forms is foundational to digital security.

\principle{Investigators need data} The ability to access digital data for investigation and prosecution is important
today and will be increasingly important in the future.

\principle{Mass surveillance is a danger} Mass surveillance, which \ac{EA} makes easier, is a grave danger from which
representative governments are not immune.

\principle{Transparency is necessary} Transparency of the law enforcement and intelligence communities' data access
capability and usage is critical for public trust.

\principle{\Ac{EA} is an international problem} The internet is largely borderless, and international and human rights
considerations significantly constrain workable \ac{EA} solutions.

\principle{Perfection is not the goal} To loosely quote the ``insecurity axioms,'' insecurity cannot be destroyed, only
moved around \cite{nrc_schneider_1999}. \Ac{EA}, like all security, is a matter of finding the right balance of risk.

\principlesend


\section{Contribution}
\label{sec-contribution}

The paper serves to clarify the arguments composing debate and the nature of the technical problem. After reproducing a
brief U.S.-focused history of encryption regulation, technology, and conflict, it introduces argument maps and threat
models using data flow diagrams. In the following chapters it analyzes exceptional access arguments in detail and
creates a multi-level threat model, broken down by the significantly different challenges posed by \ac{DAR} and
\ac{DIM}. The paper concludes with a set of policies and principles to apply to the problem in light of valid arguments
and threats.
