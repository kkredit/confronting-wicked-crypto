\chapter{Introduction}
\label{chap-introduction}

Certain policy issues can be described as ``\acp{wicked-prob}.'' Originally coming from the field of design theory, the
term ``\ac{wicked-prob}'' uses ``wicked'' not in the moral sense, but in the malignant, vicious, and tricky sense
\cite{rittel_dilemmas_1973}. Unlike their ``tame'' counterparts which science, engineering, and traditional policymaking
are well equipped to answer, \acp{wicked-prob} lack clear formulations, causes, resolutions, and measurements. Each
attempted solution has permanent and often unintended consequences, and is likely to exist in a pattern of chronic
policy failure.

\Acl{EA} is a wicked problem \cite{rozenshtein_wicked_2018}. In encryption policy, \ac{EA} is alternative means of
decryption intended for law enforcement use. Characterized by a dynamic technological environment, disagreement about
underlying values, and resistance to a clear solution, the debate on \ac{EA} will not go away. This thesis does not
attempt to end the debate, but to structure and analyze it.


\section{Motivation}
\label{sec-motivation}

The conflict at the heart of the encryption and \ac{EA} debate is this: the same cryptographic and design principles
that underlie nearly all digital security also enable an unprecedented degree of individual privacy. Encryption is a
foundational tool to the integrity and confidentiality of all connected computing systems. Its increasing ubiquity in
communications and storage provides clear benefits. In a world where information security is frightfully poor yet
increasingly important, the necessity of strong encryption cannot be understated. However, the privacy afforded by
certain encryption technologies hampers law enforcement investigations and hides wrongdoing \cite{cox_2020}
\cite{keller_internet_2019}. In a society that cares about bringing wrongdoers to justice, this risk should not be
ignored.

The conversation regarding encryption has reached a stalemate. Governments and law enforcement agencies frequently cite
the need to access encrypted data to perform their duties \cite{ministerial_2018} \cite{intl_2020} \cite{comey_2014}.
Human rights groups and technical leaders counter that a weakened encryption environment would fatally compromise
privacy and security as we know it \cite{abelson_2015} \cite{eightythree_2017} \cite{ruiz_there_2018}. As the debate
continues, many on both sides have become entrenched in their positions and have engaged in disingenuous attacks against
their opponents.

In the long term, increased regulation of the tech industry is inevitable and regulation of encryption is possible.
Despite the benefits of strong encryption, regulatory interest in the subject has not subdued. The form that regulation
will take in the field of encryption depends on the good faith efforts made to equitably balance the benefits and risks
involved in deploying a cryptographic system. If the technical community acts, it can lead and shape legislation rather
than be subjected to it.

Most importantly, in the pursuit of data privacy, regulatory action is part of the threat model. If the technical
community fails to act but regulators move forward, everyone may become subject to technically misguided, harmful laws.
Bad policy is just as much a threat as weak passwords. Due to this threat, as long as lawmakers continue pursuing
\ac{EA} regulation, it is the responsibility of the technical community to engage in discussion and respond to the
presented arguments.

For these reasons, it is important that the technical community keeps moving the debate forward.


\section{Premises}
\label{sec-premises}

This paper accepts and builds on the following premises. These premises are not principles for potential \ac{EA}
designs, but the foundation for discussing the \ac{EA} debate.

\newcommand{\principlesstart}{\begin{enumerate}}
\newcommand{\principle}[2]{\item \textbf{#1.} #2}
\newcommand{\principlesend}{\end{enumerate}}

\principlesstart

\principle{Cybersecurity is critical}{Due to modern culture's reliance on computer information systems, cybersecurity is
critical to the wellbeing of society. Two important elements of cybersecurity are cryptography and system architecture.
Policy that supports security does not undercut cryptographic integrity or require high-risk architectures.}

\principle{Absolute privacy is not an absolute right}{Certain rights supersede the legitimate claims of government, but
privacy in all contexts does not. While individuals under a limited government are entitled to an expectation of
privacy, an absolute right to privacy does not apply in all circumstances. Investigators should have access to some
classes of data. In particular, access to certain classes of digital data is important today and will become
increasingly important in the future.}

\principle{\Ac{EA} is an inherently complex problem}{The factors that make \ac{EA} a \ac{wicked-prob} require proposed
solutions to be analyzed at many levels. These factors include \ac{EA}'s relation to \ac{masssurv}, potential for abuse,
international consequences, economic impact, and need for transparency.}

\principle{Perfection is not the standard}{Wicked policy problems do not have perfect solutions; they do not even have
verifiably optimal solutions. Therefore, we cannot use perfection as the standard. To quote a famous security axiom,
insecurity cannot be destroyed, only moved around \cite{nrc_schneider_1999}. The \Ac{EA} problem, like all security
problems, involves finding the right balance of risk given the threats under consideration---including the threat of
regulatory action.}

\principlesend


\section{Contribution}
\label{sec-contribution}

This thesis aims to clarify the arguments of the debate and the nature of the technical problem. After reproducing a
brief U.S.--focused history of encryption regulation, technology, and conflict, I analyze strategies for tackling
\acp{wicked-prob} and introduce \acp{argmap} and threat modeling with data flow diagrams. In the following chapters, I
map the \acl{EA} arguments in detail and demonstrate the security risks with data flow diagrams and threat analysis,
focusing on one EA proposal in particular. The thesis concludes with paths forward for research and policy that take the
arguments and threats discussed into consideration.
