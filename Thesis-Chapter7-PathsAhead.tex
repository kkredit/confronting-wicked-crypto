\chapter{Paths Forward}
\label{chap-pathsforward}

The interests of security, safety, privacy, and trust are converging on the issue of encryption and \acl{EA} and
creating a \ac{wicked-prob}. Information security is increasingly important as more data, business, and infrastructure
goes online. The technical community is struggling with security; defending against the average attacker is hard, and
against the best attackers is nearly impossible, even for the most motivated and well-resourced organizations.
\Ac{cryptography} plays a fundamental role in security, yet its simultaneous role in enabling unprecedented privacy
means it has controversial side effects.

One of these side effects is encryption-enabled privacy's impact on public safety. Despite a shortage of quantitative
analyses showing its impact on crime---something that would be impossible to fully quantify, after all---encryption
decidedly withholds evidence of wrongdoing from service providers and law enforcement, enabling truly harmful activity
to go unchecked, and decreasing public safety. This a legitimate concern and the primary motivation for \ac{EA}.

However, at the same time as encryption is enabling strong privacy, the broader technological path society is taking is
stripping it away. The ubiquity and insecurity of networked computers has been a bear to privacy, enabling threats
ranging from identity theft, to inappropriate data mining, to \ac{masssurv}. To the extent that encryption mitigates
these, it increases public safety.

Finally, choosing how to respond to these issues---whether with inaction, \ac{EA}, or some alternative---involves
considerable questions of trust. Law enforcement agencies, intelligence agencies, technology makers, individual users,
and the public at large are all players; each have demonstrated untrustworthiness, and different approaches disperse
trust differently.

This complex mix of issues is what makes encryption policy a \ac{wicked-prob}. To merely formulate the problem requires
explanation upon explanation and research upon research. Any formulation is concomitant with a solution, but the outcome
of a solution cannot be accurately forecasted or evaluated in isolation. The underlying conditions are constantly in
flux, affected by external forces as well as every policy action and inaction.

In this thesis I propose a method to confront \acp{wicked-prob}, a modified \ac{OODAloop} that emphasizes collaborative
debate between individuals diverse in background and values, facilitated in part by argument maps, and focusing on
specific proposals. I try to capture that here by surveying the debate, mapping the arguments, and ultimately performing
threat model analysis of a specific \ac{EA} proposal for \acl{DAR}. I now turn to conclusions and paths forward for both
the technical and policymaking communities.


\section{Conclusions}

Regarding the arguments analyzed in \mychap{chap-arguments}, while neither side of the \ac{EA} debate fully incorporates
the whole picture, the anti-\ac{EA} argument is currently stronger. Encryption poses a present danger to public safety
in certain respects. The bogeymen of terrorists and criminals using encryption to evade the law are real---in the case
of those peddling and viewing \acl{CSAM}, shockingly so \cite{keller_internet_2019}. However, the argument that
\ac{encryption} is creating circumstances that are to blame for \acl{LE} failure is not currently strong enough to
justify an \ac{EA} mandate. For \ac{DIM}, the primary issues are poor coordination and training \cite{carter_2018}; for
\ac{DAR}, commercial \ac{lawful-hacking} tools currently have the upper hand on mobile phone manufacturers, meaning
\acl{LE} can get into almost any mobile device they acquire \cite{koepke_2020}. Additionally, strong \ac{encryption}
plays a key role in protecting security and privacy, public goods that other technological developments often fail to
provide.

Though the pro-\ac{EA} argument is ultimately not strong enough to justify an \ac{EA} mandate, incorporating it to the
technologist's point of view teaches important lessons. One is that wholistic cryptosystem threat models need to include
two non-traditional threat actors, the malicious user and threatening lawmaker. Encryption lends power to its user, and
this power can be abused. Technology makers should treat abuse of their creations as a threat, and take mitigating
steps. Governments are naturally sensitive to shifts in power, and thus are interested in the impacts of encryption;
lawmakers represent a unique threat in that they can outlaw certain types of encryption entirely if they deem them too
dangerous. The threat of an \ac{EA} mandate should be treated as a true threat in the security modeling sense.

Another lesson is that the current situation is temporary. This is necessarily true of \acp{wicked-prob}, but especially
true for encryption policy in the context of rapidly evolving technology and a vulnerability arms race between device
manufacturers and lawful hackers. The facts of the case are bound to change, and the conclusion that \ac{EA} is
unjustified may change as well.

The additions to the threat model and the instability of present circumstances mean that additional research into
\ac{EA} is fully justified. It belongs as just a component of the effort that society should be putting towards
addressing the broader set of issues. Fortunately, developments in secure hardware mark real progress in the search for
acceptably secure \ac{EA}. Secure hardware technology is clearly imperfect, and alone does not resolve the numerous
challenges that full-fledged \ac{EA} proposals must overcome, but device escrow for \acl{DAR} such as that described in
Savage's proposal offers transparency, protection from \ac{masssurv}, and \acl{LE} utility while making progress on
security.

The conclusions I draw do not settle the issue. However, when dealing with \acp{wicked-prob}, meaningful debate and
discussion counts as progress even when it doesn't seem like it. \Acp{wicked-prob} cannot be analyzed into resolution;
grasping the problem alone constitutes much of the work towards solving it. The conclusions above do not bring closure,
but they should bring more clarity, and its clarity that counts.


\section{Paths Forward: Technology Makers}

The technical community needs to accept and deal with the enormous role technology is playing in current policy issues,
such as encryption, automation, and misinformation. Government sometimes suggests that the tech world is not trying hard
enough, and that ``nerding harder'' will result in solutions. The tech world is absolutely right that nerding harder
will not work on these issues---because they are \acp{wicked-prob}---but that does not mean they cannot deal with them.
Tech is used to addressing tame problems, but it is time that they turn their massive financial and intellectual capital
towards strategies designed for \acp{wicked-prob}. This means including people of diverse experiences, backgrounds, and
values in decision making processes that steer investment and design decisions. It means creating products and business
models that are responsive to evolving conditions. It means considering the ways their creations will be abused.

Work on \acp{wicked-prob} spins out smaller problems, some tame and some wicked yet manageable (see
\myfig{fig-policy-ooda-process}). Aside from making the changes to tackle \acp{wicked-prob} head on, the technical
community should also work on the tame sub-problems, a few of which are listed here.

\newcommand{\taskstart}[0]{\begin{itemize}}
\newcommand{\taskitem}[2]{ % Name, description
    \item \textbf{#1} \nopagebreak

    \vspace{0.5\baselineskip} \parbox{\linewidth}{#2} \vspace{0.5\baselineskip}
}
\newcommand{\taskend}{\end{itemize}}

\taskstart
    \taskitem{Basic security research}{

\ac{EA} is only applicable when security is actually strong enough to keep attackers out in the first place, which is
not the case for current device encryption. To some extent, this is a security ergonomics problem more than a technical
problem: cryptographers can make device encryption incredibly strong today, but users will not let them because of its
adverse effects on usability. Solving the strong yet usable device encryption problem will make the most promising class
of \ac{EA}, device key escrow, relevant. Improving basic security across the board will make any \ac{EA} proposal more
tenable.

}

    \taskitem{Research technical transparency mechanisms}{

One of the largest non-technical risks of \ac{EA} is abuse by jurisdictions with poor rule of law, or where law
insufficiently respects civil rights in the first place. This is the motive for transparency as a goal of the threat
model. Device key escrow can achieve partial transparency through physical possession and disassembly requirements.
Ideally, usage of \ac{EA} mechanisms would be fully auditable to the public through inherent technical mechanisms. This
would mark an improvement for \acl{DAR} and is a necessity for progress on \acl{DIM}.

}

    \taskitem{Research distributed trust mechanisms}{

Centralized trust in \ac{EA} proposals represent their greatest risk, whether it is in the form of a database of keys or
a control center with the power to add ``ghost users'' to conversations. Knowing that insecurity (and trust) cannot be
destroyed, only moved around, the goal is a distribution of trust that allows the system to function without large
amounts of trust assigned to any single party. This would contribute to transparency and greatly reduce the risk profile
of an \ac{EA} proposal overall.

}
\taskend

Fortunately, research in these areas is occurring. More is necessary. Some \ac{EA} proposals include built in
transparency and distributed trust mechanisms \cite{goos_oblivious_1996} \cite{phan_key_2017}
\cite{servan_schreiber_jje_2020}; these proposals' weakness is that the approaches are insufficiently understood to be
trusted with the high stakes of implementation. While these problems remain unsolved, technologists must continue to
provide clear security advice and be vigilant against bad technology policy.


\section{Paths Forward: Policymakers}

% Prof. Walsh:
% - Liked the lessons for techies; this distills the same problems for policymakers
% - Enjoyed perspective of the "hawkish lawmaker" and how that connects to LAED Act and fallacious argmap
% - Would love to see this in the conclusion -- tie it all together and make sure it hits home for policymakers

Government has not been silent on the issue of encryption and lawful access. The dynamics created by the executive,
legislative, and judicial branches ensures that the regulatory environment will evolve with changing federal policies,
new laws, and rulings that refine implementation existing law. The judiciary will indeed shape encryption policy through
its rulings, and is correct to be reevaluating current Fourth Amendment jurisprudence in light of rapidly changing
circumstances. Still, the primary actors capable of consciously taking part in strategic policymaking will be the
executive and legislative branches.

Executive and legislative policymakers must strategically approach encryption policy as a \ac{wicked-prob}. Industry has
failed at analyzing the problem away, and government has failed at muddling the problem away. It is true that
incrementalism has safely avoided dangerous outcomes such as blanket \ac{EA} mandates, but it has also resulted in a
research-stifling high-tension debate and circumstances of poor utilization of current capabilities mixed with
unregulated use of invasive \ac{lawful-hacking} tools. Policymakers and law enforcement agencies need to be forthcoming
with the real challenges they face. They are important contributors to the diverse groups that can iteratively refine
the problem definition and possible solutions. Just like the technical side, the policy side of the solutions must be
responsive to evolving conditions. Most importantly, any initiative needs follow through, since a single \ac{OODAloop}
iteration will not be enough to tame this issue.

Based on the analysis performed thus far, there are a few policy changes that would be good first steps to take today.

\taskstart
    \taskitem{Take the low-hanging fruit}{

Law enforcement is often ill-equipped to handle digital evidence and coordinate with service providers using existing
access mechanisms \cite{carter_2018}. Justice Department task forces to combat \acl{CSAM} are underfunded and
understaffed \cite{keller_internet_2019}. Improving the effectiveness of existing capabilities and programs carries no
risk and guaranteed results.

}

    \taskitem{Regulate current lawful hacking}{

\Ac{lawful-hacking} using commercial mobile device forensic tools has become a common practice but invokes serious
privacy and civil rights concerns \cite{koepke_2020}. Regardless of any long-term strategy to encryption and lawful
access policy, \ac{lawful-hacking} must be constrained. Michigan's 2020 constitutional amendment represents one positive
development: it explicitly protects electronic data and communications from search without a warrant
\cite{ballotpedia_2020}. Protections like this enable a non-abusive use of current \ac{lawful-hacking} capability.

}

    \taskitem{Fund research}{

The government wants technical advancement in \ac{EA} technology, but there is very little financial incentive for
private companies to research \ac{EA}, and the threatening climate is stifling academic research. If the federal
government is going to compare building secure \ac{EA} to going to the moon \cite{cushing_moon_2018}, then they should
fund it like the Apollo program. In order to be welcomed, such funding should be accompanied with a binding statute
excluding the possibility of any premature \ac{EA} mandate.

}
\taskend


\section{Decision and Action}

The goal of this thesis is to clarify encryption and \ac{EA} as a \ac{wicked-prob}, to break down the debate, and to
evaluate the security of current \ac{EA} proposals. I aim to move the discussion forward with the ideas of how to tackle
\acp{wicked-prob}, additions to the traditional cryptosystem threat model, and security analysis of a modern proposal.
In the language of the \ac{OODAloop}, this study is observation and orientation; it is up to technologists,
policymakers, and common people to decide and act. Contending with these issues is difficult on many levels, but the
future of security, safety, and privacy depend on coming together to find a good answer where no perfect answer exists.



%%%%%%%%%%%%%%%%%%%%%%%%%
% OLD NOTES for reference
%%%%%%%%%%%%%%%%%%%%%%%%%

% In this chapter, describe actions to take in discussion, regulation, and research.

% \section{Legislative Priorities}

% \begin{itemize}
%     \item Reject EA until technical capabilities are ready \cite{varia_2018}
%     \item Be forthcoming with data so that we know what to work on \cite{devlin_2018}
%     \item Don't prescribe technical methods, but be clear about principles \cite{matyas_incommensurability_2018}
%         (e.g., ECPA became outdated because was too tied to specific technology \cite{shamsi_2011})
%     \item Fund the research \cite{varia_2018} (like \cite{goss_hr2616_1999})
% \end{itemize}

% \section{Technical Research}

% \begin{itemize}
%     \item Acknowledge the moral character of cryptography \cite{rogaway_moral_2015}
%     \item Do the research
%     \item Focus on principles \cite{levy_robinson_2018}
%     \item Be clear on definitions \cite{varia_2018}, but don't expect govt to provide those (unlike
%             \cite{abelson_2015}), and clearly define their interfaces with principles
%             \cite{matyas_incommensurability_2018}
% \end{itemize}
