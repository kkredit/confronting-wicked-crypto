\chapter{Threat Model}
\label{chap-threatmodel}

In this chapter, develop a threat model using data flow diagrams. May need separate ones for different scenarios (data
at rest on encrypted devices, data in motion using E2E), but want at least one high level diagram encapsulating
everything.


\section{Developing a Threat Model}

% For a wholistic threat model, include both the threats of government (legislation) and of criminal abusers of your
% system.

% For analysis of threat actors, see bindenagel_discussion_2017


\section{Basic Data at Rest}

\begin{figure}[h]
    \centering\CaptionFontSize
    \includegraphics[width=\linewidth]{dfds/build/DAR-level-1-crypto-iphone.png}
    \caption{The basic encrypted mobile phone data flow diagram.}
    \label{fig-dfd-dar}
\end{figure}


\section{The LDAWMSR Proposal}

\begin{figure}[h]
    \centering\CaptionFontSize
    \includegraphics[width=0.81\linewidth]{dfds/build/DAR-level-1-crypto-iphone-LDAMSR.png}
    \caption{The LDAWMSR data flow diagram.}
    \label{fig-dfd-ldawmsr}
\end{figure}


\section{Discussion of Threats}



%%%%%%%%%%%%
% Old notes for reference
% -----------------------
% Chapter 4, threat models
% (how to be specific enough to be useful but generic enough to apply to many systems? Focus on the EA type (DAR v DIM),
%   then apply to specific systems)
% DAR, DIM threat models (make sure to cover abelson_2015)
% Basic proposals
%  --> mappings onto threat models
% At work: 3 example systems analyzed with this threat model
% 1. IM (E2E IM too?) -- in motion
% 2. Cloud storage -- at rest
% 3. Mobile phone -- at rest, but way different scenario than cloud
%  --> include Ozzie's CLEAR, since it was higher profile
%%%%%%%%%%%%
