\chapter{Threat Model}
\label{chap-threatmodel}

This chapter consists of a threat model and analysis of a specific device key escrow \ac{EA} proposal, Stefan Savage's
``Lawful Device Access without Mass Surveillance Risk: A Technical Design Discussion'' \cite{savage_lawful_2018}.
Lacking a concise name, this proposal will be referred to by its acronym, \ldawmsR. Note that the creators of the
proposal intentionally chose the phrase ``lawful access'' over ``\acl{EA}'' in order to emphasize the role of legal
process and dispel the negative stigma that accompanies \ac{EA}. I refer to \ldawmsr as an \ac{EA} proposal, but use the
term entirely in the spirit of their paper.

% TODO: remove this paragraph if I end up changing EA to LA


\section{Developing a Threat Model}

The essential threat modeling questions are ``what are we building'' and ``what can go wrong.'' The first step to
answering those questions is to establish goals for the former and a list of threats in anticipation of the latter. The
\ldawmsr proposal describes a threat model against which it was designed. This model takes cues from that one but is
intended to be more generally applicable for any \ac{DAR} \ac{EA} proposal.

\newcommand{\modelstart}[0]{\begin{itemize}}
\newcommand{\modelitem}[2]{ % Name, description
    \item \textbf{#1} \nopagebreak

    \vspace{0.5\baselineskip} \parbox{\linewidth}{#2} \vspace{0.5\baselineskip}
}
\newcommand{\modelend}{\end{itemize}}

% \newcommand{\modelstart}{\begin{itemize}}
% \newcommand{\modelitem}[2]{\item \textbf{#1.} #2}
% \newcommand{\modelend}{\end{itemize}}

\subsection{Goals}

The goals of an \ac{EA} cryptosystem are taken from the metrics of \mychap{chap-arguments}. Their justification stems
from the preceding arguments. Each goal is listed according to priority, though none are absolute.

\modelstart

\modelitem{Security}{The system should minimize increased security risk with reference to the current state of the art.
Risk is considered both with respect to individual devices and to the integrity of the entire class of devices.}

\modelitem{Protection of civil liberties}{The system should include measures to prevent mass surveillance and abuse by
political regimes that violate civil rights. This includes technologically enforcing the fundamental right to privacy
but upon warranted search pursuant to the rule of law.}

\modelitem{Transparency}{The system should be auditable by the public to enable trust that this significant power is
being used responsibly. In the case of \ac{DAR}, the subject of the search should be aware of the access.}

\modelitem{Law enforcement utility}{The system should actually be useful to law enforcement in terms of reliability,
speed, and cost.}

\modelitem{Economic impact}{The system should consider global competitiveness of devices using the cryptosystem as well
as the distribution of costs associated with its development and administration. Even though this is a security-oriented
threat model, architectural decisions determine the system's economics, and economics determines the system's ultimate
feasibility.}

\modelend

\subsection{Threat Actors}

\ac{EA} threat models necessarily include the full cast of traditional threat actors.

\modelstart

\modelitem{Criminals: ordinary, organized, cyber}{Criminals represent a threat to the \ac{EA} system by trying to use it
to gain access without authority. Ordinary thieves may want to spy on a target or steal and resell phones. They will
have physical access but only simple tools, using ordinary device interfaces. Organized criminals will have the same
motives but considerably more advanced technical capabilities, exerting full control over all interfaces. Cyber
criminals have a wide spectrum of motivations and capabilities. Their significant contribution to the threat landscape
is that they will target the backend systems that operate the \ac{EA} system in order to steal and likely sell whatever
access capabilities they can find.}

\modelitem{Insider threat}{The insider threat is someone within the \ac{EA} recovery process, whether at the law
enforcement agency or a vendor, that seeks to abuse the system for their own purposes. They may represent the
motivations and capabilities of any of the criminal categories, but they begin with privileged access somewhere in the
process.}

\modelitem{Foreign intelligence}{Foreign intelligence agencies and \acp{APT} are always looking for means to escalate
access and tamper with and steal information. They have world-class expertise and computing power, which they will apply
to both individual devices and backend systems.}

\modelitem{Authoritarian regimes}{States that do not respect lawful process will try to compel administers of an \ac{EA}
system to provide access upon demand. Legal coercion may or may not be combined with state-backed hacking capability.}

\modelend

To make sense of an \ac{EA} proposal in the face of the stated goals and such a formidable list of threat actors, one
must include two non-traditional threats. To do so is to step back and look at the situation from a broader view than
security experts usually take.

\modelstart

\modelitem{The platform abuser}{Why bother with \ac{EA} at all? Because the device user may actually be a threat in the
larger safety context. The end user is not the most important thing. This is a crucial point.

All technologies have embedded values \cite{rogaway_moral_2015}, and at the top of this value system is some most
important ``thing.'' Traditional device security threat models make the end user the most important thing. This is a
very good rule. It leads to good security priorities and more importantly, honors the user's rights and autonomy.

To stray from this rule is dangerous, but---make no mistake---it is already done on a regular basis. Software that can
force updates values the vendor's judgement over the user's (even if it is a good idea). Software that enforces Digital
Rights Management values corporations' rights over the user's (sometimes to extremes \cite{eff_2020}). Software that
doesn't encrypt user communications at all says the service provider's ability to scrape personal information to create
targeted ads is more important than the user's privacy. Software that enables \ac{EA} values lawful investigatory
process over the user's ultimate privacy. Subversion of the user's will is not a pleasant concept, but a spade is a
spade.

In the broader threat model, the platform abuser is someone who is using the device encryption for cover of illicit
activities. They represent a threat to public safety.}

\modelitem{Hawkish lawmakers}{If one's very highest priorities are security and privacy for society as a whole, then
aggressive lawmakers that have the power to \ii{mandate} arbitrary data access represent perhaps the largest threat of
all. Crypto-anarchists can encrypt their data above the compromised layer all they want, but the vast majority of
individuals using default products and all above-the-table organizations will abide by U.S. law.

The hawkish lawmaker is a threat to the security of a cryptosystem (\ac{EA} or not) in that they may judge it too
inhibitive to law enforcement and outlaw its very use, mandating something weaker in its place.}

\modelend

\subsection{Out of Scope}

The following considerations are out of scope for this threat model.

\modelstart

\modelitem{Encyption workarounds}{Encryption workarounds are always in play for essentially any threat actor at any
time. Discussion for defending against those is out of scope.}

\modelitem{\ac{EA} workarounds}{The user is capable of evading \ac{EA} mechanisms by encrypting data at a higher level
in the tech stack, e.g. by encrypting files before saving them to disk. This has already been discussed, and it out of
scope for evaluating the \ac{EA} system itself.}

\modelitem{Manipulation of internal hardware}{Hyper-advanced hardware analysis and manipulation is a real threat from
foreign intelligence agencies. Defending against these attacks is a separate problem.}

\modelitem{Breakdown in rule of law}{This threat model assumes that the legal process used to obtain the warrant is
valid. Additionally, all forms of surveillance, even when protected from mass surveillance, require oversight.
Transparency via auditability, preferably ensured via technical mechanism, is a goal for good \ac{EA} proposals.
However, rule of law is an absolute prerequisite for which no \ac{EA} mechanism can compensate.}

\modelend

% What are we building?: proposal DFDs
% What can go wrong?: elicitation of threats


\section{Basic Data at Rest}

Answering ``what are we building'' and comparing it with the state of the art requires an understanding of the state of
the art. Apple iPhones are among the most secure consumer mobile phones, and serve as the point of reference for
\ldawmsR. \myfig{fig-dfd-iphone} is a \ac{DFD} depicting the iPhone's unlock and decryption process according to their
public documentation \cite{apple_2020}.

\begin{figure}[h]
    \centering\CaptionFontSize
    \includegraphics[width=\linewidth]{dfds/build/DAR-level-1-crypto-iphone.png}
    \caption{The basic encrypted mobile phone data flow diagram.}
    \label{fig-dfd-iphone}
\end{figure}

This \ac{DFD}, and all that follow, is at a specific level of abstraction, simplified in the spirit of ``all models are
wrong, but some are useful.'' The \acp{DFD} are intended to be useful for communication and threat elicitation only, but
are accurate in that capacity.

As the diagram illustrates, several steps occur between the user entering her \ac{PIN} and unlocking of the device. The
operating system forwards the \ac{PIN} to the Secure Enclave Processor, which combines it with the device-specific
Hardware \ac{UID} to generate the Class key. The Class key is used to decrypt the Volume key, which is finally handed to
the Encryption Module to perform encryption and decryption at a hardware level between the operating system and actual
encrypted storage.

The \ac{UID} is not depicted in key storage because it is actually burned into the secure processor silicon, and cannot
be read by software. The Volume key is stored encrypted by the Class key. When a user updates his \ac{PIN}, the
encrypted Volume key in storage is replaced with the copy encrypted with the new Class key. Application processors never
handle any keys or secret information besides the \ac{PIN}, and due to inline encryption in the Encryption Module,
persistent storage never handles and \ac{plaintext} data.


\section{The \ldawmsr Proposal}

\ldawmsr introduces device key escrow and an unexposed hardware interface that mediates \ac{EA} requests. When prompted,
the hardware interface produces a Device ID. The law enforcement officer (assuming proper circumstances) receives this
Device ID and reports it to her agency. The agency signs a request including the Device ID and presents it to the
vendor's access compliance department. The vendor authenticates the request, audits the corresponding legal documents,
and looks up the corresponding Device Seal key. This key is encrypted using the same agency's public key and returned to
the agency. The agency decrypts the Device Seal key and issues it to the agent, who enters it through the special
interface. The Secure Enclave Processor uses the private Device Seal key to decrypt the Class key, which was saved in
key storage encrypted by the device's public Device Seal key, and the unlock process proceeds as normal from there. Each
of these steps is analyzed in closer detail.

\begin{figure}[p]
    \centering\CaptionFontSize
    \includegraphics[width=0.81\linewidth]{dfds/build/DAR-level-1-crypto-iphone-LDAMSR.png}
    \caption{The LDAWMSR data flow diagram.}
    \label{fig-dfd-ldawmsr}
\end{figure}


\section{Discussion of Threats}

\section{Response to Frameworks}

% How does it answer (or not) committee_decrypting_2018's framework questions? What needs improvement?



%%%%%%%%%%%%
% Old notes for reference
% -----------------------
% Chapter 4, threat models
% (how to be specific enough to be useful but generic enough to apply to many systems? Focus on the EA type (DAR v DIM),
%   then apply to specific systems)
% DAR, DIM threat models (make sure to cover abelson_2015)
% Basic proposals
%  --> mappings onto threat models
% At work: 3 example systems analyzed with this threat model
% 1. IM (E2E IM too?) -- in motion
% 2. Cloud storage -- at rest
% 3. Mobile phone -- at rest, but way different scenario than cloud
%  --> include Ozzie's CLEAR, since it was higher profile
%%%%%%%%%%%%
