\chapter{EA Debate Arguments}
\label{chap-arguments}

This chapter will analyze the \ac{encryption} debate using \acp{argmap}. The maps facilitate comprehension of the
debate, but there are two important shortcomings of the maps presented here. First, argument maps should be used simply
as tools in the collaborative pursuit of the problem definition and solution. It is not a replacement for debate arising
from live discussion. Second, \acp{argmap} may deceptively portray the strength of an argument. Argument nodes do not
indicate their strength or validity. If ten unsound arguments were presented against one indisputable argument, the side
with more arguments would, regardless of their integrity, appear stronger. With those qualifications in mind, let us
begin by examining the factors at the center of the conflict.


\section{Contributing Factors}
\label{sed-arg-contrib}

\myfig{fig-args-contrib} maps the central arguments in the EA debate. As previously stated, \ac{encryption}'s dual
contributions to information security and radical privacy are central to the debate. Of course, there is only conflict
when radical privacy is perceived as a negative. The right to privacy is both a strongly held value and an enshrined
legal principle. Many use privacy concerns to argue that \ac{EA} is socially undesirable. The Snowden revelations
\cite{landau_making_2013} unveiled the scale of privacy-eroding U.S. government surveillance enabled by technological
changes, terrorism-motivated policies, and weak oversight \cite{shamsi_2011}. Privacy violations are an abstract concern
to many, but for those most vulnerable, they are frighteningly concrete. For example, mobile phone surveillance malware
enabled the Saudi government to capture and murder journalist Jamal Khashoggi for expressing dissent against the crown
prince \cite{liebermann_2019}. At a societal level, the mere presence of surveillance changes behavior and suppresses
free speech \cite{rogaway_moral_2015}, and government violations of the law degrade institutional trust.

\begin{sidewaysfigure}
  \centering\CaptionFontSize
  \includegraphics[width=\linewidth]{arguments/build/comprehensive.contrib.pdf}
  \caption{Contributing Factors to the EA Debate}
  \label{fig-args-contrib}
\end{sidewaysfigure}

\Acl{LE} argues that encryption handcuffs its investigational capacity. The \ac{FBI} brands strong encryption as
``warrant-proof'' and states that ``the government often cannot obtain the electronic evidence necessary to investigate
and prosecute threats to public and national safety'' \cite{fbi_2020}. The pro-\ac{EA} argument is typically rooted in
public safety: encryption makes it too difficult for investigators to access the evidence needed to catch and convict
law-breakers. Five-eyes nations have regularly published joint statements expressing their frustration with encryption
\cite{ministerial_2018} \cite{goodale_2017}. They were joined in 2020 by India and Japan in a statement emphasizing the
growing danger of \ac{CSAM} \cite{intl_2020}. Child pornography is one of the original \ac{horsemen} used to scare
people into supporting backdoors \cite{schneier_scaring_2019}, but it is also a problem growing to astonishing
proportions. A 2019 investigation by the \textit{New York Times} quoted a law officer's estimate that 400,000 New
Jerseyans, more than 4\% of the state's population, have violated child exploitation material laws
\cite{keller_internet_2019}.

Unfortunately, these shocking claims have not been independently confirmed with hard data. Intelligence agencies have
historically misrepresented statistics and overstepped their bounds \cite{johnson_congressional_2004}
\cite{shamsi_2011}. This undercuts their claims that they need \ac{EA} and has diminished institutional trust. For
example, the \ac{FBI} has already been found to exaggerate the number of mobile devices it could not access due to
\acl{DE} \cite{devlin_2018}. As Rozenshtein points out, ``It is impossible to know the precise extent to which
encryption frustrates law-enforcement investigations, both because law-enforcement agencies are only beginning to
collect accurate statistics, and because one can never be sure of how an investigation would have proceeded in the
absence of encryption'' \cite{rozenshtein_wicked_2018}. However, it is still crucial to have an accurate depiction of
the problem in order to come to justified and helpful solutions. This argument is analyzed further in the next section.

Information security is the remaining central issue in the debate. As described in \mysec{sec-crypto-basics},
\ac{cryptography} plays a foundational role in nearly every aspect of security. Past \ac{EA} regulation efforts sought
to compromise the cryptographic foundations of encryption; however, this is seen as too risky today.
\mysec{sec-tech-approaches} introduced several alternative technical approaches to \ac{EA}, but experts still emphasize
that current systems cannot securely provide the level of access \acl{LE} asks for \cite{abelson_2015}
\cite{abelson_risks_1997}. Security was an afterthought in early computing and networking designs. The field of
cybersecurity is still equal parts art and science; mandated \ac{EA} would put a heavy burden on a field still finding
its legs.

Finally, it is arguable that \ac{EA} could have positive alternative applications. It could enable malware scanning,
password recovery, or administrator access in a business setting. Typically these applications have other solutions,
however, and unnecessary requirements would weaken \ac{EA} systems.


\section{Going Dark vs. The Golden Age}

Much of the disconnect between government and the technical community is the result of disagreement over whether law
enforcement has \ii{too little} or \ii{too much} access to data. The argument for \ii{too little} is well-represented by
proponent former \ac{FBI} Director James Comey in his 2014 speech titled ``Going Dark'' \cite{comey_2014} (though he did
not coin the phrase \cite{swire_encryption_2011}). The argument is presented in \myfig{fig-arg-going-dark}.

\begin{figure}[t!]
  \centering\CaptionFontSize
  \includegraphics[width=\linewidth]{arguments/build/comprehensive.goingdark.pdf}
  \caption{A ``Going Dark'' Argument Map}
  \label{fig-arg-going-dark}
\end{figure}

According to the "Going Dark" argument, privacy and safety are both desirable goods, but they conflict with one another;
therefore, they must be balanced. Individuals must sacrifice some of their personal good of privacy to allow for the
public good of safety. Widespread encryption tips the scales too far towards privacy, upsetting the balance that existed
before. \ac{EA} restores the balance and enables law enforcement agencies to fulfill their duty to protect the public.

The ``Going Dark'' argument has a few shortcomings. It oversimplifies the relationship between privacy and safety by
depicting it as a zero-sum conflict. In reality, privacy does not always diminish safety, but sometimes enhances safety.
For example, data privacy makes crimes such as stalking and identify more difficult. The ``Going Dark'' argument also
categorizes privacy as a personal good and safety as a public good. However, privacy can also be a public good when it
precludes \ac{masssurv}. This argument is explored more deeply in the ``Golden Age for Surveillance'' map.

Most importantly, the ``Going Dark'' argument relies on the premise that law enforcement investigations fail due to
encryption. As described in \mysec{sed-arg-contrib}, there is no conclusive data regarding the extent to which
encryption is to blame for failed cases. The highest-profile case of \ac{the-cw2}, Apple vs. FBI, ended when the
\ac{FBI} broke into the phone and found nothing of value. Tellingly, the examples cited in Comey's 2014 speech were not
cases in which encryption inhibited investigations, but were cases in which encryption \ii{could have} inhibited
investigations, had it been a factor. \cite{comey_2014}. More recently, a 2020 report titled ``Mass Extraction: The
Widespread Power of U.S. Law Enforcement to Search Mobile Phones'' revealed that current hacking tools are stronger than
the defenses provided by mobile device encryption and that law enforcement is actually able to extract data from nearly
every mobile phone on the market \cite{koepke_2020}. The lack of conclusive data showing the extent that encryption
hinders investigations is a weakness of the ``Going Dark'' argument.

However, lack of evidence for an argument does not prove that it is incorrect. Because law enforcement has the data, the
burden of proof is on them---they must show that encryption is causing investigatory failure. Unfortunately, in most
cases, it is impossible to know what would have happened without the interference of strong encryption. Inability to
gather data from controlled tests is a characteristic of \acp{wicked-prob}. When controlled settings are unachievable,
another way to gather data is by observing naturally-occurring conditions in which everything but the variable of
interest remain are constant. These conditions exist for encryption in messaging application ecosystem. Messaging
applications WhatsApp, Facebook Messenger, and Telegram all have large user bases and similar features. Among these
services, only Facebook Messenger does not employ \ac{E2EE}. Globally, Facebook Messenger has approximately 20\% market
share as measured by active monthly users \cite{statista_2020} but accounts for 65\% of \ac{CSAM} reports
\cite{keller_internet_2019}. This indicates that encryption does indeed affect investigations.

% TODO: Possibly find a new spot for this Encrochat story
% European law enforcement's infiltration of encrypted messaging service Encrochat exposed crime rings and led to a
% massive raids \cite{cox_2020}.

Encryption is not the only challenge that law enforcement faces with digital evidence. A 2018 study found that
identifying and cooperating with service providers through existing legal frameworks were even greater challenges
\cite{carter_2018}. The report, titled ``Low-Hanging Fruit,'' also lists resource limitations and poor training as top
issues. Indeed, many current initiatives are underfunded. The \ii{New York Times}' 2019 investigation into \ac{CSAM}
uncovered a chronic lack of commitment from government initiatives \cite{keller_internet_2019}. Major legislation passed
in 2008, but has been funded at only 50\% of authorized levels; a commissioned Justice Department task force has
produced only two of five biennial reports; and a senior executive position within the Justice Department to target
\ac{CSAM} was never created. Federal entities aiming to improve training and coordination between law enforcement and
industry are similarly underfunded \cite{carter_2018}.

% Golden age

The argument for law enforcement having \ii{too much} access to data claims that law enforcement is not ``Going Dark'';
rather, they are experiencing the ``Golden Age for Surveillance'' \cite{swire_encryption_2011}. This argument posits
that although encryption makes certain data completely unaccessible, technological change on the whole has given
authorities much more digital evidence than encryption has taken away. Paired with the dangers of \ac{masssurv},
encryption is a necessary defense. \Ac{EA} would compromise encryption's defense against \ac{masssurv}, and must
therefore be opposed. The argument is mapped in \myfig{fig-arg-golden-age}.

\begin{figure}[p!]
  \centering\CaptionFontSize
  \includegraphics[width=\linewidth]{arguments/build/comprehensive.goldenage.pdf}
  \caption{A ``Golden Age for Surveillance'' Argument Map}
  \label{fig-arg-golden-age}
\end{figure}

The ``Golden Age'' argument focuses on surveillance, its undesirability, and technology's role in it. Cryptographer
Phillip Rogaway writes at length about the negative impacts of \ac{masssurv} in his essay ``The Moral Character of
Cryptographic Work'' \cite{rogaway_moral_2015}. Rogaway asserts that ``pervasive collection \ii{itself} chills
free-speech and threatens liberal democracy.'' He argues that surveillance is fundamentally a tool of power, and for
this reason it should be resisted. Though power from the people, created through representative political process, is
valid, clear abuses have undercut confidence in these institutions.

After establishing the danger of surveillance, the argument notes that technology has increased the government's
surveillance abilities. This statement is supported by strong evidence. In Comey's ``Going Dark'' speech, it was digital
evidence which would not have existed twenty years ago that cracked each case \cite{comey_2014}. Encryption is one of
the only privacy-saving technologies to emerge alongside the boom in privacy-eroding technologies. Opting out of the
technological boom is possible for some, but is not a realistic option for society as a whole. Due to the dangers of
\ac{masssurv}, \ac{encryption} must be embraced.

While stronger than the ``Going Dark'' argument, the ``Golden Age for Surveillance'' argument suffers from a significant
flaw. It focuses on the abilities and dangers of agencies like the \ac{NSA} while ignoring the problems of
resource-strapped local, state, and federal law enforcement. It is obvious that the amount of digital evidence is
increasing. However, as seen in ``Going Dark'' argument, law enforcement often struggles with digital evidence for
reasons unrelated to encryption. Exacerbating the problem for justice system is the ``tech effect,'' which refers to
jurors' increased expectations of digital evidence in cases where they suspect it exists \cite{shelton_study_2006}.

Still, law enforcement's inability to access the tools of \ac{masssurv} cannot be taken for granted. It will likely
change---they already have access to mobile devices. A more conservative criticism of the ``Golden Age'' argument rests
in the fact that \ac{EA} without \ac{masssurv} is conceivable. \ac{EA} implementations that enable law enforcement while
thwarting \ac{masssurv} subvert most of the ``Golden Age'' argument. \ac{EA} proposals with these aims will be explored
in \mysec{sec-ea-types}.


\section{Eliminating Fallacious Arguments}

Using \acp{argmap} to analyze debates allows for consideration of opposing arguments. Part of that consideration
includes identifying and eliminating fallacious arguments. In heated partisan debates, fallacious arguments spread
easily. \myfig{fig-arg-fallacies} shows those most commonly used in the encryption and \ac{EA} debate.

\begin{figure}[t!]
  \centering\CaptionFontSize
  \includegraphics[width=\linewidth]{arguments/build/comprehensive.fallacies.pdf}
  \caption{Fallacious Arguments of the EA Debate}
  \label{fig-arg-fallacies}
\end{figure}

The map begins with fallacious arguments used to attack \ac{EA}. The first is the most common: ``\ac{EA} is a bad idea
because backdoors are insecure.'' It appears on the map as a straw man argument due to the way the term ``backdoor'' is
used. ``Backdoor'' is essentially technical shorthand for ``insecure hack that should never be used in production'';
when it is used as a basket term for all \ac{EA} proposals, the audience is primed to consider them all ill-conceived
and hopelessly dangerous. Thus, the arguer is presenting their opponent's argument for any degree of \ac{EA} in the
weakest possible fashion.

Another common argument is that implementing \ac{EA}---or even researching it---inexorably moves tech policy down a
slippery slope in which government demands will never be satisfied. This argument can be used to discourage those
seeking a middle ground. Going too far and too fast is certainly dangerous, but it is both cynical and misleading to
declare the permanent destruction of privacy as the inevitable end of \ac{EA} research.

It is tempting to declare government requests for \ac{EA} disingenuous due to their hypocritical behavior towards some
of these issues. For example, if combatting \ac{CSAM} is a government priority, why has Congress underfunded the laws
it's passed, and why has the Justice Department understaffed their its task force \cite{keller_internet_2019}? These
questions demand answers, but they do not change the validity (or invalidity) of pro-\ac{EA} arguments. The absence of
government funding does not mean technical anti-\ac{CSAM} measures are unnecessary.

The final two anti-\ac{EA} fallacies are examples of false dichotomies; they present the situation as if there are only
two options. \Acp{wicked-prob} have unlimited sets of potential responses, so reducing them to two is misleading.
Arguers create a false dichotomy when they present either perfect \ac{EA} or no \ac{EA} at all as the only acceptable
solutions. This leaves no room for risk-based approaches, the typical security strategy outside the realm of
cryptography. Opponents of \ac{EA} create another false dichotomy when they suggest that regulators must allow strong
cryptography or we will live in a \ac{masssurv} dystopia. This argument has elements of the ``slippery slope'' and
appeal to emotion fallacies as well. Framing the situation in all-or-nothing terms reduces the chance of a successful
collaborative debate.

Some arguments used to support \ac{EA} rely on appeal to emotion. Politicians and law enforcement agents often argue for
\ac{EA} by appealing to pity for child abuse victims or appealing to fear of terrorists, drug dealers, and kidnappers.
Public safety is a central issue in the debate. The technical community does take these concerns seriously
\cite{schneier_scaring_2019}. These appeals become fallacious when they are used to manipulate the audience into
supporting solutions incommensurate with the problems or to manipulate the audience into believing that the technical
community does not care about these issues.

When experts argue against \ac{EA} due to security challenges, some respond that since we put a man on the moon, we can
surely create secure \ac{EA} \cite{cushing_moon_2018}. This is a weak analogy. The only similarity between the challenge
of landing an astronaut on the moon and the challenge of building a cryptosystem with secure EA is that they are both
difficult. Once the political commitment was made, the Apollo program solved mostly tame problems and had enormous
government backing. Resolving the encryption debate has neither of these advantages. Many similar arguments comparing
\ac{EA} to other feats of technological progress suffer the same faults.

Lastly, government officials often portray \ac{EA} as a problem that security experts simply have not researched enough.
If experts just ``nerd harder,'' as some put it, they will be able to find a solution \cite{schneier_2019}. However,
this notion ignores the conclusions of research done thus far. Encryption policy is a \ac{wicked-prob} for which tame,
technical-only solutions will not work \cite{rozenshtein_wicked_2018}. Even on the technical side, research has
concluded that \ac{EA} is easy from a purely cryptographic point of view. The interfaces between the cryptographic and
human portions of the system pose the real difficulty \cite{abelson_risks_1997} \cite{abelson_2015}. This may not be an
area of strong academic inquiry, but to frame lack of technical research as the primary roadblock to encryption policy
progress misses the point.


\section{EA and Alternatives}

The last four maps framed the debate, presented leading arguments, and identified fallacious arguments. The next map
explores potential solutions. \myfig{fig-arg-measures} divides potential solutions into the categories of current
capabilities, legal measures, and \ac{EA}. Current capabilities can be implemented today, and include maintenance of the
status quo and increasing investment in current programs. Legal measures require a change in the text or application of
the law, but do not require fundamentally new technical capabilities. These include compelling passwords, which requires
legal clarification, and sanctioning lawful hacking, which requires a strategic pivot and an oversight framework. (These
non-\ac{EA} approaches were first introduced in \mysec{sec-alt-approaches}.) In this section, \ac{EA} is compared to its
alternatives. \ac{EA} variations are compared in the next map.

It should be noted that the \ac{argmap} format works best when it displays the debate surrounding just one conclusion or
proposal. This allows adequate room for elaboration on points and counterpoints. When a map includes multiple
conclusions or proposals, it becomes a tangled web of connections, leaving no room for additional detail. For the sake
of brevity, this thesis compares \ac{EA}'s alternatives and variations in single maps. For a thorough discussion of both
\ac{EA}'s alternatives and variations, see the National Academies of Sciences, Engineering, and Medicine's 2018 report
titled \ii{Decrypting the Encryption Debate: A Framework for Decision Makers} \cite{committee_decrypting_2018}.

\begin{sidewaysfigure}
  \centering\CaptionFontSize
  \includegraphics[width=\linewidth]{arguments/build/comprehensive.measures.pdf}
  \caption{EA and its Alternatives}
  \label{fig-arg-measures}
\end{sidewaysfigure}

In \myfig{fig-arg-measures}, response measures are considered according to five metrics. These metrics are based on
several sources, including two separate diverse committees gathered to analyze the encryption debate
\cite{committee_decrypting_2018} \cite{group_2019} \cite{varia_2018}. They also align closely to the four central issues
of security, safety, privacy, and trust. The five metrics used to analyze response measures are (1) minimization of
increased security risk, (2) utility to law enforcement, (3) protection of civil liberties---particularly important as
after the analysis of the ``Golden Age for Surveillance'' argument---(4) transparency, and (5) economic impact. Economic
impact is not as central an issue as the others, but it does help determine practical viability of a proposal.

Potential solutions are divided into the categories of current capabilities, legal measures, and exceptional access. The
first option, in the category of current capabilities, is to do nothing. Some may conclude that the current state of
affairs actually does represent an acceptable balance of interests. If technology and government agree on this approach,
then \ac{the-cw2} can end much like the first. However, legislative action itself is a risk to be mitigated. If the
technical community settles on this approach and government continues to reject it, the government could force its will
and move encryption policy in a dangerous direction.

We could also decide to increase investment in current technologies and investigative techniques. As previously
discussed, encryption is not the most significant barrier in law enforcement's use of digital evidence
\cite{carter_2018}. Even when encryption is involved, traditional sleuthing can often lead to access. Security expert
Bruce Schneier and cyberlaw expert Orin Kerr have compiled a list of ``encryption workarounds'' that do not rely on
\ac{EA}. These include finding the key, guessing the key, accessing plaintext while the device is in use, and finding
another copy of the plaintext \cite{kerr_encryption_2017}. Though probabilistic and time-consuming in comparison to
\ac{EA}, these methods do not require new legal or technical capability. These methods are effective
\cite{greenberg_2018}; investment in current digital evidence capabilities and anti-\ac{CSAM} task forces, they are also
safe.

Compelled password disclosure is a potential solution that simply requires defendants to surrender passwords and
\acp{PIN} to their devices. Questions regarding the legality of this approach have been working their way through the
courts \cite{bittenbender_2019} \cite{sobel_2019} because of the apparent conflict with the Fifth Amendment right
against self-incrimination. Affordability is the strongest argument in favor of compelled password disclosure. Schneier
and Kerr point out several weaknesses with this approach: prosecution must overcome the aforementioned legal hurdles,
the password-holder must be available to investigators, and the password-holder must elect to provide the password
\cite{kerr_encryption_2017}. Failing to provide the password may result in being found in contempt of court, but this
would often be preferable to conviction of the original crime.

\Ac{lawful-hacking} is the most prominent proposed alternative to \ac{EA} \cite{bellovin_lawful_2013}
\cite{hennessey_lawful_2016} \cite{rozenshtein_wicked_2018} \cite{kerr_encryption_2017} \cite{soesanto_2018}.
\Ac{lawful-hacking}'s support may be due in part to the fact that it is a technical approach that isn't \ac{EA} itself.
However, its effectiveness against criminals for whom \ac{EA} would not work is its obvious benefit. This is because,
while organized crime and terror groups would still use encryption tools to evade an \ac{EA} scheme, their operational
security would fail at some point, letting law enforcement or intelligence agencies into their networks and devices.
This approach has already been used \cite{cox_2020}, most famously to bring down cartel kingpin El Chapo
\cite{feuer_chapo_2019}. Because it is already used, using \ac{lawful-hacking} as the primary strategy would require
officially sanctioning, formalizing, and funding \ac{lawful-hacking} programs.

Despite its strengths, \ac{lawful-hacking} is a sub-optimal compromise. Because it requires time and specialized skills,
it is not useful for small departments and commonplace crimes. It would also alter the relationship between law
enforcement and tech companies. Law enforcement agencies would be forced into competition with tech companies if they
were incentivized to conceal the vulnerabilities they discover. Prominent proposals suggest mandating disclosure of
discovered vulnerabilities to mitigate this issue \cite{bellovin_lawful_2013} \cite{hennessey_lawful_2016}.
Theoretically, there are enough vulnerabilities that, even though these discovered vulnerabilities would be disclosed and
patched, new ones could be found quickly enough to enable steady levels of access. Government could source these
vulnerabilities from a combination of the public domain, the commercial exploit market, and a central ``Vulnerability
Lab'' \cite{bellovin_lawful_2013}.

The availability and cost of exploits fluctuates. For example, in January 2019, an Apple iOS jailbreak bug was valued at
\$2 million on the exploit market \cite{goodin_zeroday_2019}. In contrast, a 2020 report revealed that cracking mobile
devices cost law enforcement a mere \$2000 per unit \cite{koepke_2020}. As the exploit arms race continues,
exploitability will be inconsistent. If law enforcement relies on \ac{lawful-hacking} as its strategy, it would face
periods where it struggles for access, and would need to develop costly exploits in the lab. Mandatory vulnerability
disclosure is necessary to avoid competition between law enforcement and tech companies. However, based on the
\ac{FBI}'s historical abuses of power \cite{shamsi_2011} and Congress's reluctance to fund initiatives it has passed
\cite{keller_internet_2019} and failure to provide strong oversight \cite{johnson_congressional_2004}, it seems unlikely
that government would readily forfeit expensive vulnerabilities.

The final proposal is \ac{EA} itself. \ac{EA} would provide a standardized method for law enforcement to reliably access
\ac{plaintext}. It could compliment \ac{lawful-hacking} by eliminating the incentive to conceal vulnerabilities because
hacking would be unnecessary for \ac{EA}-compliant systems. When used together, \ac{EA} would combat common crime and
\ac{lawful-hacking} would combat organized crime and terror groups. \Ac{EA}'s impact on civil liberties and
\ac{masssurv} would depend on its implementation. Any proposal would be costly to deploy and would threaten U.S. product
competitiveness due to perceived insecurity. Increased security risk is \ac{EA}'s largest weakness. Any proposal must
overcome significant challenges to minimize risk. The next map analyzes this more closely.


\section{Zooming in on EA}
\label{sec-ea-types}

\myfig{fig-arg-classes} maps the arguments surrounding several classes of \ac{EA}. Weak cryptography, trusted-party key
escrow, and distributed key escrow apply to both \ac{DAR} and \ac{DIM}. Device key escrow applies only to \ac{DAR}, and
cryptographic puzzles and ghost users apply only to \ac{DIM}.

\begin{sidewaysfigure}[p!]
  \centering\CaptionFontSize
  \includegraphics[width=\linewidth]{arguments/build/comprehensive.classes.pdf}
  \caption{Classes of EA}
  \label{fig-arg-classes}
\end{sidewaysfigure}

The first type of \ac{EA} is simply the use of weak cryptographic ciphers or short encryption keys. This was the
approach taken before \ac{the-cw1}, when strong cryptography was not readily available and was subjected to export
controls. No party seriously supports this approach today, as far too many attackers would have the capability to abuse
it.

Trusted-party key escrow relies on an entity or a small group of entities to store a key or key recovery information.
There were a wide variety of trusted-party key escrow implementations created during \ac{the-cw1}
\cite{denning_taxonomy_1996}. This type of \ac{EA} has a crushing weakness: it centralizes extremely sensitive
information, which heightens the risk of a disastrous data breach. Some argue software companies are capable of
protecting such sensitive data, citing their current practice of storing code-signing keys in \acp{HSM}
\cite{ozzie_2018}. However, such keys are rarely used in comparison to \ac{EA} keys, which would be updated and accessed
frequently. Although code-signing keys are rarely used and are protected in \acp{HSM}, they still leak
\cite{green_2018}. Once the escrowed keys leak, every device or message encrypted with them would be at risk. This is an
example of the \ac{argmap}'s sometimes imbalanced portrayal of an argument's strength---in this case, the map
underrepresents the effectiveness of the security-based argument against centralized access mechanisms.

Distributed key escrow differs from trusted-party in that it takes strong steps to remove centralized trust. This is
achieved by using distributed systems to make key extraction difficult and publicly observable \cite{phan_key_2017}
\cite{servan_schreiber_jje_2020}. One advantage of distribution over centralization is that it is more likely to fail
safely. System failure would more likely result in \ii{no one} to be able to recover plaintext instead of \ii{anyone}.
More research in this area is needed. Unfortunately, these systems would likely be costly to deploy and difficult to
update due to their complexity.

One type of \ac{DIM} \ac{EA} involves uses strong cryptography but includes extra cryptographic ``puzzle pieces'' in
message metadata. A surveiller could reconstruct the encryption key from the puzzle pieces, but only after expending
considerable computational power \cite{bellare_translucent_1996} \cite{wright_crypto_2018}. This approach precludes mass
surveillance, but it also enables arbitrary surveillance by anyone with enough computing power. Using this type of
\ac{DIM} \ac{EA} in common cases would be impractical due to its high cost. However, it would be less effective in
high-profile cases due to the ability of sophisticated adversaries to employ \ac{EA} workarounds. Additionally, puzzle
difficulty would be calibrated based on the changing standards of computational capability. Since opponents can record
data and decrypt it later when capabilities have changed, the data becomes steadily less secure over time.

Another \ac{DIM} \ac{EA} proposal involves adding authorities as a ``ghost'' participant in a conversation, analogous to
a silent listener on a conference call. This was notably suggested by \ac{GCHQ} officials Ian Levy and Crispin Robinson
\cite{levy_robinson_2018}. Their writeup is based on the same foundational principles as this thesis:

\begin{displayquote}
In any discussion of cyber security, details matter.

Unfortunately, it's the details that are missing from the discussion around lawful access to commodity end-to-end
encrypted services and devices (often called the ``going dark'' problem). Without details, the problem is debated as a
purely academic abstraction concerning security, liberty, and the role of government.

There is a better way that doesn’t involve, on one side, various governments, and on the other side lawyers,
philosophers, and vendors' PR departments continuing to shout at each other. If we can get all parties to look at some
actual detail, some practices and proposals---without asking anyone to compromise on things they fundamentally believe
in---we might get somewhere.
\cite{levy_robinson_2018}
\end{displayquote}

Although this proposal was raised with good intentions, it is nevertheless problematic. Proponents of the ghost user
proposal consider the uncompromised use of strong encryption as the proposal's great strength. Although this is a good
property, compromising a messaging app's authorization protocol is as dangerous as compromising its encryption
\cite{callas_1_2019}. Fundamentally, this proposal suffers from the same weakness as trusted-party key escrow---it
introduces an authorization vulnerability in the messaging service provider's platform, resulting in dangerous
centralized access capability \cite{schneier_ghost_2019}. It has a different form, but same nature as rejected key
escrow proposals.

Finally, one type of \ac{DAR} \ac{EA} is device key escrow. In this scheme, key recovery material exists on the device
itself. Most device key escrow proposals include physical possession requirements and make it clear when \ac{EA} has
been performed \cite{savage_lawful_2018} \cite{ozzie_2018}. Such features preclude \ac{masssurv} and ensure transparency
to the device user. This approach relies on secure hardware and a device unlock authorization process. The authorization
process can be paired with other \ac{EA} strategies, such as storing keys with trusted-parties or distributed systems.
This is where the weakness of the device escrow is revealed. First, secure hardware enclaves, though improving, are not
totally reliable. Second, authorization to the hardware device is susceptible to the same attacks as traditional escrow
\ac{EA}.

Device key escrow's greatest weaknesses, however, are still signs of progress. While it does rely on a device's secure
enclave, this is not a new risk. The same hardware device already manages device unlock and encryption functionality.
Additional functionality increases the attack surface, but it does not add a completely new threat vector. While the
authorization process is subject to the same problems as traditional key escrow after the device has been obtained, this
is still an improvement over previous designs. Device key escrow requires the device to be obtained and cannot be used
surreptitiously. Thus, advances in secure hardware design have meaningfully changed the \ac{EA} risk profile.

In 2019, Carnegie Mellon assembled an ideologically diverse group of policy and security experts to engage in the kind
of cross-disciplinary research. After describing every technical branch of the situation, their report identified the
most promising branch: \ac{EA} for domestic law enforcement (i.e., common cases instead of sophisticated adversaries)
focusing on \ac{DAR} in mobile phones using device key escrow requiring physical access \cite{group_2019}. This is
because, as explained above, traditional investigation combined with lawful hacking is best suited for advanced
opponents, while device key escrow represents actual progress towards low-risk \ac{EA} for common crimes.

Device encryption, particularly in mobile phones, may actually pose the smallest impediment to investigation of any
encryption technology. This fact undercuts the argument for device key escrow. The ``Mass Extraction'' report clearly
describes the current state of affairs: device manufacturers are losing the exploit arms race, private companies are
commodifying the \ac{lawful-hacking} approach for agencies large and small, and the lack of regulatory strategy is
leading to pervasive data collection with little oversight \cite{koepke_2020}. This state of affairs reinforces the
lessons of \mychap{chap-policy}---inaction \ii{is} action when facing \acp{wicked-prob}, and in this case, inaction has
led to an undesirable outcome. This situation does not negate the value of research into device-oriented \ac{EA}. This
is because the advantage hackers hold over device manufacturers is temporary; when law enforcement faces difficulty
accessing devices again, this issue will immediately resurface. This is likely why law enforcement continues to push for
\ac{EA} despite present circumstances. Additionally, a formalized process may reduce government abuse, which threatens
privacy and civil liberties more significantly than criminal abuse.

Despite device key escrow's tractability as a technical niche of \ac{EA}, it is not ready for deployment. There are few
proposals, the few that exist lack detail, and the matters of scaling and administration are unresolved. However, as
\ac{incrementalism}'s Lindblom and even \ac{GCHQ}'s Levy and Robinson have pointed out, focusing on specific proposals
may help to identify mutually agreeable solutions---or at least to refine the debate so that the next round of inquiry
can be better informed.

The next chapter focuses on one specific device key escrow proposal. It defines a threat model and analyzes Stefan
Savage's 2018 proposal, ``Lawful Device Access without Mass Surveillance Risk: A Technical Design Discussion''
\cite{savage_lawful_2018}.
