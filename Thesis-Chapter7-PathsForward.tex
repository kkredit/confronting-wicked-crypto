\chapter{Paths Forward}
\label{chap-pathsforward}

The interests of security, safety, privacy, and trust converge on the issue of encryption and \acl{EA}, creating a
\ac{wicked-prob}. The importance of information security is increasing as more data, business, and infrastructure goes
online. The technical community struggles with security. Even for the most well-equipped organizations, thwarting the
average attacker is difficult, and the best attacker, impossible. \Ac{cryptography} plays a fundamental role in
security, yet it remains a controversial technology due to its simultaneous role in enabling unprecedented privacy.

First, proponents of \acl{EA} argue that encryption-enabled privacy as harmful effects on public safety. Despite a
shortage of quantitative analyses showing its impact on crime, encryption certainly does allow evidence of wrongdoing to
be hidden from service providers and law enforcement. This enables harmful activity to go unchecked and decreases public
safety, which is a legitimate concern and the primary motivation for \ac{EA}.

However, while encryption adds privacy, technological change as a whole is stripping it away. The ubiquity of modern
computing has created terabytes of data on every individual; access laws, business models, and insecurity have exposed
this data to huge audiences. This trend has enabled \ac{masssurv}, inappropriate data mining, identity theft, and other
threats. When encryption mitigates these threats, it increases public safety.

Finally, deciding how to respond to these issues---whether with inaction, \ac{EA}, or some alternative---requires
determining what institutions are worthy of trust. Potential solutions assign different levels of trust to law
enforcement agencies, intelligence agencies, technology makers, individual users, and the public at large. Each of these
groups have demonstrated untrustworthiness, but all paths forward require putting trust somewhere.

This complex mix of issues makes encryption policy a \ac{wicked-prob}. Even formulating the problem requires several
iterations of research into history, arguments, and technology. Every formulation presupposes a solution, but the
effectiveness of a solution cannot be accurately evaluated. Meanwhile, the underlying conditions constantly change due
to external forces and policy actions (or lack thereof).

In this thesis, I propose a method to confront \acp{wicked-prob} based on a modified \ac{OODAloop}. The method
emphasizes collaborative debate between diverse individuals, facilitated in part by argument maps and focused on
specific proposals. The second half of the thesis applies this method by surveying the debate, mapping the arguments,
and ultimately performing threat model analysis of a specific \ac{EA} proposal for \acl{DAR}. I now turn to conclusions
and paths forward for both the technical and policymaking communities.


\section{Conclusions}

Based on the arguments analyzed in \mychap{chap-arguments}, neither side of the \ac{EA} debate addresses all of the root
issues. However, given the wide gap between the requests of government officials and the maturity of \ac{EA} technology,
the anti-\ac{EA} argument is currently stronger. Encryption does pose certain threats to public safety. Terrorists and
criminals can and do use encryption to evade the law; in particular, this has allowed for shocking levels of \acf{CSAM}
to be peddled \cite{keller_internet_2019}. However, the argument that \ac{encryption} is causing \acl{LE} failure is
unsubstantiated at this time. For \acl{DIM}, poor coordination and training hamper law enforcement more than encryption
\cite{carter_2018}. For \acl{DAR}, commercial \ac{lawful-hacking} tools currently outdo mobile phone manufacturers,
meaning \acl{LE} can get into almost any mobile device they acquire \cite{koepke_2020}. Additionally, strong
\ac{encryption} is a necessary tool in protecting security and privacy---two public goods that other technological
developments often fail to provide.

Though the pro-\ac{EA} argument does not justify an \ac{EA} mandate, it can teach technologists some important lessons.
First, holistic cryptosystem threat models must include two non-traditional threat actors, the malicious user and
threatening lawmaker. Encryption lends power to its user, and this power can be abused. Technology makers must include
abusive use in their threat modeling. Governments are inherently sensitive to changes in personal privacy; lawmakers
therefore represent a unique threat because they can outlaw uses of encryption entirely if they deem them too dangerous.
The danger of an \ac{EA} mandate must also be included in the cryptosystem threat model.

Second, \ac{EA} history teaches that the current situation is temporary. This is true of all \acp{wicked-prob}, but
especially for encryption policy due to rapidly evolving technology and a vulnerability arms race between device
manufacturers and lawful hackers. Circumstances are sure to change, and the conclusion that \ac{EA} is unjustified may
change as well.

% Research is just part of the effort that society should put towards addressing the root issues.

Further research into \ac{EA} is warranted due to the additions to the threat model and the instability of present
circumstances. Fortunately, developments in secure hardware mark real progress in the search for acceptably secure
\ac{EA}. Secure hardware technology is not perfect; on its own, it does not resolve the numerous challenges that
full-fledged \ac{EA} proposals must overcome. However, \ac{EA} proposals leveraging secure hardware, such as device
escrow for \acl{DAR}, represent progress. They offer transparency, protection from \ac{masssurv}, and \acl{LE} utility
while remaining somewhat secure.

The conclusions presented here do not settle the issue. However, when dealing with \acp{wicked-prob}, meaningful
discussion can be considered progress even if it does not reach concrete resolution. Much of the work involved in
confronting \acp{wicked-prob} is understanding the problem itself. The conclusions above do not bring closure, but
clarity.


\section{Paths Forward: Technology Makers}

It is the responsibility of the technical community to accept and address the substantial role technology plays in
current policy issues, such as encryption, automation, and misinformation. Government sometimes suggests that the tech
world simply needs to try harder. Because these are not tame problems, the tech world is absolutely right that
simply trying harder will not work---but that does not mean they cannot deal with them. The tech industry is optimized
to address tame problems, but it must now turn its financial and intellectual capital towards strategies designed for
\acp{wicked-prob}. In order to do so, the tech industry must involve people of diverse backgrounds and values in design
and investment decisions. Designers must consider the ways in which their products could be abused, and business models
must be responsive to evolving conditions.

Addressing \acp{wicked-prob} reveals smaller subproblems, some tame and some also wicked (see
\myfig{fig-policy-ooda-process}). In addition to making changes to tackle \acp{wicked-prob} directly, the technical
community should also address the tame subproblems. Some suggestions for doing so are listed below.

\newcommand{\taskstart}[0]{\begin{itemize}}
\newcommand{\taskitem}[2]{ % Name, description
    \item \textbf{#1} \nopagebreak

    \vspace{0.5\baselineskip} \parbox{\linewidth}{#2} \vspace{0.5\baselineskip}
}
\newcommand{\taskend}{\end{itemize}}

\taskstart
    \taskitem{Basic security research}{

\ac{EA} is only worthwhile when basic security renders lawful hacking insufficient, which is not the case for current
device encryption. This is a security ergonomics problem more than a technical problem---cryptographers are already
capable of creating stronger device encryption, but users would rebel due to its adverse effects on usability. Solving
the strong-yet-usable device encryption problem will make the most promising class of \ac{EA}, device key escrow,
relevant. Improving security in general will make any \ac{EA} proposal more tenable.

}

    \taskitem{Research technical transparency mechanisms}{

One of the largest non-technical risks of \ac{EA} is its potential abuse by jurisdictions with poor rule of law. This is
the reason transparency is a goal of the threat model. Device key escrow can achieve partial transparency through
physical possession and disassembly requirements. Ideally, usage of \ac{EA} mechanisms would be fully auditable to the
public through inherent technical mechanisms. This can be improved for \acl{DAR} and must be improved for \acl{DIM}.

}

    \taskitem{Research distributed trust mechanisms}{

Most \ac{EA} proposals would still result in a concentrated center of risk, whether it would take the form of a database
of keys or of a control center with the ability to add ``ghost users'' to conversations. As explained in
\mysec{sec-premises}, insecurity cannot be destroyed, it can only be moved around. Distributed trust mechanisms avoid
assigning a large amount of trust to any single party. This contributes to both transparency and security.

}
\taskend

Fortunately, this research is already happening to some degree, though more is necessary. Some \ac{EA} proposals include
built-in transparency and distributed trust mechanisms \cite{goos_oblivious_1996} \cite{phan_key_2017}
\cite{servan_schreiber_jje_2020}. Like device key escrow, approaches such as these hold promise, but must be better
understood before they could be implemented. Even while these problems remain unsolved, technologists must provide clear
security advice and speak out against bad policy.


\section{Paths Forward: Policymakers}

% TODO: Prof. Walsh:
% - Liked the lessons for techies; this distills the same problems for policymakers
% - Enjoyed perspective of the "hawkish lawmaker" and how that connects to LAED Act and fallacious argmap
% - Would love to see this in the conclusion -- tie it all together and make sure it hits home for policymakers

Policymakers have not been silent on the issue of encryption and lawful access. The influence of the executive,
legislative, and judicial branches ensures that the regulatory environment will evolve with changing federal policies,
laws, and rulings. The judiciary will shape encryption policy through its rulings and by reevaluating Fourth Amendment
jurisprudence in light of rapidly changing circumstances. Still, the primary actors capable of consciously taking part
in strategic policymaking will be the executive and legislative branches.

Executive and legislative policymakers must strategically approach encryption policy as a \ac{wicked-prob}. Industry has
failed to resolve the problem with analysis, and government has failed to resolve the problem with \ac{incrementalism}.
\Ac{incrementalism} has safely avoided dangerous outcomes such as blanket \ac{EA} mandates. However, it has also
resulted in a research-stifling debate, the poor utilization of current capabilities, and the unregulated use of
invasive \ac{lawful-hacking} tools. While policymakers and law enforcement agencies have made important contributions to
the debate, they need to share data more openly. Just like the technical aspects, the policy aspects of the solutions
must be responsive to evolving conditions.

Based on the analysis performed thus far, the following steps could be taken today.

\taskstart
    \taskitem{Take the low-hanging fruit}{

Law enforcement is ill-equipped to access digital evidence with existing mechanisms \cite{carter_2018}. Justice
Department task forces for combatting \acl{CSAM} are underfunded and understaffed \cite{keller_internet_2019}.
Investment in existing programs will likely produce results without introducing risk.

}

    \taskitem{Regulate current lawful hacking}{

\Ac{lawful-hacking} using commercial mobile device forensic tools has become a common practice; this invokes serious
privacy and civil rights concerns \cite{koepke_2020}. Regardless of long-term encryption policy strategy,
\ac{lawful-hacking} must be regulated. Michigan's 2020 constitutional amendment represents one positive development: it
explicitly protects electronic data and communications from search without a warrant \cite{ballotpedia_2020}.
Protections like this can prevent abuse of current \ac{lawful-hacking} capability.

}

    \taskitem{Fund research}{

Lawmakers request research into \ac{EA} technology, but private companies lack financial incentive and universities lack
interest due to the threatening climate. The federal government cannot compare building secure \ac{EA} to going to the
moon if it does not provide the same level of funding. In order to be welcomed by the academic community, such funding
must be accompanied by a binding statute excluding the possibility of a premature \ac{EA} mandate.

}
\taskend


\section{Decision and Action}

The goal of this thesis is to establish encryption policy as a \ac{wicked-prob}, to analyze the debate, and to evaluate
the security of current \ac{EA} technology. I aim to advance the discussion with a strategy to tackle \acp{wicked-prob},
additions to the cryptosystem threat model, and detailed analysis of a modern proposal. The strategy is based on the
\acl{OODAloop}. In those terms, this paper serves to \ii{observe} and \ii{orient}. Technologists, policymakers, and
common people must \ii{decide} and \ii{act}. Confronting these issues may be difficult, but the future of security,
safety, and privacy depends on finding an acceptable solution where no perfect solution exists.
