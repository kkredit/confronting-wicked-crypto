\chapter{Background}

Private use of encryption has a contentious history. [Some highlights.]

Electronic encryption must be understood as a technological development embedded in the longer history of privacy and
human rights. [This is where I'd like to understand more of the legal side. With a historical perspective.]

Privacy is not encryption's only purpose, however. Encryption is the technical foundation for many forms of computer and
network security. [Properties: confidentiality, authentication, integrity, non-repudiation; primitives: hashed
passwords, digital signatures, ...; technologies: PKI, TLS, SSO/Kerberos, VPNs, disk encryption, blockchains (might get
eye-rolls).]

Electronic encryption's dual role as enabler of perfect privacy and cornerstone of computer security may be the central
conflict in the exceptional access debate. Those who focus on its role in privacy object to its absolute power. Those
who focus on its role in security rely on its absolute power.

% Encryption requires absolute cryptographic fidelity in order to provide security.

\section{Exceptional Access: Arguments For and Against}

\subsection{Arguments For}

Primary reasons (why EA itself is beneficial):
\begin{itemize}
    \item Absolute privacy is not a right
    \item Encryption gives too much power to the individual
    \item EA would allow more crimes to be prevented and prosecuted
\end{itemize}

Secondary reasons (why studying EA is beneficial):
\begin{itemize}
    \item EA done well would be more secure and privacy preserving than EA done poorly
    \item The threat of legislative action creates urgency
\end{itemize}

\subsection{Arguments Against}

Primary reasons (why EA itself is bad):
\begin{itemize}
    \item Privacy is an absolute right
    \item Exceptional access gives too much power to the state
    \item Exceptional access is not possible in an acceptably secure manner
\end{itemize}

Secondary reasons (why studying EA is non-beneficial):
\begin{itemize}
    \item With both security and privacy in a poor state, research should be focused on strengthening, not weakening,
            them
    \item The technical community should not entertain delusions about what is truly possible
\end{itemize}

\section{Technical State of the Art}

\subsection{Historical Approaches}

Historical approaches and proposals out of the first crypto war that saw various amounts of support.

\begin{itemize}
    \item Clipper chip
    \item Government key escrow
    \item Oblivious key escrow \cite{goos_oblivious_1996}
    \item Partial key escrow (see \cite{denning_taxonomy_1996})
\end{itemize}

\subsection{Modern Proposals}

Recent proposals out of the second crypto war that have various levels of support.

\begin{itemize}
    \item CLEAR \cite{ozzie_2018}
    \item Device escrow with HW \cite{savage_lawful_2018}
    \item Computational, blockchain approaches \cite{phan_key_2017}
    \item Lawful hacking \cite{nguyen_lawful_2017} \cite{soesanto_2018}
    \item Forced password disclosure \cite{bittenbender_2019}, see \cite{kerr_encryption_2017}
\end{itemize}

\section{Paths Ahead}

\subsection{Legislative Priorities}

\begin{itemize}
    \item Reject EA until technical capabilities are ready \cite{varia_2018}
    \item Be forthcoming with data so that we know what to work on \cite{armerding_2017}
    \item Don't prescribe technical methods, but be clear about principles \cite{matyas_incommensurability_2018}
    \item Fund the research \cite{varia_2018}
\end{itemize}

\subsection{Technical Research}

\begin{itemize}
    \item Acknowledge the moral character of cryptography \cite{rogaway_moral_2015}
    \item Do the research
    \item Focus on principles \cite{levy_robinson_2018}
    \item Be clear on definitions \cite{varia_2018}, but don't expect govt to provide those (unlike
            \cite{abelson_2015}), and clearly define their interfaces with principles
            \cite{matyas_incommensurability_2018}
\end{itemize}
