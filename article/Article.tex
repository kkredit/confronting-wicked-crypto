%%%%%%%%%%%%%%%%%%%%%%%%%%%%%%%%%%%%%%%%%%%%%%%%%%%%%%%%%%%%%%%%%%%%%%% SETUP %
\documentclass[conference]{IEEEtran}
\usepackage{cite}
\usepackage{amsmath,amssymb,amsfonts}
\usepackage{algorithmic}
\usepackage{graphicx}
\usepackage{textcomp}
\usepackage{xcolor}

\def\BibTeX{{\rm B\kern-.05em{\sc i\kern-.025em b}\kern-.08em
    T\kern-.1667em\lower.7ex\hbox{E}\kern-.125emX}}

\def\tt#1{\mbox{\texttt{#1}}}

\newcommand*{\thead}[1]{\multicolumn{1}{c}{\bfseries #1}}

\begin{document}


%%%%%%%%%%%%%%%%%%%%%%%%%%%%%%%%%%%%%%%%%%%%%%%%%%%%%%%%%%%%%%%%%%%%%%% TITLE %

\title{Wicked Exceptional Access}

\author{\IEEEauthorblockN{Kevin Kredit}
\IEEEauthorblockA{\textit{GVSU ACS Department} \\
Grand Rapids, Michigan \\
Email: k.kredit.us@ieee.org}
}

\maketitle


%%%%%%%%%%%%%%%%%%%%%%%%%%%%%%%%%%%%%%%%%%%%%%%%%%%%%%%%%%%%%%%%%%%%%%% ABSTRACT %

\begin{abstract}

The debate has resumed over whether encryption systems should support alternative means of decryption intended for law
enforcement use, called exceptional access (EA). Recent events have renewed interest in EA from the U.S. legislature.
Although EA would inevitably reduce digital security and privacy compared to free use of encryption, the quality of the
technical and regulatory approach can make a substantial difference. If one considers legislative action as part of the
threat model, one has no choice but to engage the topic. In this thesis I aim to clarify and analyze the debate. I begin
by outlining the issue's technical and historical context. In the following chapters I use argument maps to break down
the arguments used by both sides and use data flow diagrams to create threat models for the separate problems of EA for
data in motion and data at rest.

\end{abstract}

\begin{IEEEkeywords}
encryption, exceptional access, policy, wicked problems
\end{IEEEkeywords}


%%%%%%%%%%%%%%%%%%%%%%%%%%%%%%%%%%%%%%%%%%%%%%%%%%%%%%%%%%%%%%%%%%%%%%% PAPER %

\section{Introduction}

Qwer \cite{flawfinder}.

\section{Conclusion}

Asdf.

%%%%%%%%%%%%%%%%%%%%%%%%%%%%%%%%%%%%%%%%%%%%%%%%%%%%%%%%%%%%%%%%%%%%%%% BIBLIOGRAPHY %

\begin{thebibliography}{00}

\bibitem{flawfinder} D. Wheeler, "Flawfinder Home Page", \textit{Dwheeler.com}, 2019. [Online].
Available: https://dwheeler.com/Flawfinder/. [Accessed: 02- Nov- 2019].

\end{thebibliography}


\end{document}
