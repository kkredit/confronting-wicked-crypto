\chapter{EA Debate Argument Maps}
\label{chap-arguments}

In this chapter we will analyze the \ac{encryption} debate using \acp{argmap}. The maps help one digest the debate, but
there are two important shortcomings of the ones presented here. First, argument mapping is ideally just one tool in the
collaborative process that progressively approaches the problem definition and solution. One can make an approximation
from research, but it is not a replacement for debate arising from live discussion. Second, \acp{argmap} as portrayed in
this thesis can be deceiving about the strength of an argument. Argument nodes show no indication of their strength or
validity; one can formulate ten poor arguments against a single strong one, and a glance at the map will indicate that
the case of the ten is stronger. With those qualifications in mind, let us begin by looking at the factors at the heart
of the conflict.

\section{Contributing Factors}

As previously stated, it is \ac{encryption}'s dual role as enabler of radical privacy and cornerstone of security is at
the heart of the EA debate. Of course, there is only a conflict if one sees radical privacy as a bad thing. The right to
privacy is both a strongly held value and an enshrined legal principle, and many use privacy-based arguments to show
\ac{EA} as socially undesirable. The \ac{snowden} revelations \cite{landau_making_2013} only unveiled the scale of
privacy-eroding U.S. government surveillance enabled by technological changes, terrorism-motivated policies, and weak
oversight \cite{shamsi_2011}. Violations of privacy are concerning for those most vulnerable and for society at large.
For example, Saudi journalist Jamal Khashoggi's private communications were surveilled by the Saudi government before he
became a victim of politically motivated murder \cite{liebermann_2019}. At a societal level, presence of surveillance
alone changes behavior and chills free speech \cite{rogaway_moral_2015}, while government violations of the law degrade
institutional trust.

The government, and particularly \acl{LE}, argues that encryption is handcuffing their investigational capacity. The
\ac{FBI} has branded strong encryption as ``warrant-proof'' and states that ``the government often cannot obtain the
electronic evidence necessary to investigate and prosecute threats to public and national safety'' \cite{fbi_2020}. The
pro-\ac{EA} argument is typically rooted in public safety: investigators need access to evidence to fulfill their
commission to catch and convict lawbreakers, and encryption is making that too difficult. Five-eyes nations regularly
put out joint statements expressing their frustration with encryption \cite{ministerial_2018} \cite{goodale_2017}, and
were joined in 2020 by India and Japan in a statement emphasizing the growing danger of \ac{CSAM} \cite{intl_2020}.
Child pornography is one the original \ac{horsemen} that some argue are used to scare people into supporting backdoors
\cite{schneier_scaring_2019}, but it is also a problem growing to astonishing proportions. The \textit{New York Times}'s
2019 investigation on the matter quoted one law enforcement officer as estimating that 400,000 New Jerseyans, more than
4\% of the state's population, could be charged with violating child exploitation material laws
\cite{keller_internet_2019}.

Unfortunately, shocking assertions such as this have not been independently confirmed or backed up with hard data.
Intelligence agencies have a long history or misrepresenting positions and overstepping bounds
\cite{johnson_congressional_2004} \cite{shamsi_2011}, and this has led to lack of institutional trust undercutting their
claims that they need \ac{EA}. Indeed, the \ac{FBI} has already been caught exaggerating by several times how many
mobile devices they could not access due to \acl{DE} \cite{devlin_2018}. As Rozenshtein says on this subject, ``It is
impossible to know the precise extent to which encryption frustrates law-enforcement investigations, both because
law-enforcement agencies are only beginning to collect accurate statistics, and because one can never be sure of how an
investigation would have proceeded in the absence of encryption'' \cite{rozenshtein_wicked_2018}, but it is still
crucial to have an accurate depiction of the problem in order to come to justified and helpful solutions. This argument
is analyzed further in the next section.

Information security is a the remaining central issue in the debate. As described in \mysec{sec-crypto-basics},
\ac{cryptography} in various forms plays a foundational technical role in nearly every aspect of security, and
information confidentiality itself is one of the primary properties security practitioners fight to protect. Efforts in
the past have sought to undermine the cryptographic foundations of encryption, though this is not a feature of proposals
today, as that is universally seen as too risky. \mysec{sec-tech-approaches} introduced several alternative technical
approaches to \ac{EA}, but experts still proclaim loud and clear that the current state of the art is far from capable
of providing the type of access \acl{LE} in an acceptably secure manner \cite{abelson_2015} \cite{abelson_risks_1997}.
Security was an afterthought in early computing and networking designs, and the field of cybersecurity is still more of
an art the a science; hobbling systems with a mandate to provide \ac{EA} would inevitably diminish security to some
degree \cite{abelson_2015}.

Finally, one may argue that \ac{EA}, depending on its implementation, could have positive auxiliary use cases. It could
enable malware scanning, administrator access in a business setting, or password recovery. Typically these use cases
have other solutions, however, and to combine the requirements of these uses with those of any reasonable \ac{EA}
proposal would likely fatally weaken it.

See \myfig{fig-args-contrib} for the \ac{argmap} representing this section.

\begin{sidewaysfigure}
  \centering\CaptionFontSize
  \includegraphics[width=\linewidth]{arguments/build/comprehensive.contrib.pdf}
  \caption[Contributing Factors to the EA Debate]{Contributing Factors to the EA Debate}
  \label{fig-args-contrib}
\end{sidewaysfigure}

\section{Going Dark vs. The Golden Age}

A significant portion of the disconnect between government and technical community representatives is disagreement over
whether law enforcement has \ii{too much} or \ii{too little} access to data.

See the ``Going Dark'' argument in \myfig{fig-arg-going-dark}.

\begin{figure}[h]
  \centering\CaptionFontSize
  \includegraphics[width=\linewidth]{arguments/build/comprehensive.goingdark.pdf}
  \caption{A ``Going Dark'' Argument Map}
  \label{fig-arg-going-dark}
\end{figure}

See the ``Golden Age of Surveillance'' argument in \myfig{fig-arg-golden-age}.

\begin{figure}[h]
  \centering\CaptionFontSize
  \includegraphics[width=\linewidth]{arguments/build/comprehensive.goldenage.pdf}
  \caption{A ``Golden Age of Surveillance'' Argument Map}
  \label{fig-arg-golden-age}
\end{figure}

% Actually fund initiatives to tackle these problems!
% pfefferkorn_2019
% keller_internet_2019

\section{Eliminating Fallacious Arguments}

\section{EA and Alternatives}

\section{Zooming in on EA}

% - Avoid pitfalls of
%   - Easy answers
%     - EA mandates, complexity denial
%   - Fallacies!
%   - Bad results with long technical tails

% Note: a problem with muddling through wicked problems is that each attempted fix leads to long lasting unintended side
% effects. This is especially true with matters of encryption, where the data could be stored for years till it can be
% decrypted! So must be very careful to avoid solution attempts that leave long scars.

% See committee_decrypting_2018!
% For a wholistic threat model, include both the threats of government (legislation) and of criminal abusers of your
% system.

\section{Specific Proposals}

% See committee_decrypting_2018!

% Subject one: do we even want EA?
%  --> no? exit here
%  --> yes? carry on
%  --> pragmatist? carry on
% include fallacies argument map here
% see rozenshtein_wicked_2018, benaloh_what_2018, soesanto_2018, ministerial_2018, fbi_2020, brantley_2018, lund_2020
% ... group_2019, ruiz_there_2018, rozenshtein_2019

% Subject two: what properties would good EA have?
% easy for LEAs to use? hard? expensive? voluntary? unable to be used for mass surveillance? transparent (attempts are
%   visible)? alignment of values with security (e.g. discourages bug hoarding)? adaptable (since solution won't be
%   permanent, either technically or according to collective will as things change rozenshtein_wicked_2018)? suited to
%   rules and principles (matyas_incommensurability_2018)?

% Subject three: what options do we have?
% map of properties onto proposals--leads into threat modeling, which will check the veracity of claims














%%%%%%%%%%%%%%%%%%%%%%%%%%%%%%%%%%%%%%%%%%%%%%%%%%%%%%%%%%%%%%%%%%%%%%%%%%%%%%%%%%%%%%%%%%%%%%%%

% In this chapter, develop the arguments employed in EA debates. Make use of argument maps to break down and visualize the
% debate, as shown in \myfig{fig-args-top-level}.

% \begin{figure}[h]
%   \centering\CaptionFontSize
%   \includegraphics[width=\linewidth]{arguments/build/top-level.pdf}
%   \caption[Argument: Top Level]{Top level arguments for and against exceptional access.}
%   \label{fig-args-top-level}
% \end{figure}

% The debate over exceptional access takes place on two fundamental levels:
% \begin{enumerate}
%   \item Whether cryptosystems with exceptional access represent a net benefit to society
%   \item Whether and how to conduct exceptional access research
% \end{enumerate}

% \subsection{Arguments For}

% Why EA is a benefit to society:
% \begin{itemize}
%   \item Absolute privacy is not a right
%   \item Encryption gives too much power to the individual
%   \item EA would allow more crimes to be prevented and prosecuted
% \end{itemize}

% Why to research EA:
% \begin{itemize}
%   \item EA done well would be more secure and privacy preserving than EA done poorly
%   \item The threat of legislative action creates urgency
% \end{itemize}

% \subsection{Arguments Against}

% Why EA is a danger to society:
% \begin{itemize}
%   \item Privacy is an absolute right
%   \item Exceptional access gives too much power to the state
%   \item Exceptional access is not possible in an acceptably secure manner
% \end{itemize}

% Why not to research EA:
% \begin{itemize}
%   \item With both security and privacy in a poor state, research should be focused on strengthening, not weakening,
%             them
%   \item The technical community should not entertain delusions about what is truly possible
% \end{itemize}

% \subsection{Fallacious Arguments}

% Fallacy definitions come from the helpful University of North Carolina Writing Center \cite{unc_2020}. Also cited:
% \cite{hanna_2019}.

% Fallacies are as shown in \myfig{fig-args-fallacies}.

% \begin{figure}[h]
%   \centering\CaptionFontSize
%   \myincludeargument{comprehensive.fallacies}
%   \caption[Fallacious Arguments]{Fallacious arguments used in the exceptional access debate.}
%   \label{fig-args-fallacies}
% \end{figure}
