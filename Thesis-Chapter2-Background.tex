\chapter{Background}
\label{chap-background}

The problem of strong encryption and exceptional access exists in a technical, historical, and regulatory context. This
chapter introduces cryptography, summarizes the history of encryption and the crypto wars, lists relevant encryption
regulations, and lists prominent technical \ac{EA} proposals. It then introduces argument maps as a logical modeling
tool, and introduces threat modeling with data flow diagrams.



\section{Cryptography Basics}
\label{sec-crypto-basics}

\Ac{cryptography} is the study of techniques for communicating secretly in the presence of third parties. This is
performed by using some \ac{cipher} to translate between \ac{plaintext} and \ac{ciphertext}. The cipher is the tool or
algorithm that performs the translation, plaintext is the original data, and ciphertext is the encoded data. The
processes for translating from plaintext to ciphertext and back are \ac{encryption} and \ac{decryption}. A well-designed
cipher ensures that only those with the correct secret information, the \ac{key}, can perform encryption or decryption
on the text.

% There are two classes of cryptographic ciphers--stream and block

There are two major cryptographic protocol families--symmetric and asymmetric. In traditional \ac{sym-crypto},
encryption and decryption are performed with the same key, and both parties must have this key in order to communicate
securely. In \ac{asym-crypto}, encryption and decryption are performed with separate, paired keys, called the public key
and the private key. The innovation of \ac{public-key-crypto} is that the public key is not secret, and using this
technique, two parties can communicate securely without requiring an a previously agreed upon shared secret.

One can also define two major cryptographic applications--securing \acf{DIM} and \acf{DAR}. Each application is a
different problem that results in different solutions. \Ac{DIM} typically uses long-lived \ac{asym-crypto} keys to
perform authentication and establish short-lived \ac{sym-crypto} sessions keys; the sessions keys perform the actual
encryption of the data in transit. Important properties of \ac{DIM} encryption protocols are \ac{forward-secrecy}, which
ensures that a leaked private key or session key does not compromise any other private key or session key, and
\ac{replay-protection}, which ensures that messages cannot be replayed by an attacker without detection
\cite{bellovin_thinking_2016}. \Ac{E2EE} for instant messaging services is an example of encryption for \ac{DIM}.

\Ac{DAR} by nature must use long-lived keys for encryption. Instead of being negotiated and randomly generated at
encryption-time, as \ac{DIM} session keys are, these keys are either derived from a password stored in human memory or
are stored somewhere in computer memory, which often takes place with the assistance of dedicated hardware.
\Ac{disk-encryption} for laptops and mobile devices is an example of encryption for \ac{DAR}.

Data secrecy is not cryptography's only purpose, however.

Cryptography is the technical foundation for many forms of computer and network security, at every scale. Security is
often defined in terms of several properties: authentication, integrity, non-repudiation, confidentiality, availability,
and authorization \cite{shostack_threat_2014}. Security properties are systems properties, but systems properties emerge
from the composition of their parts, and innumerable primitives and technologies rely on cryptography.

% Spoofing                  Authentication
% Tampering                 Integrity
% Repudiation               Non- Repudiation
% Information Disclosure    Confidentiality
% Denial of Service         Availability
% Elevation of Privilege    Authorization

Cryptographic primitives include symmetric key encryption, symmetric key encryption, one-way hash functions, and digital
signatures. A few technologies that directly rely on these primitives include \ac{TLS}, \ac{PKI}, \ac{SSO} (e.g.
Kerberos), \acp{VPN}, secure password storage, disk encryption tools, and secure- and trusted-boot processes. A few
systems that directly rely on these technologies include the financial system, local and global; industrial control
systems and critical infrastructure; health information systems; private, government, and military communications; and
the integrity of the internet as a whole.

% Let us break security down into three scales: primitives, technologies, and systems. A primitive is a conceptual
% building block that serves a single technical function. A technology is a set of implemented primitives glued together
% to provide useful features. A system is a collection of technologies that collectively serve the ultimate outcome.
% The security properties critical at the systems level depend

% Primitives: one-way hashes enable integrity checks, block ciphers enable efficient confidential data storage, and
% digital signature enable non-repudiation

% Technologies: Transport Layer Security (TLS) enables confidentiality and integrity over the internet, Kerberos enables
% convenient authentication, and virtual private networks (VPNs) enable confidential and authenticated remote network
% connections

If the cryptographic primitives that compose these systems were to be compromised, the systems they comprise could
collapse. If the financial system loses integrity, health information systems lose confidentiality, critical
infrastructure loses availability, government communications lose authenticity and non-repudiation, or the internet as a
whole loses integrity, inconceivably large damage would occur.

% FIXME: Kristen doesn't like the wording; also remove word "electronic"?
Electronic encryption's dual role as enabler of strong privacy and cornerstone of computer security may be the central
conflict in the EA debate. Those who focus on its role in privacy object to its absolute power. Those who focus on its
role in security rely on its absolute power.



\section{Encryption History}
\label{sec-crypto-history}

The section provides an overview of encryption from ancient to contemporary times.

\subsection{Encryption In the Past}
\label{sec-history-old}

Computerized encryption is a new technology, but the field of cryptography is old, as demand for privacy is as old as
communication itself. The most well known ancient example of rudimentary encryption is the Caesar cipher, named after
the character transposition technique Julius Caesar used to protect private correspondence
\cite{luciano_cryptology_1987}. Significant developments include the first formal cryptographic study by Arab scholars
in the eighth century, advancements made out of necessity in the twentieth century's world wars, and the application of
computers to cryptographic problems \cite{kahn_codebreakers_1996}. Claude Shannon's formalized the modern ``mathematical
analysis of cryptography'' in 1949 \cite{shannon_communication_1949}, and Whitfield Diffie and Martin Hellman published
research on \ac{public-key-crypto} in 1976 \cite{diffie_new_1976}.

The discovery of \ac{public-key-crypto} was an important advance. Combined with computer networking and personal
computing advances in the 1980s, it for the first time put the power of strong encryption in the hands of ordinary
people. However, the privacy provided by cryptography is a powerful tool, and privacy is a property governments strive
to regulate. The use of encryption for for private purposes has a contentious history. In 1587, Mary Queen of Scots was
convicted of treason based on evidence from decrypted letters, and in 1807, prosecutors trying Aaron Burr for treason
tried to force testimony from his secretary on the contents of encrypted messages \cite{kerr_encryption_2017}.
Government's natural stance towards privacy makes it inherently uncomfortable with the rapid increase in availability of
strong encryption.

% Electronic encryption must be understood as a technological development embedded in the longer history of privacy and
% human rights.

\subsection{The First Crypto War}
\label{sec-history-cw1}

All these realities--government discomfort with absolute privacy, rapidly increasing availability of strong encryption,
and a blossoming computer industry foundationally reliant on cryptography--came to a head in what has become known as
\ac{the-cw1}.

In 1976, the same year Diffie and Hellmann published their research in public key cryptography, the U.S. Congress passed
the \ac{AECA} and declared that strong cryptography is subject to export controls \cite{kehl_right_2015}. In 1991, the
U.S. Senate introduced, but did not pass, a bill mandating access to plain text contents when authorized by law; in
response, Philip Zimmermann released \ac{PGP}, a public key encryption tool to secure email, in order that strong
cryptography be ``made available to the American public before it became illegal to use'' \cite{zimmermann_1996}. In
1993, the Clinton administration introduced the \ac{clipperchip} \cite{press_1993} with the goal of ``providing the
public with strong cryptographic tools without sacrificing the ability of law enforcement and intelligence agencies to
access unencrypted versions of those communications'' \cite{thompson_2015}. Citing the foundational role of cryptography
in security--and the potential of human rights abuses for compromised privacy--industry and prominent technical leaders
reacted negatively to the initiative \cite{kehl_right_2015} \cite{zimmermann_1996}. When security researcher Matt Blaze
discovered flaws that allowed users to subvert the \ac{clipperchip} mechanisms \cite{blaze_protocol_1994}, the proposal
died.

Despite the failure of the \ac{clipperchip}, the debate over access to strong encryption continued, primarily focused on
various key escrow proposals \cite{thompson_2015}. In 1996, two bills were introduced in the US Congress. In the Senate,
it was S.1726, the Pro-C0DE Act of 1996, to abolish the export controls on encryption software \cite{burns_s1726_1996}.
In the House, it was HR.3011, the SAFE Act of 1996, to explicitly allow arbitrarily strong encryption for all legal
activity \cite{goodlatte_hr3011_1996}. Indeed, the SAFE Act went further, including ``provisions that would have barred
the government from creating a mandatory key escrow system and would also have removed export restrictions on most
generally available software and hardware containing encryption'' \cite{kehl_right_2015}. Zimmermann, the author of PGP,
who had by this time endured an investigation by the Customs Service for publishing his work, testified before the
Senate in favor of the Pro-C0DE Act \cite{zimmermann_1996}. In November of that year, the Clinton administration
released an executive order effectively removing export controls on strong encryption products, along the lines of the
Pro-C0DE Act \cite{clinton_1996}.

From 1996 to 1999, the SAFE Act was proposed several times, had hearings, and gained support \cite{kehl_right_2015}. By
this time, there was ``an overwhelming amount of evidence against moving ahead with any key escrow schemes''
\cite{thompson_2015}, and in 1999 the Clinton administration abruptly changed course, adopting almost all SAFE Act
proposals \cite{kehl_right_2015}. This development marks the end of \ac{the-cw1}.

\subsection{The Second Crypto War}
\label{sec-history-cw2}

Two developments paved the way for \ac{the-cw2}. The first was actually a non-development: when the Clinton
administration changed its encryption policy, the U.S. House dropped the SAFE Act \cite{kehl_right_2015}. Also un-passed
were the related Pro-C0DE Act, Encrypted Communications Privacy Act (not to be confused with the \textit{Electronic}
Communications Privacy Act \acs{ECPA} of 1986), and Encryption Protects the Rights of Individuals from Violation and
Abuse in CYberspace (E-PRIVACY) Act \cite{leahy_s376_1997} \cite{ashcroft_s2067_1998}. This legislative failure meant
that rather than being written in the pen of law, encryption policy was written in the pencil of executive order.

The second development was the events of and reaction to the morning of September 11, 2001. When the U.S. had already
been expanding police surveillance activities to combat communism, crime, drugs, and terrorism, the scale and audacity
of the September 11 attacks to some justified yet greater investigatory powers \cite{bloss_escalating_2007}. October of
2001 saw passage of the USA PATRIOT Act, which had several significant impacts: it weakened the limitation for using
\ac{FISA} requests on U.S. citizens; it expanded the scope of what could be compelled via \ac{FISA} orders; it
authorized ``roving'' wiretaps; and it increased the power of \acp{NSL}, which can be used without judicial review to
compel information from digital service providers while precluding any public disclosure of the event
\cite{sensenbrenner_2001} \cite{shamsi_2011}. The government was performing more surveillance than ever, and even these
weakened limitations would be repeatedly violated \cite{shamsi_2011} \cite{tucker_2020}.

If the weakness of data protection laws and absence of encryption protection laws left the door open for a second crypto
war, and increased government surveillance set events in motion toward conflict, it was the 2013 Snowden revelations
that constituted the crossing of the threshold. Though policies such as the U.S. \ac{NSA}'s warrantless wiretapping
caused a stir, it was the mass data collection under the agency's \ac{PRISM} and related programs that caused the real
public outcry \cite{landau_making_2013}. With the public interest focused on digital privacy, encryption promised a
technical solution. Under pressure from both the human rights field and from ordinary users demanding improved digital
privacy, U.S. tech companies responded by introducing default device encryption for \acl{DAR} and \acl{E2E} encryption
for \acl{DIM} \cite{treguer_us_2018}.

The emergence of strong encryption caused \ac{the-cw1}; its proliferation is causing the second.

The U.S. corporate response to Snowden should not be overly construed as a morally motivated defense of civil rights, as
their behavior is a matter of several practical factors \cite{treguer_us_2018}, not least of which is the pursuit of
basic digital security. Recall encryption and cryptography's critical role in security as described in
\mysec{sec-crypto-basics}, and that encryption's dual role as enabler of strong privacy and cornerstone of computer
security is a central conflict in the \ac{EA} debate. When questioned by lawmakers, technical leaders and executives
have repeatedly appealed to the fact that strong cryptography is a necessity for security \cite{schulze_clipper_2017}.

\subsection{Current Context}
\label{sec-history-current}

Several events and arguments have come to characterize \ac{the-cw2}.

\paragraph*{``Security vs. Security''} The relationship between security and privacy is sometimes viewed as
antagonistic, an assumption implicit in in the ``nothing-to-hide'' argument, but discussions in recent years have shown
that the debate is more about ``Security vs. Security'' than ``Security vs. Privacy''
\cite{stalla_bourdillon_privacy_2014}. In this context, there are two types of security: public security, the pursuit of
law enforcement, and cybersecurity, the pursuit of tech professionals. Both sides of the debate have by now acknowledged
that privacy and public security are not always in conflict, and in fact \ac{EA}'s effects on public security vs.
cybersecurity--which in return affects public security--may be more important \cite{schneier_2019}.

\paragraph*{Apple vs. FBI} The 2015 San Bernardino terror attack in which the suspect was killed, leaving behind a
locked iPhone, resulted in the first major battle of \ac{the-cw2}. The \ac{FBI} issued an order under the \acrlong{AWA}
to compel Apple to unlock the device, which was among the first generation of Apple's fully encrypted iPhones; Apple
objected on grounds that it was ``unreasonably burdensome'' and would undercut the integrity of all iOS devices
\cite{schulze_clipper_2017}. The case occupied U.S. District courts, the media, and the political world's attention from
February 16, 2016, when the warrant was issued, to March 28, when the \ac{FBI} announced they had gained access to the
phone through alternate means \cite{novet_2016}. Even though the phone proved not to have any important data
\cite{schulze_clipper_2017}, the high profile case featured two characteristic elements of the new crypto wars,
terrorism and device encryption. Shortly after the case ended, a pair of Republican and Democratic senators jointly
released a discussion draft of a bill that would have forced Apple to comply \cite{burr_2016}. The draft did not make it
to the floor, and the unresolved nature of the court case left the legal debate wide open.

\paragraph*{\Ac{horsemen}} In the 1990s, an influential encryption advocate coined the phrase ``\ac{horsemen-full}'' to
describe four reasons that governments and law enforcement agencies use to undercut public support for strong encryption
\cite{may_1994}. The traditional list includes terrorists, pedophiles, drug dealers, and money launderers, though
sometimes kidnappers are substituted for money launderers \cite{schneier_scaring_2019}. Reflecting the times,
\ac{the-cw1} emphasized drug trafficking whereas \ac{the-cw2} is emphasizing terrorism and child pornography
\cite{schulze_clipper_2017}. The focus on terrorism is likely a result of culture that emerged from the September 11
terror attacks as well as the San Bernardino terror attack that launched the Apple vs. \ac{FBI} case; the focus on child
pornography is likely a result of the growth of the ``dark web'' and a severe, in-depth 2019 investigation by the
\textit{New York Times} \cite{keller_internet_2019}.

\paragraph*{\ac{DIM} vs. \ac{DAR}} As noted in \mysec{sec-crypto-basics}, securing \acl{DIM} and \acl{DAR} are two
different problems that result in different solutions. Therefore, both technical and policy proposals often split their
recommendations along these lines \cite{group_2019} \cite{owen_law_2018}. The focus from law enforcement in the U.S. has
pendulated between each of these applications a few times \cite{schneier_2019}. The typical \ac{DIM} application is
\ac{E2EE} in instant messaging applications, whereas the typical \ac{DAR} application is \ac{disk-encryption} of mobile
phones. A research group at Carnegie Mellon with participants from both sides of the debate has identified \ac{EA} for
\ac{DAR} as the more tractable problem \cite{group_2019}.

\paragraph*{Volatile Politics} Finally, \ac{the-cw2} is taking place amongst a political landscape marked by
uncertainty, extremism, misinformation, and most recently, a pandemic that will have unknown impacts. Since legislature
is part of the broader data security threat model, its behavior is important. Unfortunately, the dynamics of present
leadership and the lack of reliable data on what law enforcement agencies need make it difficult for the technical
community to engage productively with policy makers \cite{granick_2018}.

That brings us to where the debate stands today. A December 2019 U.S. Senate Judiciary Committee featured the top two
U.S. parties both expressing anger over whether data should be ``beyond the reach of the law'' and expressed threats
like ``get your act together, or we will gladly get your act together for you'' \cite{geller_2019}. In March 2020, the
Republican atop the Senate Judiciary Committee introduced S.3398, the EARN IT Act of 2020, which seeks to establish a
16-chair committee with the power to revoke online platforms' liability protection for user-submitted content if the
committee determines the platform is not doing enough to fight \ac{CSAM}, a legal synonym for
child pornography \cite{graham_s3398_2020}. As illustrated by an in-depth article by the \textit{New York Times}
\cite{keller_internet_2019} that likely triggered the December 2019 hearing, the problem is enormous; however, critics
largely see the EARN IT Act as an attempt to indirectly yet effectively outlaw \ac{E2E} encryption \cite{newman_2020}
\cite{pfefferkorn_2020}.

% NOTE: keep up to date, including the section above mentioning the pandemic
% IDEA: create timeline! https://tex.stackexchange.com/questions/196794/how-can-you-create-a-vertical-timeline

The following sections summarize the encryption regulatory environment and technical approaches to exceptional access.



\section{Regulatory Environment}
\label{sec-reg-environment}

\subsection{U.S. Efforts to Regulate Encryption}

The following laws and regulations have been applied to cases involving surveillance, cryptography, and access to
encrypted information:

\begin{itemize}
    \item \acf{AWA} of 1789 \cite{congress_1789}
    \item Bill of Rights: Amendments I, IV, and V \cite{madison_1791}
    \item \acf{AECA} of 1976 established export controls over strong encryption techniques
        introduced ITAR (International Traffic in Arms Regulations) \cite{morgan_hr13680_1976} \cite{kehl_right_2015}
    \item \acf{FISA} of 1978 \cite{rodino_1978}
        (FISA has been the subject of frequent FBI abuse: \cite{shamsi_2011} \cite{tucker_2020})
        % (the FBI almost lost some provisions; follow up on https://apnews.com/110ff653e98478da66765897dde3d613,
        %     https://www.washingtontimes.com/news/2020/mar/15/adam-schiff-fisa-reform-intervention-protects-fbi-/,
        %     https://saraacarter.com/fisa-senate-passes-measure-extending-surveillance-powers-abuses-not-addressed/)
    \item \acf{CFAA} of 1986 \cite{hughes_hr4718_1986} \cite{wolff_computer_2016}
    \item \acf{ECPA} of 1986 \cite{kastenmeier_hr4952_1986}
    \item \acf{CALEA} of 1994 \cite{edwards_hr4922_1994}
    \item Uniting and Strengthening America by Providing Appropriate Tools Required to Intercept and Obstruct Terrorism
        (USA PATRIOT ACT) Act of 2001 \cite{sensenbrenner_2001}
    \item \acf{PAA} of 2007 \cite{mcconnell_s1927_2007}
    \item \acf{FISAAA} of 2008 \cite{reyes_hr6304_2008}
    \item Uniting and Strengthening America by Fulfilling Rights and Ensuring Effective Discipline Over Monitoring
        (USA FREEDOM) Act of 2015 \cite{sensenbrenner_2015}
    \item \dots
\end{itemize}

The following are proposed laws and regulations that did not go into effect or have yet to be voted on:

\begin{itemize}
    \item The \ac{clipperchip} \cite{press_1993} \cite{thompson_2015}
    \item The SAFE Act \cite{goodlatte_hr3011_1996}
    \item Encrypted Communications Privacy Act (mix of SE protection and EA guidelines?) \cite{leahy_s376_1997}
    \item Encryption Protects the Rights of Individuals from Violation and Abuse in CYberspace (E-PRIVACY) Act
        (like above?) \cite{ashcroft_s2067_1998}
    \item Encryption for the National Interest Act and Tax Relief for Responsible Encryption Act (introduced by same rep
        on same day) \cite{goss_hr2616_1999} \cite{goss_hr2617_1999}
    \item Compliance with Court Orders Act of 2016 (discussion draft) \cite{burr_2016}
    \item The Secure Data Act (federal SE protections) \cite{lofgren_hr5823_2018}
    \item The ENCRYPT Act (state SE protections) \cite{lieu_hr4170_2019}
    \item The EARN IT Act \cite{graham_s3398_2020}
    \item \dots
\end{itemize}

\subsection{Regulations around the World}

The following is a very brief account of the regulatory environment in various jurisdictions around the world:

\begin{itemize}
    \item Australia: in 2018 passed its Assistance and Access bill \cite{ag_2018} \cite{newman_2018}
    \item UK: Investigatory Powers Act \cite{legislature_2016}
    \item EU: GDPR \cite{parliament_2016}
    \item China: \cite{donahue_comparative_2018}
    \item Israel: \cite{donahue_comparative_2018}
    \item Japan, Korea? -- can't find anything
    \item \dots General, see \cite{budish_encryption_2018}
\end{itemize}


\section{Technical Approaches}
\label{sec-tech-approaches}

\subsection{Historical Approaches}

Historical approaches and proposals out of \ac{the-cw1} that saw various amounts of support.

\begin{itemize}
    \item Clipper chip/third-party key escrow \cite{blaze_protocol_1994}
    \item Oblivious key escrow \cite{goos_oblivious_1996}
    \item Partial key escrow (see \cite{denning_taxonomy_1996})
\end{itemize}

\subsection{Modern Proposals}

Recent proposals out of \ac{the-cw2} that have various levels of support.

\begin{itemize}
    \item CLEAR \cite{ozzie_2018}
    \item Device escrow with HW \cite{savage_lawful_2018}
    \item Computational, blockchain approaches \cite{phan_key_2017}
    \item Lawful hacking \cite{nguyen_lawful_2017} \cite{soesanto_2018} \cite{kerr_encryption_2017}
    \item Forced password disclosure \cite{bittenbender_2019} \cite{kerr_encryption_2017}
    \item Alternative paths to plaintext (access device while unlocked; find copies elsewhere)
        \cite{kerr_encryption_2017}
\end{itemize}



\section{Argument Maps}
\label{sec-arg-maps-intro}

In order to advance the \ac{EA} debate, it helps to organize the all the arguments used in a cohesive manner. One way to
organize arguments is to create argument maps. Argument maps are graphical representations of the logical structures and
relationships between statements, premises, and conclusions. Argument maps are an outgrowth of the same research by
Horst Rittel that originated the idea of \acp{wicked-prob}. In 1970 his research group developed \acp{IBIS} to break
down \acp{wicked-prob} and document the reasoned approach that goes into solving them \cite{kunz_issues_1970}. Interest
in \ac{IBIS} has waxed and waned. Some software tools have been developed, including the now-defunct gIBIS
\cite{conklin_gibis_1988} and Compendium \cite{dutoit_hypermedia_2006} tools designed to produce graphical
representations. Argdown is a recent tool that generates argument maps from structures specified in a Markdown-based
language \cite{voigt_argdown_2018}.

This paper uses Argdown to analyze the arguments used in the \ac{EA} debate. \myfig{fig-args-demo} shows the basic
structure of arguments. Statements are positions on issues. Some statements are based on values that cannot be addressed
through argument. Statements and arguments may support or attack one another, and arguments support or undercut other
arguments.

\begin{figure}[ht]
    \centering\CaptionFontSize
    \includegraphics[width=\linewidth]{arguments/build/demo.pdf}
    \caption[Argdown Demo]{A demonstrative example of argument maps with Argdown.}
    \label{fig-args-demo}
\end{figure}

% IDEA:
% Use the slightly more complete IBIS notation, which includes ISSUES and POSITIONS as well as ARGUMENTS.
% (Argdown) basically just has arguments and positions. The additional types may make the total picture clearer.
% Either use styling with #hashtags (https://argdown.org/syntax/#hashtags with
%                                    https://argdown.org/guide/colorizing-maps.html#the-colorscheme-setting)
% Or add a plugin (https://argdown.org/guide/extending-argdown.html)



\section{Threat Models and Data Flow Diagrams}
\label{sec-threat-model-intro}

Threat modeling is an important step in analyzing the security at the systems level. Using models abstracts away fine
details and focuses on the architecture, processes, and dataflows in a system. Models assist in the understanding of
current systems and the prevention of problems in new systems, both of which are important when facing the prospect of
designing an \ac{EA} mechanism built on top of a complex and aging technology stack.

Threat modeling begins with the questions, ``What are you building?'' and ``What can go wrong?''
\cite{shostack_threat_2014}. Answering the first question accurately is crucial, particularly for the field of
cryptography, which hinges on precise definitions of security requirements \cite{varia_2018}. Answering the second
question depends on the types of attackers to worry about, and determines the scope of threats to be considered. ``What
are you building?'' hasn't been asked enough in the current phase of the \ac{EA} debate, and ``What can go wrong?''
cannot be faithfully answered without knowing what is being built.

\Acp{DFD} are one tool designed to address these questions. \Acp{DFD} feature processes, data flows, data stores, and
external entities. \Acp{DFD} well suited for threat modeling because security vulnerabilities tend to follow data flow,
not control flow \cite{shostack_threat_2014}. They are particularly well suited for \ac{EA}, as data privacy is the
primary concern. \myfig{fig-dfd-demo} shows the basic elements of a \ac{DFD}. This paper creates \acp{DFD} in this style
to analyze the threats already present and those that \ac{EA} would introduce.

\begin{figure}[ht]
    \centering\CaptionFontSize
    \includegraphics[width=\linewidth]{dfds/build/DFD-demo.png}
    \caption[\Acs{DFD} Demo]{A demonstrative example of a \acf{DFD}.}
    \label{fig-dfd-demo}
\end{figure}
