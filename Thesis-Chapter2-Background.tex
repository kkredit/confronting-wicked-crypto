\chapter{Background}
\label{chap-background}

Private use of encryption has a contentious history. [Some highlights.]

Electronic encryption must be understood as a technological development embedded in the longer history of privacy and
human rights. [This is where I'd like to understand more of the legal side. With a historical perspective.]

Privacy is not encryption's only purpose, however. Encryption is the technical foundation for many forms of computer and
network security. [Properties: confidentiality, authentication, integrity, non-repudiation; primitives: hashed
passwords, digital signatures, ...; technologies: PKI, TLS, SSO/Kerberos, VPNs, disk encryption, blockchains, ...]

Electronic encryption's dual role as enabler of perfect privacy and cornerstone of computer security may be the central
conflict in the exceptional access debate. Those who focus on its role in privacy object to its absolute power. Those
who focus on its role in security rely on its absolute power.

% Encryption requires absolute cryptographic fidelity in order to provide security.

\section{Exceptional Access: Arguments For and Against}
\label{sec-argsbackground}

The debate over exceptional access takes place on two fundamental levels:
\begin{enumerate}
    \item Whether cryptosystems with exceptional access represent a net benefit to society
    \item Whether and how to conduct exceptional access research
\end{enumerate}

These arguments are introduced here and analyzed in detail in \mychap{chap-arguments}.
[This section is currently written in outline-format, but will be drafted in a narrative, introductory style, as these
arguments are the focus of a later chapter.]

\subsection{Arguments For}

Why EA is a benefit to society:
\begin{itemize}
    \item Absolute privacy is not a right
    \item Encryption gives too much power to the individual
    \item EA would allow more crimes to be prevented and prosecuted
\end{itemize}

Why to research EA:
\begin{itemize}
    \item EA done well would be more secure and privacy preserving than EA done poorly
    \item The threat of legislative action creates urgency
\end{itemize}

\subsection{Arguments Against}

Why EA is a danger to society:
\begin{itemize}
    \item Privacy is an absolute right
    \item Exceptional access gives too much power to the state
    \item Exceptional access is not possible in an acceptably secure manner
\end{itemize}

Why not to research EA:
\begin{itemize}
    \item With both security and privacy in a poor state, research should be focused on strengthening, not weakening,
            them
    \item The technical community should not entertain delusions about what is truly possible
\end{itemize}

\section{Regulatory Environment}

\subsection{US Efforts to Regulate Encryption}

\subsection{Regulations around the World}

\section{Technical Approaches}

\subsection{Historical Approaches}

Historical approaches and proposals out of the first crypto war that saw various amounts of support.

\begin{itemize}
    \item Clipper chip
    \item Government key escrow
    \item Oblivious key escrow \cite{goos_oblivious_1996}
    \item Partial key escrow (see \cite{denning_taxonomy_1996})
\end{itemize}

\subsection{Modern Proposals}

Recent proposals out of the second crypto war that have various levels of support.

\begin{itemize}
    \item CLEAR \cite{ozzie_2018}
    \item Device escrow with HW \cite{savage_lawful_2018}
    \item Computational, blockchain approaches \cite{phan_key_2017}
    \item Lawful hacking \cite{nguyen_lawful_2017} \cite{soesanto_2018}
    \item Forced password disclosure \cite{bittenbender_2019}, see \cite{kerr_encryption_2017}
\end{itemize}
