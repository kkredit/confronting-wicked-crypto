\chapter{Background}
\label{chap-background}

This chapter summarizes the history of encryption and the exceptional access debate, introduces argument maps as a
logical modeling tool, and introduces threat modeling with data flow diagrams.

\section{Historical Context}
\label{sec-history}

\subsection{Encryption History and the First Crypto War}

Electronic encryption is a new technology, but the field of cryptography is old, as demand for privacy is as old as
communication itself. The most well known ancient example of rudimentary encryption is the Caesar cipher, named after
the character transposition technique Julius Caesar used to protect private correspondence
\cite{luciano_cryptology_1987}. Significant developments include the first formal cryptographic study by Arab scholars
in the eighth century, advancements made out of necessity in the twentieth century's world wars, and the application of
computers to cryptographic problems \cite{kahn_codebreakers_1996}. Claude Shannon's formalized the modern ``mathematical
analysis of cryptography'' in 1949 \cite{shannon_communication_1949}, and Whitfield Diffie and Martin Hellman published
research on public key cryptography in 1976 \cite{diffie_new_1976}.

Public key cryptography, paired with computer networking and personal computing advances in the 1980s, for the first
time put the power of strong encryption in the hands of ordinary people. However, the privacy provided by cryptography
is a powerful tool, and privacy is a property governments strive to regulate. The use of encryption for for private
purposes has a contentious history. In 1587, Mary Queen of Scots was convicted of treason based on evidence from
decrypted letters, and in 1807, prosecutors trying Aaron Burr for treason tried to force testimony from his secretary on
the contents of encrypted messages \cite{kerr_encryption_2017}. Government's natural stance towards privacy makes it
inherently uncomfortable with the rapid increase in availability of strong encryption.

% Electronic encryption must be understood as a technological development embedded in the longer history of privacy and
% human rights. [This is where I'd like to understand more of the legal side. With a historical perspective.]

Privacy is not encryption's only purpose, however.

% Spoofing                  Authentication
% Tampering                 Integrity
% Repudiation               Non- Repudiation
% Information Disclosure    Confidentiality
% Denial of Service         Availability
% Elevation of Privilege    Authorization

Cryptography is the technical foundation for many forms of computer and network security, at every scale. Security is
often defined in terms of several properties: authentication, integrity, non-repudiation, confidentiality, availability,
and authorization \cite{shostack_threat_2014}. Security properties are systems properties, but systems properties emerge
from the composition of their parts, and innumerable primitives and technologies rely on cryptography.

Cryptographic primitives include symmetric key encryption, public key encryption, one-way hash functions, and digital
signatures. A few technologies that directly rely on these primitives include TLS (transport layer security), PKI
(public key infrastructure), SSO (single sign on, e.g. Kerberos), VPNs (virtual private networks), secure password
storage, disk encryption tools, and secure- and trusted-boot processes. A few systems that directly rely on these
technologies include the financial system, local and global; industrial control systems and critical infrastructure;
health information systems; private, government, and military communications; and the integrity of the internet as a
whole.

If the cryptographic primitives that compose these systems were to be compromised, the systems they comprise could
collapse. If the financial system loses integrity, health information systems lose confidentiality, critical
infrastructure loses availability, government communications lose authenticity and non-repudiation, or the internet as a
whole loses integrity, inconceivably large damage would occur.

% Let us break security down into three scales: primitives, technologies, and systems. A primitive is a conceptual
% building block that serves a single technical function. A technology is a set of implemented primitives glued together
% to provide useful features. A system is a collection of technologies that collectively serve the ultimate outcome.
% The security properties critical at the systems level depend

% Primitives: one-way hashes enable integrity checks, block ciphers enable efficient confidential data storage, and
% digital signature enable non-repudiation

% Technologies: Transport Layer Security (TLS) enables confidentiality and integrity over the internet, Kerberos enables
% convenient authentication, and virtual private networks (VPNs) enable confidential and authenticated remote network
% connections

% Systems: ...

Electronic encryption's dual role as enabler of strong privacy and cornerstone of computer security may be the central
conflict in the exceptional access debate. Those who focus on its role in privacy object to its absolute power. Those
who focus on its role in security rely on its absolute power.

% Encryption requires absolute cryptographic fidelity in order to provide security.

All these realities--government discomfort with absolute privacy, rapidly increasing availability of strong encryption,
and a blossoming computer industry foundationally reliant on cryptography--came to a head in what has become known as
the First Crypto War.

In 1976, the same year Diffie and Hellmann published their research in public key cryptography, the U.S. Congress passed
the Arms Export Control Act and declares that strong cryptography is subject to export controls \cite{kehl_right_2015}.
In 1991, the U.S. Senate introduced, but did not pass, a bill mandating access to plain text contents when authorized by
law; in response, Philip Zimmermann released PGP (Pretty Good Privacy), a public key encryption tool to secure email, in
order that strong cryptography be ``made available to the American public before it became illegal to use''
\cite{zimmermann_1996}. In 1993, the Clinton administration introduced Clipper Chip \cite{press_1993} with the goal of
``providing the public with strong cryptographic tools without sacrificing the ability of law enforcement and
intelligence agencies to access unencrypted versions of those communications'' \cite{thompson_2015}. Citing the
foundational role of cryptography in security--and the potential of human rights abuses for compromised
privacy--industry and prominent technical leaders reacted negatively to the initiative \cite{kehl_right_2015}
\cite{zimmermann_1996}. When security researcher Matt Blaze discovered flaws that allowed users to subvert the Clipper
Chip mechanisms \cite{blaze_protocol_1994}, the proposal died.

Despite the failure of the Clipper Chip, the debate over access to strong encryption continued, primarily focused on
various key escrow proposals \cite{thompson_2015}. In 1996, two bills were introduced in the US Congress. In the Senate,
it was S.1726, the Pro-C0DE Act of 1996, to abolish the export controls on encryptions software \cite{burns_s1726_1996}.
In the House, it was HR.3011, the SAFE Act of 1996, to explicitly allow arbitrarily strong encryption for all legal
activity \cite{goodlatte_hr3011_1996}. Indeed, the SAFE Act went further, including ``provisions that would have
barred the government from creating a mandatory key escrow system and would also have removed export restrictions on
most generally available software and hardware containing encryption'' \cite{kehl_right_2015}. Zimmermann, the author of
PGP, who had by this time endured an investigation by the Customs Service for publishing his work, testified before the
Senate in favor of the Pro-C0DE Act \cite{zimmermann_1996}. The the Pro-C0DE Act passed, but the SAFE Act did not.

From 1996 to 1999, the SAFE Act was proposed several times, had hearings, and gained support \cite{kehl_right_2015}. By
this time, there was ``an overwhelming amount of evidence against moving ahead with any key escrow schemes''
\cite{thompson_2015}, and in 1999 the Clinton administration abruptly changed course, adopting almost all SAFE Act
proposals \cite{kehl_right_2015}. This development marks the end of the First Crypto War.

\subsection{The Second Crypto War and the Current Context}

Two developments sewed the seeds for the Second Crypto War. The first was actually a non-development: when the Clinton
administration changed its encryption policy, the U.S. House dropped the SAFE Act \cite{kehl_right_2015}. This meant
that rather than being written in the pen of law, encryption policy was written in the pencil of executive order. The
second development was the events of and reaction to morning of September 11, 2001.

Three developments have triggered the Second Crypto War.
\begin{enumerate}
    \item The 2013 Snowden revelations
    \item The mid-2010s spread of default device encryption in smartphones
    \item The late-2010s spread of default E2E encryption in messaging services
\end{enumerate}

Characteristic elements of the Second Crypto War include:
\begin{enumerate}
    \item The Apple v. FBI case over access to the San Bernardino shooter's iPhone \cite{schulze_clipper_2017}
    \item A return to the Four Horsemen \cite{schneier_scaring_2019}, with a (warranted \cite{keller_internet_2019})
            focus currently on CSAM \cite{geller_2019}
    \item ``Security vs Security'' vs ``Security vs Privacy'' \cite{stalla_bourdillon_privacy_2014} \cite{schneier_2019}
    \item Pendulating emphasis on DIM/E2E and DAR/DE \cite{schneier_2019}
    \item A political landscape marked by uncertainty, extremism, misinformation, and most recently, a pandemic that
            will have unknown impacts
\end{enumerate}

That brings us to where the debate stands today. A December 2019 U.S. Senate Judiciary Committee featured the top two
U.S. parties both expressing anger over whether data should be ``beyond the reach of the law'' and expressed threats
like ``get your act together, or we will gladly get your act together for you'' \cite{geller_2019}. In March 2020, the
Republican atop the Senate Judiciary Committee introduced S.3398, the EARN IT Act of 2020, which seeks to establish a
16-chair committee with the power to revoke online platforms' liability protection for user-submitted content if the
committee determines the platform is not doing enough to fight child sexual abuse material (CSAM), a legal synonym for
child pornography \cite{graham_s3398_2020}. As illustrated by an in-depth article by the \textit{New York Times}
\cite{keller_internet_2019} that likely triggered the December 2019 hearing, the problem is enormous; however, critics
largely see the EARN IT Act as an attempt to indirectly yet effectively outlaw E2E \cite{newman_2020}
\cite{pfefferkorn_2020}.

% TODO: create timeline! https://tex.stackexchange.com/questions/196794/how-can-you-create-a-vertical-timeline

The following sections summarize the encryption regulatory environment and technical approaches to exceptional access.

\subsection{Regulatory Environment}
\label{sec-reg-environment}

\subsubsection*{US Efforts to Regulate Encryption}

The following laws and regulations have been applied to cases involving cryptography and access to encrypted
information:

\begin{itemize}
    \item The All Writs Act \cite{congress_1789}
    \item The Bill of Rights: Amendments I, IV, and V \cite{madison_1791}
    \item The Arms Export Control Act (AECA) of 1976 established export controls over strong encryption techniques
        introduced ITAR (International Traffic in Arms Regulations) \cite{morgan_hr13680_1976} \cite{kehl_right_2015}
    \item The Computer Fraud and Abuse Act (CFAA) of 1986 \cite{hughes_hr4718_1986} \cite{wolff_computer_2016}
    \item The Communications Assistance for Law Enforcement Act (CALEA) of 1994 \cite{edwards_hr4922_1994}
    \item \dots
\end{itemize}

The following are proposed laws and regulations that did not go into effect or have yet to be voted on:

\begin{itemize}
    \item The Clipper Chip \cite{press_1993} \cite{thompson_2015}
    \item The SAFE Act \cite{goodlatte_hr3011_1996}
    \item The EARN IT Act \cite{graham_s3398_2020}
    \item \dots
\end{itemize}

\subsubsection*{Regulations around the World}

The following is a very brief account of the regulatory environment in various jurisdictions around the world:

\begin{itemize}
    \item Australia: in 2018 passed its Assistance and Access bill \cite{ag_2018} \cite{newman_2018}
    \item UK: Investigatory Powers Act \cite{legislature_2016}
    \item EU: GDPR
    \item China: \cite{donahue_comparative_2018}
    \item Israel: \cite{donahue_comparative_2018}
    \item Japan, Korea? -- can't find anything
    \item \dots
\end{itemize}

\subsection{Technical Approaches}
\label{sec-tech-approaches}

\subsubsection*{Historical Approaches}

Historical approaches and proposals out of the first crypto war that saw various amounts of support.

\begin{itemize}
    \item Clipper chip/third-party key escrow \cite{blaze_protocol_1994}
    \item Oblivious key escrow \cite{goos_oblivious_1996}
    \item Partial key escrow (see \cite{denning_taxonomy_1996})
\end{itemize}

\subsubsection*{Modern Proposals}

Recent proposals out of the second crypto war that have various levels of support.

\begin{itemize}
    \item CLEAR \cite{ozzie_2018}
    \item Device escrow with HW \cite{savage_lawful_2018}
    \item Computational, blockchain approaches \cite{phan_key_2017}
    \item Lawful hacking \cite{nguyen_lawful_2017} \cite{soesanto_2018} \cite{kerr_encryption_2017}
    \item Forced password disclosure \cite{bittenbender_2019} \cite{kerr_encryption_2017}
    \item Alternative paths to plaintext (access device while unlocked; find copies elsewhere)
        \cite{kerr_encryption_2017}
\end{itemize}

\section{Argument Maps}
\label{sec-arg-maps-intro}



\section{Threat Models and Data Flow Diagrams}
\label{sec-threat-model-intro}
