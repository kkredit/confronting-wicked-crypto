%%
%% The first command in your LaTeX source must be the \documentclass command.
\documentclass[manuscript,screen,review]{acmart}

%%
%% \BibTeX command to typeset BibTeX logo in the docs
\AtBeginDocument{%
  \providecommand\BibTeX{{%
    \normalfont B\kern-0.5em{\scshape i\kern-0.25em b}\kern-0.8em\TeX}}}

%% Rights management information.  This information is sent to you
%% when you complete the rights form.  These commands have SAMPLE
%% values in them; it is your responsibility as an author to replace
%% the commands and values with those provided to you when you
%% complete the rights form.
%% TODO: update with real info.
\setcopyright{acmcopyright}
\copyrightyear{2021}
\acmYear{2021}
\acmDOI{10.1145/1122445.1122456}

%%
%% These commands are for a JOURNAL article.
%% TODO: update with real info.
\acmJournal{TOPS}
\acmVolume{0}
\acmNumber{1}
\acmArticle{2}
\acmMonth{3}

%%
%% Submission ID.
%% Use this when submitting an article to a sponsored event. You'll
%% receive a unique submission ID from the organizers
%% of the event, and this ID should be used as the parameter to this command.
%%\acmSubmissionID{123-A56-BU3}

%%
%% The majority of ACM publications use numbered citations and
%% references.  The command \citestyle{authoryear} switches to the
%% "author year" style.
%%
%% If you are preparing content for an event
%% sponsored by ACM SIGGRAPH, you must use the "author year" style of
%% citations and references.
%% Uncommenting
%% the next command will enable that style.
%%\citestyle{acmauthoryear}

%%
%% end of the preamble, start of the body of the document source.
\begin{document}

%%
%% The "title" command has an optional parameter,
%% allowing the author to define a "short title" to be used in page headers.
\title{Threat Modeling for a Lawful Device Access Proposal}

%%
%% The "author" command and its associated commands are used to define
%% the authors and their affiliations.
%% Of note is the shared affiliation of the first two authors, and the
%% "authornote" and "authornotemark" commands
%% used to denote shared contribution to the research.
\author{Kevin Kredit}
\email{k.kredit.us@ieee.org}
% \orcid{1234-5678-9012}
\affiliation{%
  \institution{Grand Valley State University}
  \streetaddress{318C DeVos Center, 401 W. Fulton Street}
  \city{Grand Rapids}
  \state{Michigan}
  \country{USA}
  \postcode{49504}
}
% \author{G.K.M. Tobin}
% \authornote{Both authors contributed equally to this research.}
% \authornotemark[1]
% \email{webmaster@marysville-ohio.com}
% \affiliation{%
% }

%%
%% By default, the full list of authors will be used in the page
%% headers. Often, this list is too long, and will overlap
%% other information printed in the page headers. This command allows
%% the author to define a more concise list
%% of authors' names for this purpose.
\renewcommand{\shortauthors}{Kredit}

%%
%% The abstract is a short summary of the work to be presented in the
%% article.
\begin{abstract}
  Abstract text.
\end{abstract}

%%
%% The code below is generated by the tool at http://dl.acm.org/ccs.cfm.
%% Please copy and paste the code instead of the example below.
%%
\begin{CCSXML}
<ccs2012>
  <concept>
    <concept_id>10002978.10003006.10011608</concept_id>
    <concept_desc>Security and privacy~Information flow control</concept_desc>
    <concept_significance>500</concept_significance>
  </concept>
  <concept>
    <concept_id>10002978.10003029.10003032</concept_id>
    <concept_desc>Security and privacy~Social aspects of security and privacy</concept_desc>
    <concept_significance>300</concept_significance>
  </concept>
    <concept>
    <concept_id>10003456.10003462.10003487</concept_id>
    <concept_desc>Social and professional topics~Surveillance</concept_desc>
    <concept_significance>300</concept_significance>
  </concept>
  <concept>
    <concept_id>10003456.10003462.10003588.10003589</concept_id>
    <concept_desc>Social and professional topics~Governmental regulations</concept_desc>
    <concept_significance>300</concept_significance>
  </concept>
</ccs2012>
\end{CCSXML}

\ccsdesc[500]{Security and privacy~Information flow control}
\ccsdesc[300]{Security and privacy~Social aspects of security and privacy}
\ccsdesc[300]{Social and professional topics~Surveillance}
\ccsdesc[300]{Social and professional topics~Governmental regulations}

%%
%% Keywords. The author(s) should pick words that accurately describe
%% the work being presented. Separate the keywords with commas.
\keywords{encryption, exceptional access, public policy, threat modeling}

\maketitle

\section{Introduction}

Begin article text.

\end{document}
\endinput
