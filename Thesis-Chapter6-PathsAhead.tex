\chapter{Paths Ahead}
\label{chap-pathsahead}

The interests of security, safety, privacy, and trust are converging on the issue of encryption and \acl{EA} and
creating a \ac{wicked-prob}. Information security is increasingly important as more data, business, and infrastructure
goes online. The technical community is struggling with security; defending against the average attacker is hard, and
against the best attackers is nearly impossible, even for the most motivated and well-resourced organizations.
\Ac{cryptography} plays a fundamental role in security, yet its simultaneous role in enabling unprecedented privacy
means it has controversial side effects.

One of these side effects is encryption-enabled privacy's impact on public safety. Despite a shortage of quantitative
analyses showing its impact on crime---something that would be impossible to fully quantify, after all---encryption
decidedly withholds evidence of wrongdoing from service providers and law enforcement, enabling truly harmful activity
to go unchecked, and decreasing public safety. This a legitimate concern and the primary motivation for \ac{EA}.

However, at the same time as encryption is enabling strong privacy, the broader technological path society is taking is
stripping it away. The ubiquity and insecurity of networked computers has been a bear to privacy, enabling threats
ranging from identity theft, to inappropriate data mining, to mass surveillance. To the extent that encryption mitigates
these, it increases public safety.

Finally, choosing how to respond to these issues---whether with inaction, \ac{EA}, or some alternative---involves
considerable questions of trust. Law enforcement agencies, intelligence agencies, technology makers, individual users,
and the public at large are all players; each have demonstrated untrustworthiness, and different approaches disperse
trust differently.

% TODO: one paragraph here^^ or four?

This complex mix of issues is what makes encryption policy a \ac{wicked-prob}. To merely formulate the problem requires
explanation upon explanation and research upon research. Any formulation is concomitant with a solution, but the outcome
of a solution cannot be accurately forecasted or independently evaluated. The underlying conditions are constantly in
flux, affected by external forces as well as every policy action and inaction.

In this thesis I propose a method to contend with \acp{wicked-prob}, a modified \ac{OODAloop} that emphasizes
collaborative debate between individuals diverse in background and values, facilitated in part by argument maps, and
focusing on specific proposals. I try to capture that here by surveying the debate, mapping the arguments, and
ultimately performing threat model analysis of a specific \ac{EA} proposal for \acl{DAR}.


% What can we conclude?
%   - It may not feel like it, but (meaningful) debate and discussion counts as progress for a wicked problem
%   - Neither side has the whole picture, but the "no EA" arguments are currently stronger
%   - A threat model including malicious users and threatening governments makes EA make more sense
%   - Device escrow with secure hardware represents true progress towards acceptably secure EA, so hope is not lost
% Where do we go from here?
%   - This was an attempt an "OO" iteration in the OODA; we need more, with involvement all around
%   - committee_decrypting_2018 presented a framework to use to judge proposals. I have a simple version of it.
%     here are where we are weak:
%       - X
%       - Y
%       - Z
%   - Government:
%% Prof. Walsh:
%% The dynamic regulatory environment and competing governmental branches at the executive, legislative, and judicial
%% level will continue to shape the...
%       - Take the low hanging fruit first, please
%       - Fund research, don't mislead, don't be premature
%           - If you're going to compare it to going to the moon, fund us like it!
%       - Take on weak area XYZ
%   - Tech:
%       - Do research, provide clear security advice, be vigilant
%       - Take on weak area XYZ
%   - Either:
%       - Know that there is no stable solution, balance rules and principles, serve your neighbor
% see varia_2018 (maybe a good quote?)




%%%%%%%%%%%%%%%%%%%%%%%%%
% OLD NOTES for reference
%%%%%%%%%%%%%%%%%%%%%%%%%

% In this chapter, describe actions to take in discussion, regulation, and research.

% \section{Legislative Priorities}

% \begin{itemize}
%     \item Reject EA until technical capabilities are ready \cite{varia_2018}
%     \item Be forthcoming with data so that we know what to work on \cite{devlin_2018}
%     \item Don't prescribe technical methods, but be clear about principles \cite{matyas_incommensurability_2018}
%         (e.g., ECPA became outdated because was too tied to specific technology \cite{shamsi_2011})
%     \item Fund the research \cite{varia_2018} (like \cite{goss_hr2616_1999})
% \end{itemize}

% \section{Technical Research}

% \begin{itemize}
%     \item Acknowledge the moral character of cryptography \cite{rogaway_moral_2015}
%     \item Do the research
%     \item Focus on principles \cite{levy_robinson_2018}
%     \item Be clear on definitions \cite{varia_2018}, but don't expect govt to provide those (unlike
%             \cite{abelson_2015}), and clearly define their interfaces with principles
%             \cite{matyas_incommensurability_2018}
% \end{itemize}
