\chapter{Analysis Tools}
\label{chap-tools}


% FIXME: include a more descriptive key to understand argument maps.

% - Prof. Dulimarta: Could you apply fuzzy logic to the argument map? Quantify it? -- I think this would run into limits
%   once you start comparing unquantifiable things, but it could be used to suss out the valid arguments, and would
%   probably be really interesting otherwise.

% Kalafut: I'm not sure that 2.5 and 2.6 really fit well as subsections of 2. Consider separating them into their own
% (short) chapter.
% Kalafut: Argument maps are used in chapter 4. I'm not sure that you need to mention argdown at all. If you do, I would
% only do so in the context of stating that it is the tool you are using to create argument maps.


\section{Argument Maps}
\label{sec-arg-maps-intro}

The arguments used in the \ac{EA} debate are diverse and complex; in order to advance the \ac{EA} discussion, it would
help to organize all the arguments used in a cohesive manner. One way to organize arguments is to create \acp{argmap}.
\Acp{argmap} are graphical representations of the logical structures and relationships between statements, premises, and
conclusions. \Acp{argmap} are an outgrowth of the same research by Horst Rittel that originated the idea of
\acp{wicked-prob}. In 1970 his research group developed \acp{IBIS} to break down \acp{wicked-prob} and document the
reasoned approaches that go into solving them \cite{kunz_issues_1970}. Interest in \ac{IBIS} has waxed and waned. Some
software tools have been developed, including the now-defunct gIBIS \cite{conklin_gibis_1988} and Compendium
\cite{dutoit_hypermedia_2006} tools designed to produce graphical representations. Argdown is a recent tool that
generates \acp{argmap} from structures specified in a Markdown-based language \cite{voigt_argdown_2018}.

Argdown is used in \mychap{chap-arguments} to analyze the arguments used in the \ac{EA} debate. \myfig{fig-args-demo}
shows the basic structure of arguments. Statements are positions on issues. Some statements are based on values that
cannot be addressed through argument. Statements and arguments may support or attack one another, and arguments support
or undercut other arguments.

\begin{figure}[ht]
    \centering\CaptionFontSize
    \myincludeargument{demo}
    \caption{A demonstrative example of argument maps with Argdown}
    \label{fig-args-demo}
\end{figure}

%%%%%%%%%%%%%%%%% Hack job

Recall from \mysec{sec-arg-maps-intro} that \acp{argmap} resulted from the same research that produced the concept of
\acp{wicked-prob}. Though formal research on the technique is thin, \acp{argmap} have proven helpful in simplifying
complex arguments around \acp{wicked-prob} \cite{renton_2007} and in facilitating debate by helpfully slowing down
discussion, depersonalizing conflict, and lending structure and rhythm \cite{dutoit_hypermedia_2006}. This makes them
well suited as a tool for conducting the collaborative debate that helps problem solvers with disparate views narrow in
on a problem definition and concrete proposals on which they can agree.

% FIXME: Kalafut: It's not clear to me that this paragraph is a good fit here. It is good content. Perhaps this
% paragraph or its ideas should be at the start of the Argument Maps section.
% PLAN: I will move this to when I talk more about argument maps. After the refactor, they won't even be introduced yet
% here.


\section{Threat Models and Data Flow Diagrams}
\label{sec-threat-model-intro}

Threat modeling is an important step in analyzing the security at the systems level. Using models abstracts away fine
details and focuses on the architecture, processes, and dataflows in a system. Models assist in the understanding of
current systems and the prevention of problems in new systems, both of which are important when facing the prospect of
designing an \ac{EA} mechanism built on top of a complex and aging technology stack.

Threat modeling begins with the questions, ``What are you building?'' and ``What can go wrong?''
\cite{shostack_threat_2014}. Answering the first question accurately is crucial, particularly for the field of
cryptography, which hinges on precise definitions of security requirements \cite{varia_2018}. Answering the second
question depends on the types of attackers to worry about, and determines the scope of threats to be considered. ``What
are you building?'' hasn't been asked enough in the current phase of the \ac{EA} debate, and ``What can go wrong?''
cannot be faithfully answered without knowing what is being built.

\Acp{DFD} are one tool designed to address these questions. \Acp{DFD} feature processes, data flows, data stores, and
external entities. \Acp{DFD} are well suited for threat modeling because security vulnerabilities tend to follow data
flow, not control flow \cite{shostack_threat_2014}. They are particularly well suited for \ac{EA}, as data privacy is of
primary concern. \myfig{fig-dfd-dh-traditional} shows the basic elements of a \ac{DFD} of a simplified Diffie-Hellman
key exchange.

\begin{figure}[p]
    \centering\CaptionFontSize
    \includegraphics[width=0.8\linewidth]{dfds/build/diffie-hellman-traditional.png}
    \caption{A \acf{DFD} for a simplified Diffie-Hellman key exchange}
    \label{fig-dfd-dh-traditional}
\end{figure}

\Acp{DFD} show what data is transferred via labels on the flows. However, labels can be insufficient in complex
\acp{DFD}, and the method has no syntax for computation or cryptographic protocols. In order to maximize their
communicative ability in the context of encryption and \acl{EA} schemes, I introduce the additional syntax show in
\myfig{fig-dfd-dh-updated}.

\begin{figure}[p]
    \centering\CaptionFontSize
    \includegraphics[width=\linewidth]{dfds/build/diffie-hellman.png}
    \caption{A Diffie-Hellman key exchange illustrated with additional syntax}
    \label{fig-dfd-dh-updated}
\end{figure}

The new syntax illustrates the data in more detail as it is stored, transferred, and operated on. By illustrating the
data itself, it eliminates the need for labels on lines and can be used to communicate entire protocols in a single
% diagram. \myfig{fig-dfd-key} contains the full set of symbols. In \mychap{chap-threatmodel} I create \acp{DFD} in this
style to analyze the threats already present and those that \ac{EA} would introduce.

% \begin{figure}[p]
%     \centering\CaptionFontSize
%     \includegraphics[width=\linewidth]{dfds/build/dfd-key.png}
%     \caption{A Diffie-Hellman key exchange illustrated with additional syntax}
%     \label{fig-dfd-key}
% \end{figure}
