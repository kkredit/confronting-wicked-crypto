\chapter{Strategy for Tackling Wicked Problems}
\label{chap-policy}

Before going into detail about the arguments used in the \ac{EA} debate, it is worth studying \acp{wicked-prob} in
greater detail. What are they, and why is the case of encryption, privacy, and \ac{EA} in this category? What approaches
to tackling \acp{wicked-prob} succeed and fail? How can we collectively situate ourselves to make real progress? This
chapter seeks to answer these questions.


\section{Wicked Problems}

\Acp{wicked-prob} were previously introduced as pernicious and tricky issues that resist straightforward solutions.
In this section we analyze the nature of wicked problems and strategies for tackling them.

\subsection{Characteristics}
\label{wicked-characteristics}

Rittel's categorization of \acp{wicked-prob} grew out of frustration with the problems' resistance to classical problem
solving methods. Since the Enlightenment, society has applied the scientific method to problems of every kind; the
sweeping application of scientific analysis has delivered reliable clean water, improved crop yields, shaped government
structures, and bestowed material wealth previously unimaginable. With our material problems largely solved in the
twentieth century, believers in the power of reason thought this progress would march on in the realm of public
planning. Policymaking would function by setting goals, identifying problems, evaluating alternatives, implementing
solutions, and analyzing outcomes in order to correct errors. Functioning as a continuous process, this approach was
primed to revolutionize governing the same way it did industry, agriculture, and economics---until it didn't. In the
context of what he describes as an ``anti-professional movement,'' Rittel explains how the scientific method has failed:

\begin{displayquote}
A great many barriers keep us from perfecting such a planning/governing system: theory is inadequate for decent
forecasting; our intelligence is insufficient to our tasks; plurality of objectives held by pluralities of politics
makes it impossible to pursue unitary aims; and so on. The difficulties attached to rationality are tenacious, and we
have so far been unable to get untangled from their web. This is partly because the classical paradigm of science and
engineering---the paradigm that has underlain modern professionalism---is not applicable to the problems of open
societal systems. One reason the publics have been attacking the social professions, we believe, is that the cognitive
and occupational styles of the professions---mimicking the cognitive style of science and the occupational style of
engineering---have just not worked on a wide array of social problems. The lay customers are complaining because
planners and other professionals have not succeeded in solving the problems they claimed they could solve. We shall want
to suggest that the social professions were misled somewhere along the line into assuming they could be applied
scientists---that they could solve problems in the ways scientists can solve their sorts of problems. The error has been
a serious one. \cite{rittel_dilemmas_1973}
\end{displayquote}

When applied to social problems, the prescribed method---here defined as setting goals, identifying problems, evaluating
alternatives, implementing solutions, and analyzing outcomes---fails at every step. In the U.S., a nation with countless
cultures and subcultures which are effectively represented by two mutually hostile political parties, agreeing on goals
is a challenge in itself. When goals are set, we often discover that we are contending with \acp{wicked-prob} that defy
the process at each remaining step. Rittel provides a list of ten characteristics of problems in this category
\cite{rittel_dilemmas_1973}:

\begin{displayquote}
  \begin{enumerate}
    \item There is no definitive formulation of a wicked problem [and each formulation presupposes a solution].
    \item Wicked problems have no stopping rule.
    \item Solutions to wicked problems are not true-or-false, but good-or-bad.
    \item There is no immediate and no ultimate test of a solution to a wicked problem.
    \item Every solution to a wicked problem is a ``one-shot operation''; because there is no opportunity to learn by
          trial-and-error, every attempt counts significantly.
    \item Wicked problems do not have an enumerable (or an exhaustively describable) set of potential solutions, nor is
          there a well-described set of permissible operations that may be incorporated into the plan.
    \item Every wicked problem is essentially unique.
    \item Every wicked problem can be considered to be a symptom of another problem.
    \item The existence of a discrepancy representing a wicked problem can be explained in numerous ways. The choice of
          explanation determines the nature of the problem's resolution.
    \item The planner has no right to be wrong [i.e., they are liable for the consequences of their decisions].
  \end{enumerate}
\end{displayquote}

Based on this description, \acp{wicked-prob} are impossible to definitively identify, their alternative solutions are
infinite in number but cannot be proved or independently tested, solution implementations carry large risks, and outcome
analysis is subject to interpretation. The policy version of the scientific method cannot be used under these
conditions.

Several of these characteristics are results of the fact that we have no accurate predictive model of the world and
human behavior at large. This does not need to be expounded upon.

The more important insight is one step removed: precisely because there is an inexhaustable set of potential solutions,
the problem definition---that is, the set of information required to produce a solution---is not self-contained. Each
proposed solution demands research and context to be added to the problem definition. This feedback from proposal to
definition violates the linearity of the traditional approach. One cannot reason from problem statement to proposals to
a solution; instead, one is forced to constantly refine the problem statement based on the content of proposals
themselves.

A 2018 study by the Australian government uses climate change as an example of a \ac{wicked-prob}
\cite{commission_tackling_2018}. Say the problem is initially formulated as long-term change to the environment caused
by the effects of accumulating greenhouse gasses. Responses typically take one of three broad forms: profligate behavior
in consumeristic societies must be reigned in at a local and personal level; global inter-governmental coordination is
the only solution, as individual changes will have no impact; or the situation is overblown by idealists and
power-mongers, and technological progress and adaptive markets will handle any negative effects that come to pass
\cite{commission_tackling_2018}. Which response is correct? One cannot tell from the problem statement. How much do
individual choices contribute to greenhouse gas emissions? How effective are international accords? How costly will the
changes be, and how capable is technology to respond? Evaluating the validity of each proposal requires updates to the
problem statement itself.

Elements of the problem may be tame---one can independently confirm that reducing global carbon emissions by a certain
amount will result in a certain difference in response with a certain statistical confidence. Refining the problem
statement and combining the tame elements in a cohesive solution is the ``wicked'' part. Despite certain tame elements,
one cannot verify the effectiveness of a plan ahead of time, port a solution from a different domain, or even
definitively evaluate its performance after the fact.

\subsection{Encryption and EA as a Wicked Problem}

Dogged by concerns over privacy, security, safety, and trust, \ac{encryption} and the presupposed solution of \ac{EA} is
a \ac{wicked-prob}. It has each of the characteristics from Rittel's list above:

\begin{enumerate}
  \item There is no formulation that encapsulates the problem of encryption's interplay with privacy, security,
        safety, and trust.
  \item Balancing each value in the face of constantly evolving technology is never ending.
  \item \ac{EA} or other proposals cannot definitively solve the problem.
  \item \ac{EA} or other proposals cannot be objectively tested.
  \item Every policy implementation has irreversible effects.
  \item There is an inexhaustable set of potential solutions to the problem.
  \item Solutions from other domains do not apply directly to this problem.
  \item The need for \ac{EA} or some alternative is a symptom of differing values, rapid technological change, and
        criminal behavior.
  \item The problem can be framed as insufficient investigatory access to digital data (presupposing \ac{EA} as the
        solution), outdated cyberlaw (presupposing legal hacking or compelled password disclosure as solutions), and
        more.
  \item The decisions of regulators and technologists alike have real impacts in the world today.
\end{enumerate}

Cybersecurity law scholar Alan Rozenshtein argues for treating \ac{encryption} and \ac{EA} as a \ac{wicked-prob} at
length \cite{rozenshtein_wicked_2018}. He breaks the root issues down into three categories. First, that there is
disagreement on goals (or even basic premises). Sides can't agree whether things are ``going dark'' or we're in the
``golden age of surveillance'' or how much to value competing notions of security. Second, that information is
``uncertain and diffuse.'' There are insufficient statistics on encryption's effect on investigations, and though the
consensus is that we are not presently capable of acceptably secure \ac{EA}, that is not certain to always be the case,
especially as the area is under-researched. Third, that the problem cannot be definitively solved. Evolving values and
technology mean that the subject is always up for renegotiation.

Rozenshtein reflects on the diagnosis optimistically:

\begin{displayquote}
Recognizing that something is a wicked problem is not an admission of its insolubility; rather, it’s just a realistic
appreciation of its challenges. Progress on difficult social problems reflects, almost by definition, progress on wicked
problems, whether economic inequality, environmental degradation, or government access to data. Progress can be made,
but it first requires a clear-eyed appreciation of the nature of the problem and the nature of its challenges.
\cite{rozenshtein_wicked_2018}
\end{displayquote}

Embracing reality is the first step to dealing with it \cite{baker_2019}. We have by now embraced the reality of
\acp{wicked-prob}. In the next two sections, we look at strategies for dealing with it.


\section{Failure of Current Policymaking Approaches}

This section describes two common policymaking approaches, the \ac{classical-method} and \ac{incrementalism}.

\subsection{The Classical Analytic Method}

The \ac{classical-method} \cite{feeley_judicial_2000} (or ``the modern-classical model of planning''
\cite{rittel_dilemmas_1973}, ``the rational-comprehensive method'' \cite{lindblom_muddling_1959}, ``traditional policy
analysis'' \cite{rozenshtein_wicked_2018}, or ``linear thinking'' \cite{commission_tackling_2018}) has already been
introduced. It is the reason-based method that functions by setting goals, identifying problems, evaluating
alternatives, implementing solutions, and analyzing outcomes in order to correct errors. \myfig{fig-classical-method}
illustrates the approach in the context of encryption and \ac{EA}. It has a purely linear flow from problem to solution
except for the ``refinement'' step, in which the method analyzes outcomes and corrects errors. However, it is important
to note that refinement is always in-kind---it represents a reinforcement, as opposed to a reassessment, of the chosen
solution.

\begin{figure}[h]
  \centering\CaptionFontSize
  \includegraphics[width=\linewidth]{dfds/build/OODA-classical.png}
  \caption[The Classical Analytic Method]{The Classical Analytic Method}
  \label{fig-classical-method}
\end{figure}

The shortcomings of the \ac{classical-method} were discussed in \mysec{wicked-characteristics}. It fails due to
disagreement over goals, the dependence of the problem definition on the alternatives generated (violating the linear
flow), the inability to evaluate alternatives, and the lack of a definitive stopping rule.

\subsection{Incrementalism}

\Ac{incrementalism} is an intuitive and iterative approach posed as an alternative to the \ac{classical-method}.
\Ac{incrementalism} as a policymaking strategy is often referred to as ``\ac{muddling-through}'' after political
scientist Charles Lindblom's classic paper defining and defending the approach \cite{lindblom_muddling_1959}. Written
before Rittel developed the idea of ``\acp{wicked-prob},'' Lindblom nonetheless identified many of the same shortcomings
of the classical method and sought to formalize the process policymakers were already often using.

Lindblom's ``\ac{muddling-through}'' operates by taking successive steps chosen through comparative analysis. The
alternatives selected for comparison are defined relative to the status quo and are close enough to one another to be
able to be analyzed on the margin. This is done first out of practical necessity, due to our inability to rationally
predict policy outcomes, and second out of political realism, as non-incremental changes are usually politically
impossible to impose in a democratic system \cite{lindblom_muddling_1959}. The formality of the process varies;
policymakers may use this method consciously, with considerable comparative analysis, or unconsciously, led by
intuition. \myfig{fig-muddling-through} illustrates this method.

%Footnote: see Lindblom's bracing quote: ``Party behavior is in turn rooted in public attitudes, and political theorists
%cannot conceive of democracy's surviving in the United States in the absence of fundamental agreement on potentially
%disruptive issues, with consequent limitation of policy debates to relatively small differences in policy.''

\begin{figure}[h]
  \centering\CaptionFontSize
  \includegraphics[width=0.55\linewidth]{dfds/build/OODA-muddling.png}
  \caption[The Incrementalist Method]{The Incrementalist Method}
  \label{fig-muddling-through}
\end{figure}

\Ac{incrementalism} has several advantages over the \ac{classical-method} for handling \acp{wicked-prob}. It is rooted
in realism about the limits of rational analysis. It accepts that the problem will not be conclusively solved,
emphasizing iteration (``Policy is not made once and for all; it is made and re-made endlessly. Policy-making is a
process of successive approximation to some desired objectives in which what is desired itself continues to change under
reconsideration.'' \cite{lindblom_muddling_1959}). Most importantly, it eschews the linear
problem-definition--alternatives-analysis steps out of respect that this is, indeed, impossible. Lindblom argues at
length that policy ends and means are interlinked, eventually concluding:

\begin{displayquote}
As to whether the attempt to clarify objectives in advance of policy selection is more or less rational than the close
intertwining of marginal evaluation and empirical analysis, the principal difference established is that for complex
[i.e., wicked] problems the first is impossible and irrelevant, and the second is both possible and relevant. The second
is possible because the administrator need not try to analyze any values except the values by which alternative policies
differ and need not be concerned with them except as they differ marginally. His need for information on values or
objectives is drastically reduced as compared with the root [i.e., classical analytic] method; and his capacity for
grasping, comprehending, and relating values to one another is not strained beyond the breaking point.
\cite{lindblom_muddling_1959}
\end{displayquote}

Despite its strengths, \ac{incrementalism} has its weaknesses at addressing \acp{wicked-prob}. Its main weakness is its
lack of high level strategic analysis. It is incapable of drastic change, which is sometimes necessary. By Lindblom's
admission, it lacks a safeguard for consideration of all relevant values and may ``overlook excellent policies for no
other reason than that they are not suggested by the chain of successive policy steps leading up to the present''
\cite{lindblom_muddling_1959}. Analysis based only on one present ``status quo'' to the next can result in messy policy
that none are particularly fond of---``As Lindblom's sobriquet suggests, it often [leads] to a considerable muddle''
\cite{feeley_judicial_2000}. Consider application of the \acl{CFAA} as one such muddle \cite{wolff_computer_2016}.

Unfortunately, one cannot simply expand the strategic view of the incrementalist method by allowing it to look further
into the past than the present status quo. Simplifying analysis by limiting it to marginal differences to a given
baseline is at the heart of \ac{incrementalism}. If one tries to be a ``strategic incrementalist'' by looking into the
past, they still have to choose a baseline from which to perform analysis. This approach is susceptible to two
weaknesses in baseline-based reasoning that Rozenshtein describes in his analysis of \ac{EA} as a \ac{wicked-prob}
\cite{rozenshtein_wicked_2018}. First, the choice of baseline is arbitrary, yet heavily colors analysis:

\begin{displayquote}
Both the government and its critics have operated from the status-quo baseline, though from opposite directions. For the
government, the relevant baseline is recent history---specifically, right before companies like Apple and WhatsApp
encrypted their products. From this baseline, the government's ability to surveil has diminished. For critics of
government surveillance, the relevant baseline is the pre-digital age, before smartphones and social media vastly
expanded the government's surveillance capabilities. From this baseline, the technological changes underlying the
``going dark'' problem are mere blips on the otherwise rocketing growth of the surveillance state.
\cite{rozenshtein_wicked_2018}
\end{displayquote}

Second, unlike legal baselines, policy baselines do not carry normative force. Constant changes in the underlying
situation mean that even optimal policy in the past is not necessarily desirable in the present. Due to these
weaknesses, applying incrementalist methods at the strategic level does not work.

A final weakness in \ac{incrementalism} is its assumption of basic agreement and political stability. The method works
by limiting analysis to marginal comparisons of broadly similar and familiar proposals. Proposals that differ widely
from one another or the status quo are considered irrelevant because the debating parties both share the same general
goals and lack the ability to unilaterally impose their will. Rather a symptom of present circumstances (see
\mysec{sec-history-current}) than an inherent weakness in the incrementalist approach, both of these assumptions are
incorrect.


\subsection{Lessons}

The \ac{classical-method} and \ac{incrementalism} are not the only styles of policymaking. They represent perhaps two
extremes on a spectrum of rational planning and intuitive muddling. Both have their strengths and weaknesses. One may
intuit that a reasonable strategy is the selective use of both approaches according to the situation, and indeed this
has also been formally suggested \cite{etzioni_scanning_1967}. However, even a combination of these methods does not
precisely suit all classes of wicked problems. Is a problem tame? Use the \ac{classical-method}. Is it wicked, but
strategically under control and relatively non-controversial? Use \ac{incrementalism}. Is it wicked, and lacking a
strategic response or highly controversial? What then?

Let us define one strategy for handling \ac{wicked-prob} as corralling it back into the classes of problems we are
better at dealing with. That means breaking it into tame sub-problems and developing agreeable strategies for its
irreducibly wicked elements. Once that is done, we can rationally analyze and increment our way to victory. There is no
handbook for how to do this, but several sources offer advice.

\newcommand{\wickedtipsstart}{\begin{itemize}}
\newcommand{\wickedtip}[2]{ % Name, citation, description
  \item \textbf{#1} \nopagebreak

  \vspace{0.5\baselineskip} \parbox{\linewidth}{#2} \vspace{0.5\baselineskip}
}
\newcommand{\wickedtipsend}{\end{itemize}}

\wickedtipsstart

  \wickedtip{Reject Easy Answers}{

Easy answers, or any solutions that artificially tame the problem, will not bring the matter under control. While by
definition there is no step that will truly solve the problem, easy answers deliberately emphasize one value to the
exclusion of others, leaving chaos in those neglected areas. As a result of ignoring root issues feeding the problem,
such solutions are even more likely to produce unintended consequences \cite{commission_tackling_2018}.

}

  \wickedtip{Bring Everyone to the Table}{

Making sure every relevant group is included is important for two reasons. First, because information is ``uncertain and
diffuse,'' generating an accurate problem statement requires diverse input \cite{rozenshtein_wicked_2018}. Second, for
\ac{incrementalism} to work on the irreducibly wicked roots of the problem, there needs to be some degree of consensus
on overall strategy. Consensus building is not easy among groups with differing values and priorities, but it is
impossible without each group being represented.

}

  \wickedtip{Unite Problem Definition and Analysis Steps}{

The failures of the \ac{classical-method} and \ac{incrementalism} teach that we must be able to think strategically
while respecting the non-linear nature of \acp{wicked-prob}. One way to use the high level, holistic view of the
classical approach while intertwining the problem definition and analysis steps as in the incrementalist approach. In
practice, this means that the collective understanding of the problem and potential solutions must co-evolve. As Rittel
puts it, ``The systems-approach `of the first generation' [i.e., classical analytics] is inadequate for dealing with
wicked-problems. Approaches of the `second generation' should be based on a model of planning as an argumentative
process in the course of which an image of the problem and of the solution emerges gradually among the participants, as
a product of incessant judgment, subjected to critical argument'' \cite{rittel_dilemmas_1973}.

}

\wickedtip{Embrace Flexible, Risk-Based Solutions}{

We know that proposals cannot be comprehensively evaluated before or even after implementation, and that each action (or
decision not to act) has irreversible effects. \Ac{incrementalism} is the decision making model to follow here, as it is
inherently responsive to unpredictable changes in the environment. Proposals must therefore be agile. All decisions
involve unknowns, but risk- and uncertainty-management strategies can optimize the expected outcome and maximize the
worst outcome \cite{sunstein_beyond_2015}.

}

  \wickedtip{Focus Discussion around Concrete Proposals}{

This follows partially from the previous suggestion---problem definition and analysis are combined precisely because the
problem definition depends on the nature of proposed solutions. But this advice deserves emphasis for another reason:
consensus is easier to build around concrete proposals. While debate rages in the abstract, groups are bound to disagree
due to their conflicting values and priorities, but Linblom writes encouragingly about ``the ease with which individuals
of different ideologies often can agree on concrete policy'' in an example about congressional compromise; he goes on to
say, ``Labor mediators report a similar phenomenon: the contestants cannot agree on criteria for settling their disputes
but can agree on specific proposals. Similarly, when one administrator's objective turns out to be another's means, they
often can agree on policy'' \cite{lindblom_muddling_1959}.

% So should I follow my own advice and focus on
%   group_2019 (argues for DAR) and
%   savage_lawful_2018 (a concrete DAR proposal)? -- and related, servan_schreiber_jje_2020

}

\wickedtipsend


\section{Proposal: The OODA Loop for Policymaking}

\ac{OODAloop}.

\begin{figure}[h]
    \centering\CaptionFontSize
    \includegraphics[width=\linewidth]{dfds/build/OODA-loop.png}
    \caption[The OODA Loop]{The OODA Loop}
    \label{fig-ooda-loop}
\end{figure}
