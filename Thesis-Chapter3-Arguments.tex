\chapter{Arguments}
\label{chap-arguments}

% In this chapter, develop the arguments employed in EA debates. Make use of argument maps to break down and visualize the
% debate, as shown in \myfig{fig-args-top-level}.

% \begin{figure}[h]
%     \centering\CaptionFontSize
%     \includegraphics[width=\linewidth]{arguments/build/top-level.pdf}
%     \caption[Argument: Top Level]{Top level arguments for and against exceptional access.}
%     \label{fig-args-top-level}
% \end{figure}

The debate over exceptional access takes place on two fundamental levels:
\begin{enumerate}
    \item Whether cryptosystems with exceptional access represent a net benefit to society
    \item Whether and how to conduct exceptional access research
\end{enumerate}

\subsection{Arguments For}

Why EA is a benefit to society:
\begin{itemize}
    \item Absolute privacy is not a right
    \item Encryption gives too much power to the individual
    \item EA would allow more crimes to be prevented and prosecuted
\end{itemize}

Why to research EA:
\begin{itemize}
    \item EA done well would be more secure and privacy preserving than EA done poorly
    \item The threat of legislative action creates urgency
\end{itemize}

\subsection{Arguments Against}

Why EA is a danger to society:
\begin{itemize}
    \item Privacy is an absolute right
    \item Exceptional access gives too much power to the state
    \item Exceptional access is not possible in an acceptably secure manner
\end{itemize}

Why not to research EA:
\begin{itemize}
    \item With both security and privacy in a poor state, research should be focused on strengthening, not weakening,
            them
    \item The technical community should not entertain delusions about what is truly possible
\end{itemize}

\subsection{Fallacious Arguments}

Fallacy definitions come from the helpful University of North Carolina Writing Center \cite{unc_2020}.
