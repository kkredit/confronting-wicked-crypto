\chapter{Arguments}
\label{chap-arguments}

% How to tackle wicked problems in general? Take some tips from commission_tackling_2018, rozenshtein_wicked_2018,
% feeley_judicial_2000, and matyas_incommensurability_2018:
% Correctly state the problem--identify the root causes. They aren't going away, there by definition is no clear long
% term solution. Put the proposed solutions in context--none are perfect, none are permanent. For EA in particular,
% consider balance of principles and rules.

% Subject one: do we even want EA?
%  --> no? exit here
%  --> yes? carry on
%  --> pragmatist? carry on
% include fallacies argument map here
% see rozenshtein_wicked_2018, benaloh_what_2018, soesanto_2018, ministerial_2018, fbi_2020, brantley_2018, lund_2020
% ... group_2019, ruiz_there_2018, rozenshtein_2019

% Subject two: what properties would good EA have?
% easy for LEAs to use? hard? expensive? voluntary? unable to be used for mass surveillance? transparent (attempts are
%   visible)? alignment of values with security (e.g. discourages bug hoarding)? adaptable (since solution won't be
%   permanent, either technically or according to collective will as things change rozenshtein_wicked_2018)? suited to
%   rules and principles (matyas_incommensurability_2018)?

% Subject three: what options do we have?
% map of properties onto proposals--leads into threat modeling, which will check the veracity of claims

%%%%%%%%%%%%
% Chapter 4, threat models
% (how to be specific enough to be useful but generic enough to apply to many systems? Focus on the EA type (DAR v DIM),
%   then apply to specific systems)
% DAR, DIM threat models (make sure to cover abelson_2015)
% Basic proposals
%  --> mappings onto threat models
% At work: 3 example systems analyzed with this threat model
% 1. IM (E2E IM too?) -- in motion
% 2. Cloud storage -- at rest
% 3. Mobile phone -- at rest, but way different scenario than cloud
%  --> include Ozzie's CLEAR, since it was higher profile
%%%%%%%%%%%%

% Leave for conclusion chapter:
% Subject four: now what?
%  --> How current proposals stack up against desired properties, threat model
%  --> security expert? do research, provide clear security advice, be vigilant
%  --> policymaker? fund research, don't mislead, don't be premature
%  --> either? know that there is no stable solution, balance rules and principles, serve your neighbor
% see varia_2018























%%%%%%%%%%%%%%%%%%%%%%%%%%%%%%%%%%%%%%%%%%%%%%%%%%%%%%%%%%%%%%%%%%%%%%%%%%%%%%%%%%%%%%%%%%%%%%%%

% In this chapter, develop the arguments employed in EA debates. Make use of argument maps to break down and visualize the
% debate, as shown in \myfig{fig-args-top-level}.

% \begin{figure}[h]
%     \centering\CaptionFontSize
%     \includegraphics[width=\linewidth]{arguments/build/top-level.pdf}
%     \caption[Argument: Top Level]{Top level arguments for and against exceptional access.}
%     \label{fig-args-top-level}
% \end{figure}

The debate over exceptional access takes place on two fundamental levels:
\begin{enumerate}
    \item Whether cryptosystems with exceptional access represent a net benefit to society
    \item Whether and how to conduct exceptional access research
\end{enumerate}

\subsection{Arguments For}

Why EA is a benefit to society:
\begin{itemize}
    \item Absolute privacy is not a right
    \item Encryption gives too much power to the individual
    \item EA would allow more crimes to be prevented and prosecuted
\end{itemize}

Why to research EA:
\begin{itemize}
    \item EA done well would be more secure and privacy preserving than EA done poorly
    \item The threat of legislative action creates urgency
\end{itemize}

\subsection{Arguments Against}

Why EA is a danger to society:
\begin{itemize}
    \item Privacy is an absolute right
    \item Exceptional access gives too much power to the state
    \item Exceptional access is not possible in an acceptably secure manner
\end{itemize}

Why not to research EA:
\begin{itemize}
    \item With both security and privacy in a poor state, research should be focused on strengthening, not weakening,
            them
    \item The technical community should not entertain delusions about what is truly possible
\end{itemize}

\subsection{Fallacious Arguments}

Fallacy definitions come from the helpful University of North Carolina Writing Center \cite{unc_2020}. Also cited:
\cite{hanna_2019}.

Fallacies are as shown in \myfig{fig-args-fallacies}.

\begin{figure}[h]
    \centering\CaptionFontSize
    \myincludeargument{comprehensive.fallacies}
    \caption[Fallacious Arguments]{Fallacious arguments used in the exceptional access debate.}
    \label{fig-args-fallacies}
\end{figure}
