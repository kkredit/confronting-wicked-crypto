\chapter{Arguments}
\label{chap-arguments}

\section{Wicked Problems}

\subsection{Encryption and Exceptional Access as a Wicked Problem}

- rozenshtein_wicked_2018 convincingly argues the point at length

\subsection{Approaches to Solving Wicked Problems}

\subsubsection{Failure of the Classical Analytic Method}

\subsubsection{Muddling Through}

% Mention hermeneutics and complex adaptive systems?

\subsubsection{Pitfalls to Avoid}

- Easy answers
- Fallacies!

\subsection{Plan for this Chapter}

% - But we're _exactly_ following the classical analytic method! Well, yes, but in a cooperative manner, knowing it will
%     require iteration, and being careful about side effects. It's an agile approach to policy making, but it's still
%     policy making.
% Already covered defining the problem. Next will discuss goals and list alternatives. That will feed into threat
% modeling.

\section{Desireable Outcomes}

\section{Proposed Alternatives}


commission_tackling_2018:
- Strategies
  - Authoritative
  - Competitive
  - Collaborative
- Approaches
  - Linear problem solving
    - Ineffective! Several reasons, good breakdown. Dovetails with rozenshtein_wicked_2018.
  - Holistic problem solving: adaptive, innovative, flexible, cross-organizational
- Avoid
  - Narrow solutions

rozenshtein_wicked_2018:
- Approaches
  - Muddling through (he cites Charles E. Lindblom, just like feeley_judicial_2000!)
  - Risk management (--finally, something familiar to security experts!)
    - So, a broader view of security, inclusive of both public safety and information security
    - In this model, threat actors now include terrorists, CSAM distributors, organized crime
      - Add a new character? Evil Alice and evil Bob (Alex and Becca?)? This automatically raises to question of how to
          identify bad actors--AI (lee_detecting_2020)? PhotoDNA (microsoft_2020)? Government orders (warrants)? Could
          be useful.
    - I have already suggested adding government mandates to the traditional IS threat model
  - Imperfect solutions, like lawful hacking
  - Meanwhile, invest in knowledge production and improved relationships (like "cross-organizational" from
      commission_tackling_2018)
- Avoid
  - Easy answers ("narrow solutions" from commission_tackling_2018)
    - Such as EA mandates or "golden age of surveillance"-based denials

matyas_incommensurability_2018:
- Approaches to reducing incommensurability
  - Reduce gray area between rules and principles (i.e. keep laws up to date with technology)
  - Use transparency and accountability (through technical means)

Note: a problem with muddling through wicked problems is that each attempted fix leads to long lasting unintended side
effects. This is especially true with matters of encryption, where the data could be stored for years till it can be
decrypted! So must be very careful to avoid solution attempts that leave long scars.

feeley_judicial_2000:
- Classical analytic method ("linear" in commission_tackling_2018)
- Incremental intuitive decision making ("Muddling through" in rozenshtein_wicked_2018)
- The hermeneutic circle

% How to tackle wicked problems in general? Take some tips from commission_tackling_2018, rozenshtein_wicked_2018,
% feeley_judicial_2000, and matyas_incommensurability_2018:
% Correctly state the problem--identify the root causes. They aren't going away, there by definition is no clear long
% term solution. Put the proposed solutions in context--none are perfect, none are permanent. For EA in particular,
% consider balance of principles and rules.

% Subject one: do we even want EA?
%  --> no? exit here
%  --> yes? carry on
%  --> pragmatist? carry on
% include fallacies argument map here
% see rozenshtein_wicked_2018, benaloh_what_2018, soesanto_2018, ministerial_2018, fbi_2020, brantley_2018, lund_2020
% ... group_2019, ruiz_there_2018, rozenshtein_2019

% Subject two: what properties would good EA have?
% easy for LEAs to use? hard? expensive? voluntary? unable to be used for mass surveillance? transparent (attempts are
%   visible)? alignment of values with security (e.g. discourages bug hoarding)? adaptable (since solution won't be
%   permanent, either technically or according to collective will as things change rozenshtein_wicked_2018)? suited to
%   rules and principles (matyas_incommensurability_2018)?

% Subject three: what options do we have?
% map of properties onto proposals--leads into threat modeling, which will check the veracity of claims














%%%%%%%%%%%%%%%%%%%%%%%%%%%%%%%%%%%%%%%%%%%%%%%%%%%%%%%%%%%%%%%%%%%%%%%%%%%%%%%%%%%%%%%%%%%%%%%%

% In this chapter, develop the arguments employed in EA debates. Make use of argument maps to break down and visualize the
% debate, as shown in \myfig{fig-args-top-level}.

% \begin{figure}[h]
%     \centering\CaptionFontSize
%     \includegraphics[width=\linewidth]{arguments/build/top-level.pdf}
%     \caption[Argument: Top Level]{Top level arguments for and against exceptional access.}
%     \label{fig-args-top-level}
% \end{figure}

% The debate over exceptional access takes place on two fundamental levels:
% \begin{enumerate}
%     \item Whether cryptosystems with exceptional access represent a net benefit to society
%     \item Whether and how to conduct exceptional access research
% \end{enumerate}

% \subsection{Arguments For}

% Why EA is a benefit to society:
% \begin{itemize}
%     \item Absolute privacy is not a right
%     \item Encryption gives too much power to the individual
%     \item EA would allow more crimes to be prevented and prosecuted
% \end{itemize}

% Why to research EA:
% \begin{itemize}
%     \item EA done well would be more secure and privacy preserving than EA done poorly
%     \item The threat of legislative action creates urgency
% \end{itemize}

% \subsection{Arguments Against}

% Why EA is a danger to society:
% \begin{itemize}
%     \item Privacy is an absolute right
%     \item Exceptional access gives too much power to the state
%     \item Exceptional access is not possible in an acceptably secure manner
% \end{itemize}

% Why not to research EA:
% \begin{itemize}
%     \item With both security and privacy in a poor state, research should be focused on strengthening, not weakening,
%             them
%     \item The technical community should not entertain delusions about what is truly possible
% \end{itemize}

% \subsection{Fallacious Arguments}

% Fallacy definitions come from the helpful University of North Carolina Writing Center \cite{unc_2020}. Also cited:
% \cite{hanna_2019}.

% Fallacies are as shown in \myfig{fig-args-fallacies}.

% \begin{figure}[h]
%     \centering\CaptionFontSize
%     \myincludeargument{comprehensive.fallacies}
%     \caption[Fallacious Arguments]{Fallacious arguments used in the exceptional access debate.}
%     \label{fig-args-fallacies}
% \end{figure}
