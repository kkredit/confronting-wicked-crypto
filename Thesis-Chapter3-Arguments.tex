\chapter{Wicked Problems and the EA Debate}
\label{chap-arguments}

Before going into detail about the arguments used in the \ac{EA} debate, it is worth studying \acp{wicked-prob} in
greater detail. What are they, and why is the case of encryption, privacy, and \ac{EA} in this category? What approaches
to tackling \acp{wicked-prob} succeed and fail? How can we collectively situate ourselves to make real progress? This
chapter seeks to answer these questions.


\section{Wicked Problems}

\Acp{wicked-prob} were previously introduced as pernicious and tricky issues that resist straightforward solutions.
In this section we analyze the nature of wicked problems and strategies for tackling them.

\subsection{Characteristics}
\label{wicked-characteristics}

Rittel's categorization of \acp{wicked-prob} grew out of frustration with their resistance to classical problem solving
methods. Since the Enlightenment, society has applied the scientific method to problems of every kind; the sweeping
application of scientific analysis has delivered reliable clean water, improved crop yields, shaped government
structures, and bestowed material wealth previously unimaginable. With our material problems largely solved in the
twentieth century, believers in the power of reason thought this progress would march on in the realm of public
planning. Policymaking would function by setting goals, identifying problems, evaluating alternatives, implementing
solutions, and analyzing outcomes in order to correct errors. Functioning as a continuous process, this approach was
primed to revolutionize governing the same way it did industry, agriculture, and economics---until it didn't. In the
context of what he describes as an ``anti-professional movement,'' Rittel explains how the scientific method has failed:

\begin{displayquote}
A great many barriers keep us from perfecting such a planning/governing system: theory is inadequate for decent
forecasting; our intelligence is insufficient to our tasks; plurality of objectives held by pluralities of politics
makes it impossible to pursue unitary aims; and so on. The difficulties attached to rationality are tenacious, and we
have so far been unable to get untangled from their web. This is partly because the classical paradigm of science and
engineering---the paradigm that has underlain modern professionalism---is not applicable to the problems of open
societal systems. One reason the publics have been attacking the social professions, we believe, is that the cognitive
and occupational styles of the professions---mimicking the cognitive style of science and the occupational style of
engineering---have just not worked on a wide array of social problems. The lay customers are complaining because
planners and other professionals have not succeeded in solving the problems they claimed they could solve. We shall want
to suggest that the social professions were misled somewhere along the line into assuming they could be applied
scientists---that they could solve problems in the ways scientists can solve their sorts of problems. The error has been
a serious one. \cite{rittel_dilemmas_1973}
\end{displayquote}

When applied to social problems, the prescribed method---here defined as setting goals, identifying problems, evaluating
alternatives, implementing solutions, and analyzing outcomes---fails at every step. In the U.S., a nation with countless
cultures and subcultures which are effectively represented by two mutually hostile political parties, agreeing on goals
is a challenge in itself. When goals are set, we often discover that we are contending with \acp{wicked-prob} that defy
the process at each remaining step. Rittel provides a list of ten characteristics of problems in this category
\cite{rittel_dilemmas_1973}:

\begin{displayquote}
  \begin{enumerate}
    \item There is no definitive formulation of a wicked problem [and each formulation presupposes a solution].
    \item Wicked problems have no stopping rule.
    \item Solutions to wicked problems are not true-or-false, but good-or-bad.
    \item There is no immediate and no ultimate test of a solution to a wicked problem.
    \item Every solution to a wicked problem is a ``one-shot operation''; because there is no opportunity to learn by
          trial-and-error, every attempt counts significantly.
    \item Wicked problems do not have an enumerable (or an exhaustively describable) set of potential solutions, nor is
          there a well-described set of permissible operations that may be incorporated into the plan.
    \item Every wicked problem is essentially unique.
    \item Every wicked problem can be considered to be a symptom of another problem.
    \item The existence of a discrepancy representing a wicked problem can be explained in numerous ways. The choice of
          explanation determines the nature of the problem's resolution.
    \item The planner has no right to be wrong [i.e., they are liable for the consequences of their decisions].
  \end{enumerate}
\end{displayquote}

Based on this description, \acp{wicked-prob} are impossible to definitively identify, their alternative solutions are
infinite in number but cannot be proved or independently tested, solution implementations carry large risks, and outcome
analysis is subject to interpretation. The policy version of the scientific method cannot be used under these
conditions.

Several of these characteristics are results of the fact that we have no accurate predictive model of the world and
human behavior at large. This does not need to be expounded upon.

The more important insight is one step removed: precisely because there is an inexhaustable set of potential solutions,
the problem definition---that is, the set of information required to produce a solution---is not self-contained. Each
proposed solution demands research and context to be added to the problem definition. This feedback from proposal to
definition violates the linearity of the traditional approach. One cannot reason from problem statement to proposals to
a solution; instead, one is forced to constantly refine the problem statement based on the content of proposals
themselves.

A 2018 study by the Australian government uses climate change as an example of a \ac{wicked-prob}
\cite{commission_tackling_2018}. Say the problem is initially formulated as long-term change to the environment caused
by the effects of accumulating greenhouse gasses. Responses typically take one of three broad forms: profligate behavior
in consumeristic societies must be reigned in at a local and personal level; global inter-governmental coordination is
the only solution, as individual changes will have no impact; or the situation is overblown by idealists and
power-mongers, and technological progress and adaptive markets will handle any negative effects that come to pass
\cite{commission_tackling_2018}. Which response is correct? One cannot tell from the problem statement. How much do
individual choices contribute to greenhouse gas emissions? How effective are international accords? How costly will the
changes be, and how capable is technology to respond? Evaluating the validity of each proposal requires updates to the
problem statement itself.

Elements of the problem may be tame---one can independently confirm that reducing global carbon emissions by a certain
amount will result in a certain difference in response with a certain confidence level. Combining the tame elements in a
cohesive solution is the ``wicked'' part. Despite certain tame elements, one cannot verify the effectiveness of a plan
ahead of time, port a solution from a different domain, or even definitively evaluate its performance after the fact.

\subsection{Encryption and EA as a Wicked Problem}

Dogged by concerns over privacy, security, safety, and trust, \ac{encryption} and the presupposed solution of \ac{EA} is
a \ac{wicked-prob}. It has each of the characteristics from Rittel's list above:

\begin{enumerate}
  \item There is no formulation that encapsulates the problem of encryption's interplay with privacy, security,
        safety, and trust.
  \item Balancing each value in the face of constantly evolving technology is never ending.
  \item \ac{EA} or other proposals cannot definitively solve the problem.
  \item \ac{EA} or other proposals cannot be objectively tested.
  \item Every policy implementation has irreversible impacts.
  \item There is an inexhaustable set of potential solutions to the problem.
  \item Solutions from other domains do not apply directly to this problem.
  \item The need for \ac{EA} or some alternative is a symptom of differing values, rapid technological change, and
        criminal behavior.
  \item The problem can be framed as insufficient investigatory access to digital data (presupposing \ac{EA} as the
        solution), outdated cyberlaw (presupposing legal hacking or compelled password disclosure as solutions), and
        more.
  \item The decisions of regulators and technologists alike have real impacts in the world today.
\end{enumerate}

Cybersecurity law scholar Alan Rozenshtein argues this point at length \cite{rozenshtein_wicked_2018}. He breaks the
root issues down into three categories. First, that there is disagreement on goals (or even basic premises). Sides can't
agree whether things are ``going dark'' or we're in the ``golden age of surveillance'' or how much to value competing
notions of security. Second, that information is ``uncertain and diffuse.'' There are insufficient statistics on
encryption's effect on investigations, and though the consensus is that we are not presently capable of acceptably
secure \ac{EA}, that is not certain to always be the case, especially as the area is under-researched. Third, that the
problem cannot be definitively solved. Evolving values and technology mean that the subject is always up for
renegotiation.

Rozenshtein reflects on the diagnosis optimistically:

\begin{displayquote}
Recognizing that something is a wicked problem is not an admission of its insolubility; rather, it’s just a realistic
appreciation of its challenges. Progress on difficult social problems reflects, almost by definition, progress on wicked
problems, whether economic inequality, environmental degradation, or government access to data. Progress can be made,
but it first requires a clear-eyed appreciation of the nature of the problem and the nature of its challenges.
\cite{rozenshtein_wicked_2018}
\end{displayquote}

Embracing reality is the first step to dealing with it \cite{baker_2019}. We have by now embraced the reality of
\acp{wicked-prob}. In the next two sections, we look at strategies for dealing with it.


\section{Failure of Current Policymaking Approaches}

This section describes the two common current policymaking approaches, the classical analytical method and ``muddling
through.''

\subsection{The Classical Analytical Method}

The classical analytical method \cite{feeley_judicial_2000} (or ``the modern-classical model of planning''
\cite{rittel_dilemmas_1973}, ``traditional policy analysis'' \cite{rozenshtein_wicked_2018}, or ``linear thinking''
\cite{commission_tackling_2018}) has already been introduced. It is the reason-based method that functions by setting
goals, identifying problems, evaluating alternatives, implementing solutions, and analyzing outcomes in order to correct
errors. \myfig{fig-classical-method} illustrates the approach in the context of encryption and \ac{EA}. Per its name, it
has a linear flow from problem to solution. The refinement step reflects the fact that this method tracks outcomes and
tweaks solutions over time; however, refinement is always in-kind---it represents a reinforcement, as opposed to a
reassessment, of the chosen solution.

\begin{figure}[h]
  \centering\CaptionFontSize
  \includegraphics[width=\linewidth]{dfds/build/OODA-classical.png}
  \caption[The Classical Analytical Method]{The Classical Analytical Method}
  \label{fig-classical-method}
\end{figure}

The shortcomings of the classical analytical method were discussed in \mysec{wicked-characteristics}. It fails due to
disagreement over goals, the dependence of the problem definition on the alternatives generated (violating the linear
flow), the inability to evaluate alternatives, and the lack of a definitive stopping rule.

\subsection{Muddling Through}

``Muddling through'' represents an alternative policy making strategy.

See \myfig{fig-muddling-through}.

\begin{figure}[h]
  \centering\CaptionFontSize
  \includegraphics[width=0.55\linewidth]{dfds/build/OODA-muddling.png}
  \caption[The ``Muddling Through'' Method]{The ``Muddling Through'' Method}
  \label{fig-muddling-through}
\end{figure}

% Mention hermeneutics and complex adaptive systems?
% As feeley_judicial_2000 points out, muddling through is too focused on, or even consists entirely of, the
% implementation step. Because of the long tail effects of any implementation in a wicked scenario, and especially in
% the context of encryption, mere muddling through is too risky. Therefore I suggest iteration. Classical methods, but
% applied in a cyclical manner.

\subsection{Lessons}

- Avoid pitfalls of
  - Easy answers
    - EA mandates, complexity denial
  - Fallacies!
  - Bad results with long technical tails

% Note: a problem with muddling through wicked problems is that each attempted fix leads to long lasting unintended side
% effects. This is especially true with matters of encryption, where the data could be stored for years till it can be
% decrypted! So must be very careful to avoid solution attempts that leave long scars.

\section{Proposal: The OODA Loop for Policymaking}

\ac{OODAloop}.

\begin{figure}[h]
    \centering\CaptionFontSize
    \includegraphics[width=\linewidth]{dfds/build/OODA-loop.png}
    \caption[The OODA Loop]{The OODA Loop}
    \label{fig-ooda-loop}
\end{figure}

% Already covered defining the problem. Next will discuss goals and list alternatives. That will feed into threat
% modeling

\section{Mapping the EA Debate}

\subsection{Desireable Outcomes}

% See committee_decrypting_2018!
% For a wholistic threat model, include both the threats of government (legislation) and of criminal abusers of your
% system.

\subsection{Proposed Alternatives}

% See committee_decrypting_2018!


commission_tackling_2018:
- Strategies
  - Authoritative
  - Competitive
  - Collaborative
- Approaches
  - Linear problem solving
    - Ineffective! Several reasons, good breakdown. Dovetails with rozenshtein_wicked_2018.
  - Holistic problem solving: adaptive, innovative, flexible, cross-organizational
- Avoid
  - Narrow solutions

rozenshtein_wicked_2018:
- Approaches
  - Muddling through (he cites Charles E. Lindblom, just like feeley_judicial_2000!)
  - Risk management (--finally, something familiar to security experts!)
    - So, a broader view of security, inclusive of both public safety and information security
    - In this model, threat actors now include terrorists, CSAM distributors, organized crime
      - Add a new character? Evil Alice and evil Bob (Alex and Becca?)? This automatically raises to question of how to
          identify bad actors--AI (lee_detecting_2020)? PhotoDNA (microsoft_2020)? Government orders (warrants)? Could
          be useful.
    - I have already suggested adding government mandates to the traditional IS threat model
  - Imperfect solutions, like lawful hacking
  - Meanwhile, invest in knowledge production and improved relationships (like "cross-organizational" from
      commission_tackling_2018)
- Avoid
  - Easy answers ("narrow solutions" from commission_tackling_2018)
    - Such as EA mandates or "golden age of surveillance"-based denials

matyas_incommensurability_2018:
- Approaches to reducing incommensurability
  - Reduce gray area between rules and principles (i.e. keep laws up to date with technology)
  - Use transparency and accountability (through technical means)

feeley_judicial_2000:
- Classical analytic method ("linear" in commission_tackling_2018)
% 1. Classical analytic method; each step has its own approaches, e.g. cost-benefit analysis (loosely based on natural
% sciences)
%   (missing: Observe)
%   a. Define the problem (Orient)
%   b. Identify a goal (Orient)
%   c. Generate a range of alternatives for achieving that goal (Orient)
%   d. Select the alternative that seems most promising (Decide)
%   e. Implement the selected alternative (Act)
%   (missing: loop)
- Incremental intuitive decision making ("Muddling through" in rozenshtein_wicked_2018)
- The hermeneutic circle

% How to tackle wicked problems in general? Take some tips from commission_tackling_2018, rozenshtein_wicked_2018,
% feeley_judicial_2000, and matyas_incommensurability_2018:
% Correctly state the problem--identify the root causes. They aren't going away, there by definition is no clear long
% term solution. Put the proposed solutions in context--none are perfect, none are permanent. For EA in particular,
% consider balance of principles and rules.

% Subject one: do we even want EA?
%  --> no? exit here
%  --> yes? carry on
%  --> pragmatist? carry on
% include fallacies argument map here
% see rozenshtein_wicked_2018, benaloh_what_2018, soesanto_2018, ministerial_2018, fbi_2020, brantley_2018, lund_2020
% ... group_2019, ruiz_there_2018, rozenshtein_2019

% Subject two: what properties would good EA have?
% easy for LEAs to use? hard? expensive? voluntary? unable to be used for mass surveillance? transparent (attempts are
%   visible)? alignment of values with security (e.g. discourages bug hoarding)? adaptable (since solution won't be
%   permanent, either technically or according to collective will as things change rozenshtein_wicked_2018)? suited to
%   rules and principles (matyas_incommensurability_2018)?

% Subject three: what options do we have?
% map of properties onto proposals--leads into threat modeling, which will check the veracity of claims














%%%%%%%%%%%%%%%%%%%%%%%%%%%%%%%%%%%%%%%%%%%%%%%%%%%%%%%%%%%%%%%%%%%%%%%%%%%%%%%%%%%%%%%%%%%%%%%%

% In this chapter, develop the arguments employed in EA debates. Make use of argument maps to break down and visualize the
% debate, as shown in \myfig{fig-args-top-level}.

% \begin{figure}[h]
%     \centering\CaptionFontSize
%     \includegraphics[width=\linewidth]{arguments/build/top-level.pdf}
%     \caption[Argument: Top Level]{Top level arguments for and against exceptional access.}
%     \label{fig-args-top-level}
% \end{figure}

% The debate over exceptional access takes place on two fundamental levels:
% \begin{enumerate}
%     \item Whether cryptosystems with exceptional access represent a net benefit to society
%     \item Whether and how to conduct exceptional access research
% \end{enumerate}

% \subsection{Arguments For}

% Why EA is a benefit to society:
% \begin{itemize}
%     \item Absolute privacy is not a right
%     \item Encryption gives too much power to the individual
%     \item EA would allow more crimes to be prevented and prosecuted
% \end{itemize}

% Why to research EA:
% \begin{itemize}
%     \item EA done well would be more secure and privacy preserving than EA done poorly
%     \item The threat of legislative action creates urgency
% \end{itemize}

% \subsection{Arguments Against}

% Why EA is a danger to society:
% \begin{itemize}
%     \item Privacy is an absolute right
%     \item Exceptional access gives too much power to the state
%     \item Exceptional access is not possible in an acceptably secure manner
% \end{itemize}

% Why not to research EA:
% \begin{itemize}
%     \item With both security and privacy in a poor state, research should be focused on strengthening, not weakening,
%             them
%     \item The technical community should not entertain delusions about what is truly possible
% \end{itemize}

% \subsection{Fallacious Arguments}

% Fallacy definitions come from the helpful University of North Carolina Writing Center \cite{unc_2020}. Also cited:
% \cite{hanna_2019}.

% Fallacies are as shown in \myfig{fig-args-fallacies}.

% \begin{figure}[h]
%     \centering\CaptionFontSize
%     \myincludeargument{comprehensive.fallacies}
%     \caption[Fallacious Arguments]{Fallacious arguments used in the exceptional access debate.}
%     \label{fig-args-fallacies}
% \end{figure}
