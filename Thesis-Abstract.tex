
Public debate has resumed over whether encryption systems should support alternative means of decryption intended for
law enforcement use, called exceptional access (EA). Rather than a renege on a resolute promise made at the end of the
1990s ``crypto war,'' this represents a valid reassessment of optimal policy in light of changing circumstances.
Achieving proper balance between privacy and access in the context of constantly changing society and technology is a
wicked problem that has and will evade a permanent solution. As policymakers consider next steps, it behooves the
technical community to stay engaged. Although the introduction of EA would inevitably introduce risk, the quality of the
technical and regulatory approach can make a substantial difference. Furthermore, if one considers hard-line legislative
action and malicious abuse of cryptosystems as part of the threat model, well-designed EA may reduce overall risk.

The root of the conflict lies in cryptography's dual role as enabler of unprecedented privacy and cornerstone of
security. The emergence of strong encryption triggered the first crypto war, and its proliferation is causing the
second. In response to both polarized and conciliatory voices, I analyze strategies that do and do not work on wicked
problems and promote an iterative approach to the case of encryption and EA. Along the way I illustrate the components
of the debate in argument maps and demonstrate the security risks with data flow diagrams and threat analysis, focusing
on one EA proposal in particular, Stefan Savage's ``Lawful Device Access without Mass Surveillance Risk.''
