
Public debate has resumed on the topic of exceptional access (EA), which refers to alternative means of decryption
intended for law enforcement use. The resumption of this debate is not a renege on a resolute promise made at the end of
the 1990s ``crypto war''; rather, it represents a valid reassessment of optimal policy in light of changing
circumstances. The imbalance between privacy, access, and security in the context of constantly changing society and
technology is a wicked problem that has and will continue to evade a permanent solution. As policymakers consider next
steps, it is necessary that the technical community remain engaged. Although any EA framework would increase risk, the
magnitude of that increase varies greatly with the quality of the technical and regulatory approach. Furthermore, if one
considers hard-line legislative action and malicious abuse of cryptosystems as part of the threat model, well-designed
EA may reduce risk overall.

The root of the conflict lies in cryptography's dual role as an enabler of unprecedented privacy and a cornerstone of
security. The emergence of strong encryption incited the first crypto war, and its proliferation is causing the second.
In response to both polarized and conciliatory voices, this paper analyzes strategies for confronting wicked problems
and proposes an iterative approach to the case of encryption and EA. Along the way, it illustrates the components of the
debate in argument maps and demonstrate the security risks with data flow diagrams and threat analysis, focusing on one
EA proposal in particular, Stefan Savage's ``Lawful Device Access without Mass Surveillance Risk.''
